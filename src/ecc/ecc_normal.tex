\section{Strong Normalization}


From the quasinormalization theorem~\ref{quasinormalization} we get the
following result: All valid types $T$ can be reduced to

\begin{itemize}

\item a syntactical kind $\Pi \vec x^{\vec A}. s$

\item or a syntactical type $\Pi \vec x^{\vec A}. y \vec a$

\end{itemize}
%
where $A$ is either a syntactical kind or a syntactical type.


\subsection{Constructor Interpretation}



\begin{definition}
    \emph{Base Terms} The set of base terms $\baseterms$ is defined by the
    grammar
    $$
    \begin{array}{lll}
        b &::=& x
        \\
        &\mid& b a
    \end{array}
    $$
    where $b$ ranges over base terms, $x$ over variables and $a$ over any terms.
    I.e. a base term is a variable applied to zero or more arbitrary arguments.

    Base terms have the property that beta reduction does not change the
    structure. Beta reduction of a base term reduces exactly one of the
    arguments.
\end{definition}



\begin{definition}
    \emph{Key Reduction} A key reduction is the binary relation $\keyreduce$
    defined by the rules
    \begin{enumerate}
    \item
        $$
            (\lambda x^A. e) a \keyreduce e[x:=a]
        $$

    \item
        $$
        \rulev {
            a \keyreduce b
        }
        {
            a c \keyreduce b c
        }
        $$
    \end{enumerate}
    A key reduction reduces a redex only in a key position i.e. the redex has to
    be the leftmost subterm of the term.
\end{definition}





\begin{definition}
    \emph{Lambda Function Space}
    Let $A$ and $B$ two sets of lambda terms. The lambda function space $A
    \tolambda B$ is defined as
    $$
    A \tolambda B := \set{f: \forall a. a \in A \imp fa \in B}
    $$
\end{definition}





\begin{definition}
    \emph{Strongly Normalizing Terms}
    The set of strongly normalizing terms $\SN$ is defined inductively by the
    rule
    $$
    \ruleh{
        \forall b. \ruleh{a \reduce b}{b \in \SN}
    }
    {
        a \in \SN
    }
    $$
\end{definition}


\begin{definition}
    \emph{Saturated Sets} A saturated set $X$ is a set of strongly normalizing terms
    which satisfy the following closure conditions
    \begin{enumerate}
    \item $X$ contains all strongly normalizing base terms
        $$
            \rulev{b \in \baseterms \\ b \in \SN}{b \in X}
        $$

    \item $X$ contains all strongly normalizing key redexes if the key reduct is
        in $X$
        $$
        \rulev {
            b \in X
            \\
            a \in \SN
            \\
            a \keyreduce b
        }
        {
            a \in X
        }
        $$
    \end{enumerate}
\end{definition}


\begin{theorem}
    \emph{The saturated sets are closed under arbitrary intersections}
    Let $S \subseteq \satsets$ be a set of saturated sets. Then the intersection
    $$
    \bigcap S := \set{t: \forall A \in S \imp t \in A}
    $$
    is saturated as well
    $$
    \bigcap S \in \satsets
    $$
    \begin{proof}
        We have to proof the following closure conditions of $\bigcap S$:

        \begin{enumerate}
        \item All strongly normalizing base terms are in $\bigcap S$:

            Since all strongly normalizing base terms are in each element of $S$
                they are certainly in the intersection as well.

        \item If key reduct is in $\bigcap S$ then its strongly normalizing key
            redex is in $\bigcap S$ as well:


            Let's assume $b$ is in all sets of $S$ and $a \keyreduce b$ and $a$
                is strongly normalizing. Then by definition $a$ is in all sets
                of $S$. Therefore it is the intersection $\bigcap S$ as well.
        \end{enumerate}
    \end{proof}
\end{theorem}


\begin{definition} \emph{Syntactical Kinds and Types} Let range $s$ over sorts,
    $b$ over base terms, $T$ over syntactical types and $K$ over syntactical
    kinds. Then syntactical kinds and types are defined by the mutually
    recursive grammar
    $$
    \begin{array}{lll}
        K &::=& s
        \\
        &\mid& \Pi x^T. K
        \\
        &\mid& \Pi x^K. K
        \\
        T &::=& b
        \\
        &\mid& \Pi x^T. T
        \\
        &\mid& \Pi x^K. T
    \end{array}
    $$

    We use $\skinds$ to describe the set of syntactical kinds and $\stypes$ to
    describe the set of syntactical types.
\end{definition}



\begin{definition}
    \emph{Kinds}
    The set of kinds $\skinds^*$ is the set of all terms which
    reduce to a syntactical kind in zero or more steps.
    $$
    \skinds^* := \set{A: \exists K \in \skinds. A \reducestar K}
    $$
\end{definition}





\begin{definition}
    \emph{Proper Types}
    The set of proper types $\stypes^*$ is the set of all
    terms which reduce to a syntactical type in zero or more steps.
    $$
    \stypes^* := \set{A: \exists T \in \stypes. A \reducestar T}
    $$
\end{definition}





\begin{definition}
    \emph{Constructors}
    The set of constructors $\constructors_\Gamma$ in the
    context $\Gamma$ is defined as
    $$
    \constructors_\Gamma := \set{A: \exists K\in\skinds^*. \Gamma \vdash A: K}
    $$
\end{definition}




\begin{definition}
    \emph{Model Sets} A model set is either the set of saturated sets or a
    function from a model set to a model set. Let $M$ range over model sets,
    then the model sets are described by the grammar
    $$
    \begin{array}{lll}
        M &::=& \satsets
        \\
        &\mid& M \to M
    \end{array}
    $$
    We use $\satsets^*$ to describe the set of model sets.

    We can map each syntactical kind to a set of models. We
    define a function $\nu: \skinds \to \satsets^*$.
    $$
    \nu K :=
    \begin{cases}
        \nu s := \satsets
        %
        \\

        \nu (\Pi x^A. B) := \nu A \to \nu B & \mbox{if } A \in \skinds
        %
        \\
        %
        \nu (\Pi x^A. B) := \nu B & \mbox{if } A \in \stypes
    \end{cases}
    $$
\end{definition}

The reduction of a type to a syntactical kind or type is not unique. E.g. the
type $A$ might reduce to the kinds $K_1$ and $K_2$ i.e. $A \reducestar K_1$ and
$A \reducestar K_2$. According to the Church Rosser theorem there is a term $K$
with $K_1 \reducestar K$ and $K_2 \reducestar K$. $K$ is a syntactical kind,
since reduction preserves syntactical kinds and types.

The definition of the function $\nu$ guarantees $K_i \reducestar K \imp \nu K_i
= \nu K$ which implies $\nu K_1 = \nu K_2$. Therefore we an extend the function
$\nu$ to all terms $A$ which reduce to a syntactical kind
$$
\nu : \skinds^* \to \satsets^*
$$


\begin{definition}
    \emph{Interpretation of Constructor Variables} An interpretation $\xi$ of
    constructor variable is generated by the grammar
    $$
    \begin{array}{lll}
        \xi &::=& []
        \\
        &\mid& \xi, x^S
    \end{array}
    $$
    where $S \in \satsets^*$.

    A valid interpretation of constructor variables $\xi$ for the
    context $\Gamma$ written as $\xi \vDash \Gamma$ is an inductive relation
    defined by the rules
    \begin{enumerate}
    \item
        $$
            [] \vDash []
        $$

    \item
        $$
        \rulev {
            \xi \vDash \Gamma
            \\
            \Gamma \vdash A: s
            \\
            A \in \stypes^*
        }
        {
            \xi \vDash \Gamma, x^A
        }
        $$
    \item
        $$
        \rulev {
            \xi \vDash \Gamma
            \\
            \Gamma \vdash A: s
            \\
            A \in \skinds^*
            \\
            S \in \nu A
        }
        {
            \xi, x^S \vDash \Gamma, x^A
        }
        $$
    \end{enumerate}
\end{definition}


\begin{definition}
    \emph{Constructor Interpretation}  A constructor interpretation
    $\typeinter{-}{\xi \Gamma}$ is a map from valid constructors $C$ in a valid
    context $\Gamma$ with the interpretation $\xi$ of the constructor variables
    into $\satsets^*$.
    $$
    \rulev{
        \xi \vDash \Gamma
        \\
        \Gamma \vdash C: K
        \\
        K \in \skinds^*
    }
    {
        \typeinter{C}{\xi \Gamma} \in \nu K
    }
    $$
    We define the map $\typeinter{A}{\xi \Gamma}$ recursively over the
    construction of $C$.
    \begin{enumerate}
    \item $C$ is the sort $\Any_i$ where $-1 \le i$:
        $$
            \rulev{
                \xi \vDash \Gamma
                \\
                \Gamma \vdash \Any_i: \Any_{i+1}
                \\
                \Any_{i+1} \in \skinds
            }
            {
                \typeinter{\Any_i}{\xi\Gamma} := \SN
            }
        $$
        The postcondition $\SN \in \nu \Any_{i+1}$ is satisfied, because $\nu
        \Any_{i+1} = \satsets^*$ and $\SN \in \satsets^*$.

    \item $C$ is the variable $x$:
        $$
            \rulev{
                \xi \vDash \Gamma
                \\
                \Gamma \vdash x: K
                \\
                K \in \skinds^*
            }
            {
                \typeinter{x}{\xi\Gamma} := \xi x
            }
        $$
        Proof of the postcondition $\xi x \in \nu K$:

        The condition $\Gamma \vdash x:K$ implies $\Gamma x \le K$ and $\Gamma x
        \in \skinds^*$. Therefore by definition $\xi x \in \nu (\Gamma x)$.
        Since $\Gamma x$ is a subtype of $K$, $\nu (\Gamma x) = \nu K$.

    \item $C$ is the product $\Pi x^A. B$:

        We have to distinguish the cases $A \in \stypes^*$ and $A \in \skinds^*$.

        \begin{enumerate}
        \item $A$ is a proper type $A \in \stypes^*$:
        $$
            \rulev{
                \xi \vDash \Gamma
                \\
                \Gamma \vdash \Pi x^A.B : K
                \\
                K \in \skinds^*
            }
            {
                \typeinter{\Pi x^A. B}{\xi\Gamma}
                :=
                \typeinter{A}{\xi\Gamma}
                \tolambda
                \typeinter{B}{\xi(\Gamma,x^A)}
            }
        $$
        From $\Gamma \vdash \Pi x^A.B : K$ we infer the existence of two sorts
                $s_A$ and $s_B$ such that $\Gamma \vdash A: s_A$ and $\Gamma,x^A
                \vdash B: s_B$ are valid. $x$ is not a constructor since $A \in
                \stypes^*$. Therefore the typeinterpretations of $A$ and $B$ are
                valid and we have $\typeinter{A}{\xi\Gamma} \in \nu s_A$ and
                $\typeinter{B}{\xi(\Gamma,x^A)} \in \nu s_B$ and $\nu s_A = \nu
                s_B = \satsets$.

        The typeinterpretations of $A$ and $B$ are saturated sets i.e. set of
                lambda terms and therefore valid arguments for the lambda
                function space function $\tolambda$. The resulting set of lambda
                terms is a saturated set as well, because saturated sets are
                closed under lambda function space construction.

        Since $K$ must be beta equivalent to some sort we have $\nu K =
                \satsets$ and therefore the postcondition of the function is
                satisfied.

        \item $A$ is a kind $A \in \skinds^*$:
            $$
            \rulev{
                \xi \vDash \Gamma
                \\
                \Gamma \vdash \Pi x^A.B : K
                \\
                K \in \skinds^*
            }
            {
                \typeinter{\Pi x^A. B}{\xi\Gamma}
                :=
                \typeinter{A}{\xi\Gamma}
                \tolambda
                \bigcap_{S \in \nu A}
                \typeinter{B}{(\xi,x^S)(\Gamma,x^A)}
            }
            $$
            The validity of the type interpretations of $A$ and $B$ follow from the
            same arguments as in the previous case. However since the
            variable $x$ is now a constructor variable the type
            interpretation of $B$ needs an interpretation of the constructor
            variable as well i.e. it needs the interpretation $\xi, x^S$
            where $S \in \nu A$.

            Saturated sets are closed under arbitrary intersections. Therefore the
            intersection over all possible interpretations of the constructor
            variable $x$ is a saturated set as well.

            The final lambda function space construction results in a saturated set
            i.e. the postcondition of the function is satisfied.
        \end{enumerate}

    \item $C$ is the abstraction $\lambda x^A. e$:

        We have to distinguish the cases $A \in \stypes^*$ and $A \in \skinds^*$.

        \begin{enumerate}
        \item $A$ is a proper type $A \in \stypes^*$:
            $$
            \rulev{
                \xi \vDash \Gamma
                \\
                \Gamma \vdash \lambda x^A.e : K
                \\
                K \in \skinds^*
            }
            {
                \typeinter{\lambda x^A. e}{\xi\Gamma}
                :=
                \typeinter{e}{\xi (\Gamma,x^A)}
            }
            $$

            From the generation lemma~\ref{GenerationLemma} we infer the
            existence of a sort $s$ and a type $B$ such that
            $$
            \begin{array}{lll}
                \Gamma &\vdash& \Pi x^A. B: s
                \\
                \Gamma,x^A &\vdash& e : B
                \\
                \Pi x^A. B &\le& K
            \end{array}
            $$
            are valid. $B \in \skinds^*$, because $\Pi x^A.B \le K$ and $K \in
            \skinds^*$.

            Since $A$ is a proper type,
            $\typeinter{e}{\xi (\Gamma,x^A)} \in \nu B$
            is a valid type interpretation.

            By definition $\nu (\Pi x^A. B) = \nu B$. Therefore
            $\typeinter{e}{\xi (\Gamma,x^)}$ satisfies the postcondition.

        \item $A$ is a kind $A \in \skinds^*$:
            $$
            \rulev{
                \xi \vDash \Gamma
                \\
                \Gamma \vdash \lambda x^A.e : K
                \\
                K \in \skinds^*
            }
            {
                \typeinter{\lambda x^A. e}{\xi\Gamma}
                :=
                (S \in \nu A) \mapsto
                \typeinter{e}{(\xi,x^S) (\Gamma,x^A)}
            }
            $$
            where $(x \in T) \mapsto e$ is a function mapping an element $t \in
            T$ to the value $e[x:=t]$

            By the same reasoning as in the previous case we infer the validity
            of
            $$
            \begin{array}{lll}
                \Gamma &\vdash& \Pi x^A. B: s
                \\
                \Gamma,x^A &\vdash& e : B
                \\
                \Pi x^A. B &\le& K
                \\
                B &\in& \skinds^*
            \end{array}
            $$
            However $A$ is now a kind and therefore $x$ is a constructor
            variable. We need the interpretation of constructor variables
            $\xi,x^S$ where $S \in \nu A$ to generate a valid type
            interpretation of $e$.

            The postcondition requires the result to be in $nu K$ which is the
            same as $\nu (\Pi x^A.B)$ which by definition is $\nu A \to \nu B$.
            Obviously the result term satisfies the postcondition.

        \end{enumerate}

    \item $C$ is the application $f a$:
        There are the preconditions
        $$
        \begin{array}{lll}
            \xi &\vDash& \Gamma
            \\
            \Gamma &\vdash& f a: K
            \\
            K &\in& \skinds^*
        \end{array}
        $$
        By the generation lemma~\ref{GenerationLemma} there exist the terms $A$
        and $B$ such that
        $$
        \begin{array}{lll}
            \Gamma &\vdash& f: \Pi x^A. B
            \\
            \Gamma &\vdash& a: A
            \\
            B[x:=a] &\le& K
        \end{array}
        $$
        are valid.

        {\bf Why this cannot work}:
        By the typing rule for applications we get the following
        $$
        \rulev{
            [f^{\Pi x^{\Any_{i+1}}. x}] \vdash f: \Pi x^{\Any_{i+1}}. x
            \\
            {[}f^{\Pi x^{\Any_{i+1}}. x}] \vdash \Any_i : \Any_{i+1}
        }
        {
            f \Any_i:
            \underbrace{x[x:=\Any_i]}_{= \Any_i}
        }
        $$
        Problem:
        \begin{itemize}
        \item $f \Any_i$ is a a constructor because its type is a syntactical
            kind.

        \item $f$ is not a constructor because its type is a syntactical type.

        \end{itemize}

        In the calculus of constructions this problem cannot happen. There are
        only the two sorts $ * $ and
        $\Box$. It is not possible to construct the
        type $\Pi x^\Box. x$ because $\Box: s$ is invalid.
    \end{enumerate}
\end{definition}
