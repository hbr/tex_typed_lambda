\section{Typing}


\subsection{Substitution Lemma}
%%%%%%%%%%%%%%%%%%%%%%%%%%%%%%%%%%%%%%%%%%%%%%%%%%%%%%%%%%%%%%%%%%%%%%%%%%%%%%%%



\begin{theorem}
    \label{substitution-theorem}
    Substitution (cut) theorem.
    $$
    \rulev{
        \Gamma \vdash a: A
        \\
        \Gamma, x^A, \Delta \vdash t: T
    }
    {
        \Gamma, \Delta[x:=a] \vdash t[x:=a]: T[x:=a]
    }
    $$

    \begin{proof}
        We assume
        $\Gamma \vdash a: A$
        and prove this theorem by induction on
        $\Gamma, x^A, \Delta \vdash t: T$.

        In order to express the proof more compactly we introduce the
        abbreviation $t' := t[x:=a]$.

        \begin{enumerate}
            \item Axiom: This case is syntactically impossible, because the
                context is not empty.

            \item Variable: We have to prove the goal
                $$
                \begin{array}{l|l}
                    \Gamma, x^A, \Delta \vdash B: s
                    &
                    \Gamma, \Delta' \vdash B': s
                    \\
                    \hline
                    \Gamma, x^A, \Delta, y^B \vdash y: B
                    &
                    \Gamma, \Delta', y^{B'} \vdash y: B'
                \end{array}
                $$

                The final goal in the lower right corner follows directly from
                the induction hypothesis and the variable introduction rule.

            \item Product:
                $$
                \begin{array}{l|l}
                    \Gamma, x^A, \Delta \vdash B: s_1
                    &
                    \Gamma, \Delta' \vdash B': s_1
                    \\
                    \Gamma, x^A, \Delta, y^B \vdash C: s_2
                    &
                    \Gamma, \Delta', y^{B'} \vdash C': s_2
                    \\
                    s_2 = \Prop \lor s_1 = s2
                    \\
                    \hline
                    \Gamma, x^A, \Delta \vdash \Pi y^B. C : s_2
                    &
                    \Gamma, \Delta' \vdash \Pi y^{B'}. C': s_2
                \end{array}
                $$

                The final goal follows from the induction hypotheses and the
                product introduction rule.

            \item Abstraction:
                $$
                \begin{array}{l|l}
                    \Gamma, x^A, \Delta \vdash \Pi y^B. C: s
                    &
                    \Gamma, \Delta' \vdash \Pi y^{B'}. C': s
                    \\
                    \Gamma, x^A, \Delta, y^B \vdash e: C
                    &
                    \Gamma, \Delta', y^{B'} \vdash e': C'
                    \\
                    \hline
                    \Gamma, x^A, \Delta \vdash \lambda y^B. e: \Pi y^B. C
                    &
                    \Gamma, \Delta' \vdash \lambda y^{B'}. e' : \Pi y^{B'}. C'
                \end{array}
                $$

                Same reasoning as above.

            \item Application:
                $$
                \begin{array}{l|l}
                    \Gamma, x^A, \Delta \vdash f: \Pi y^B. C
                    &
                    \Gamma, \Delta' \vdash f' : \Pi y^{B'}. C'
                    \\
                    \Gamma, x^A, \Delta \vdash b: B
                    &
                    \Gamma, \Delta' \vdash b': B'
                    \\
                    \hline
                    \Gamma, x^A, \Delta \vdash f b: C[y:=b]
                    &
                    \Gamma, \Delta' \vdash f' b': C[y:=b]'
                \end{array}
                $$

                From the induction hypotheses and the application introduction
                rule we conclude
                $$
                    \Gamma, \Delta' \vdash f' b': C'[y:=b']
                $$
                and get the final goal by observing
                $$
                    C'[y:=b'] = C[y:=b]'
                $$
                by using the double substition lemma~\ref{DoubleSubstitution}.

            \item Structural rules: For all structural rules we prove the final
                goal by the induction hypothesis and by the facts
                $$
                \begin{array}{lll}
                    t \reduce u &\imp& t' \reduce u'
                    \\
                    A \prec B    &\imp& A' \prec B'
                \end{array}
                $$

        \end{enumerate}
    \end{proof}
\end{theorem}



\subsection{Subject Reduction Lemma}

\begin{theorem}
    \label{subject-reduction-theorem}
    Subject reduction theorem.
    $$
    \rulev{
        \Gamma \vdash t: T
        \\
        t \reduce u
    }
    {
        \Gamma \vdash u: T
    }
    $$

    \begin{proof} MISSING
    \end{proof}
\end{theorem}





\subsection{Generation Lemmata}
%%%%%%%%%%%%%%%%%%%%%%%%%%%%%%%%%%%%%%%%%%%%%%%%%%%%%%%%%%%%%%%%%%%%%%%%%%%%%%%%



\begin{theorem}
    The following generation lemmata are valid
    \begin{enumerate}
    \item
        $$
        \ruleh {
            \Gamma \vdash s_1 : T
        }
        {
            \exists
                s_2.
                \left[
                \begin{array}{l}
                    \Gamma \vdash s_1: s_2
                    \\
                    s_2 \le T
                \end{array}
                \right]
        }
        $$

    \item
        $$
        \ruleh {
            \Gamma \vdash x : T
        }
        {
            \Gamma x \le T
        }
        $$
        where $\Gamma x$ is the type of $x$ in the context $\Gamma$.

    \item % product
        $$
        \ruleh {
            \Gamma \vdash (\Pi x^A. B) : T
        }
        {
            \exists
                s_1, s_2.
                \left[
                \begin{array}{l}
                    \Gamma \vdash A: s_1
                    \\
                    \Gamma, x^A \vdash B: s_2
                    \\
                    (s_2 = \Prop \lor s_1 = s_2) \land s_2 \le T
                \end{array}
                \right]
        }
        $$

    \item % abstraction
        $$
        \ruleh {
            \Gamma \vdash (\lambda x^A. e) : T
        }
        {
            \exists
                B, s.
                \left[
                \begin{array}{l}
                    \Gamma \vdash (\Pi x^A. B): s
                    \\
                    \Gamma, x^A \vdash e: B
                    \\
                    \Pi x^A.B \le T
                \end{array}
                \right]
        }
        $$

    \item % application
        $$
        \ruleh {
            \Gamma \vdash f a : T
        }
        {
            \exists
                A, B.
                \left[
                \begin{array}{l}
                    \Gamma \vdash f: \Pi x^A. B
                    \\
                    \Gamma \vdash a: A
                    \\
                    B[x:=a] \le T
                \end{array}
                \right]
        }
        $$
    \end{enumerate}

    \begin{proof}
        Premise in all lemmata is the validity of $\Gamma \vdash t: T$ where $t$
        is either a sort, a variable, an application, a product or an
        abstraction. The proof is done by induction on $\Gamma \vdash t: T$.

        Only one of the introduction rules is valid. For the corresponding
        introduction rule the proof of the goal is trivial.

        The weakening rule modifies neither the term $t$ nor its type $T$.
        Therefore the goal is proved by the corresponding induction hypothesis.

        The reduction rule modifies the type from the type $T$ to another type
        $U$ with $T \equiv U$. Since $T \equiv U$ implies $T \le U$ the goal can
        be concluded from the induction hypothesis.

        The subtype rule modifies the type from the type $T$ to another type $U$
        with $T \prec U$. Since $T \prec U$ implies $T \le U$ the goal can be
        concluded from the induction hypothesis.
    \end{proof}
\end{theorem}



\subsection{Uninhabited Types}
%%%%%%%%%%%%%%%%%%%%%%%%%%%%%%%%%%%%%%%%%%%%%%%%%%%%%%%%%%%%%%%%%%%%%%%%%%%%%%%%



There are types which cannot be inhabited by a term in normal form in the empty
context.

\begin{theorem}
    \emph{The type $\Pi X^s. X$ cannot be inhabited by a term in normal form
    (i.e. a term with no redexes) in the empty context.}
    $$
    \ruleh {
        [] \vdash t: \Pi X^s. X
        \\
        t \text{ is in normal form}
    }
    {
        \perp
    }
    $$

    \begin{proof}
        We prove this theorem by induction on the structure of $t$.
        \begin{enumerate}
        \item $t$ is the sort $s_1$:

            By the generation lemma we postulate the existence of a sort $s_2$
                with $[] \vdash s_1: s_2 \land s_2 \le \Pi X^s. X$. This is
                contradictory because $s_2 \le \Pi X^s.X$ is imposible.

        \item $t$ is a variable:
            This is impossible, because the context is empty.

        \item $t$ is a product:
            This is impossible. The type of a product is always a sort and never
                a product.

        \item $t$ is an application:

            If $t$ is an application it must have the form $h a_0 a_1 \ldots$
                where the head term $h$ is neither an application nor a sort nor
                a product. Therefore the head term must be either a variable or
                an abstraction. It cannot be a variable, because the context is
                empty. It cannot be an abstraction, because the term is in
                normal form and therefore cannot have the redex $h a_0$.

        \item $t$ is the abstraction $\lambda x^A. e$:

            By the generation lemma we postulate the existence of $s_2$ and $B$
                with
                $$
                \begin{array}{l}
                    [] \vdash \Pi x^A. B: s_2
                    \\
                    \text{$[x^A]$} \vdash e: B
                    \\
                    \Pi x^A. B \le \Pi X^s. X
                \end{array}
                $$

                Because of the last statement we conclude $x = X$, $A \equiv s$
                and $B \le X$ where the latter implies $B = X$. I.e. we have
                lead the existence of $e$ with $[X^A] \vdash e: X$ to a
                contradiction.

                We prove this claim by induction on the structure of $e$.
                \begin{enumerate}
                \item $e$ is a sort:

                    This is impossible, because the type of a sort cannot be a
                        variable.

                \item $e$ is a variable: The only variable in the context is
                    $X$. By the generation lemma we know that $A \le X$. By
                        definition of $\le$ this is possible only if $A \equiv
                        X$ which implies $X \equiv s$ which is contradictory.
                \end{enumerate}
        \end{enumerate}
    \end{proof}
\end{theorem}
