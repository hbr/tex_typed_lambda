\section{Typing}









\subsection{Typing Relation}
%%%%%%%%%%%%%%%%%%%%%%%%%%%%%%%%%%%%%%%%%%%%%%%%%%%%%%%%%%%%%%%%%%%%%%%%%%%%%%%%


\begin{definition}
The ternary \emph{typing relation} $\Gamma \vdash t: T$ which says that in
the context $\Gamma$ then term $t$ has type $T$ is defined inductively by
the rules
\begin{enumerate}
    \item Introduction rules:
    \begin{enumerate}
        \item Axiom:
            $$
            \ruleh
            {i \in \{-1, 0, 1, 2, 3, \ldots \}}
            { [] \vdash \Any_i : \Any_{i + 1}}
            $$

        \item Variable:
            $$
            \rulev {
                \Gamma \vdash A: s
                \\
                x \notin \Gamma
            }
            {
                \Gamma, x^A \vdash x: A
            }
            $$

        \item Product:
            $$
            \rulev {
                \Gamma \vdash A: s_1
                \\
                \Gamma, x^A \vdash B: s_2
                \\
                s_1 = s_2 \lor s_1 = \Prop
            }
            {
                \Gamma \vdash \Pi x^A.B : s_2
            }
            $$

        \item Abstraction:
            $$
            \rulev {
                \Gamma \vdash \Pi x^A. B: s
                \\
                \Gamma, x^A \vdash e: B
            }
            {
                \Gamma \vdash (\lambda x^A. e): \Pi x^A. B
            }
            $$

        \item Application:
            $$
            \rulev {
                \Gamma \vdash f: \Pi x^A. B
                \\
                \Gamma \vdash a: A
            }
            {
                \Gamma \vdash f a: B[x:=a]
            }
            $$
    \end{enumerate}


    \item Structural rules:
        \begin{enumerate}
            \item Weaken:
                $$
                \rulev {
                    \Gamma \vdash t: T
                    \\
                    \Gamma \vdash A: s
                    \\
                    x \notin \Gamma
                }
                {
                    \Gamma, x^A \vdash t: T
                }
                $$

            \item Reduction:
                $$
                \rulev {
                    \Gamma \vdash t: T
                    \\
                    \Gamma \vdash U: s
                    \\
                    T \reduce U
                }
                {
                    \Gamma \vdash t: U
                }
                $$

            \item Expansion:
                $$
                \rulev {
                    \Gamma \vdash t: T
                    \\
                    \Gamma \vdash U: s
                    \\
                    U \reducestar T
                }
                {
                    \Gamma \vdash t: U
                }
                $$

            \item Subtype:
                $$
                \rulev {
                    \Gamma \vdash t: T
                    \\
                    \Gamma \vdash U: s
                    \\
                    T \prec_i U
                }
                {
                    \Gamma \vdash t: U
                }
                $$
        \end{enumerate}
    \end{enumerate}
\end{definition}






\subsection{Substitution Lemma}
%%%%%%%%%%%%%%%%%%%%%%%%%%%%%%%%%%%%%%%%%%%%%%%%%%%%%%%%%%%%%%%%%%%%%%%%%%%%%%%%



\begin{theorem}
    \label{SubstitutionLemma}
    Substitution (cut) theorem.
    $$
    \rulev{
        \Gamma \vdash a: A
        \\
        \Gamma, x^A, \Delta \vdash t: T
    }
    {
        \Gamma, \Delta[x:=a] \vdash t[x:=a]: T[x:=a]
    }
    $$

    \begin{proof}
        We assume
        $\Gamma \vdash a: A$
        and prove this theorem by induction on
        $\Gamma, x^A, \Delta \vdash t: T$.

        In order to express the proof more compactly we introduce the
        abbreviation $t' := t[x:=a]$.

        \begin{enumerate}
            \item Axiom: This case is syntactically impossible, because the
                context is not empty.

            \item Variable: We have to prove the goal
                $$
                \begin{array}{l|l}
                    \Gamma, x^A, \Delta \vdash B: s
                    &
                    \Gamma, \Delta' \vdash B': s
                    \\
                    \hline
                    \Gamma, x^A, \Delta, y^B \vdash y: B
                    &
                    \Gamma, \Delta', y^{B'} \vdash y: B'
                \end{array}
                $$

                The final goal in the lower right corner follows directly from
                the induction hypothesis and the variable introduction rule.

            \item Product:
                $$
                \begin{array}{l|l}
                    \Gamma, x^A, \Delta \vdash B: s_1
                    &
                    \Gamma, \Delta' \vdash B': s_1
                    \\
                    \Gamma, x^A, \Delta, y^B \vdash C: s_2
                    &
                    \Gamma, \Delta', y^{B'} \vdash C': s_2
                    \\
                    s_2 = \Prop \lor s_1 = s2
                    \\
                    \hline
                    \Gamma, x^A, \Delta \vdash \Pi y^B. C : s_2
                    &
                    \Gamma, \Delta' \vdash \Pi y^{B'}. C': s_2
                \end{array}
                $$

                The final goal follows from the induction hypotheses and the
                product introduction rule.

            \item Abstraction:
                $$
                \begin{array}{l|l}
                    \Gamma, x^A, \Delta \vdash \Pi y^B. C: s
                    &
                    \Gamma, \Delta' \vdash \Pi y^{B'}. C': s
                    \\
                    \Gamma, x^A, \Delta, y^B \vdash e: C
                    &
                    \Gamma, \Delta', y^{B'} \vdash e': C'
                    \\
                    \hline
                    \Gamma, x^A, \Delta \vdash \lambda y^B. e: \Pi y^B. C
                    &
                    \Gamma, \Delta' \vdash \lambda y^{B'}. e' : \Pi y^{B'}. C'
                \end{array}
                $$

                Same reasoning as above.

            \item Application:
                $$
                \begin{array}{l|l}
                    \Gamma, x^A, \Delta \vdash f: \Pi y^B. C
                    &
                    \Gamma, \Delta' \vdash f' : \Pi y^{B'}. C'
                    \\
                    \Gamma, x^A, \Delta \vdash b: B
                    &
                    \Gamma, \Delta' \vdash b': B'
                    \\
                    \hline
                    \Gamma, x^A, \Delta \vdash f b: C[y:=b]
                    &
                    \Gamma, \Delta' \vdash f' b': C[y:=b]'
                \end{array}
                $$

                From the induction hypotheses and the application introduction
                rule we conclude
                $$
                    \Gamma, \Delta' \vdash f' b': C'[y:=b']
                $$
                and get the final goal by observing
                $$
                    C'[y:=b'] = C[y:=b]'
                $$
                by using the double substition lemma~\ref{DoubleSubstitution}.

            \item Structural rules: For all structural rules we prove the final
                goal by the induction hypothesis and by the facts
                $$
                \begin{array}{llll}
                    t \reduce u     &\imp& t' \reduce u'
                    &\text{Lemma~\ref{SubstituteReduction}}
                    \\
                    A \prec_i B     &\imp& A' \prec_i B'
                    &\text{Lemma~\ref{PrecedenceSubstitution}}
                \end{array}
                $$

        \end{enumerate}
    \end{proof}
\end{theorem}






\subsection{Generation Lemmata}
%%%%%%%%%%%%%%%%%%%%%%%%%%%%%%%%%%%%%%%%%%%%%%%%%%%%%%%%%%%%%%%%%%%%%%%%%%%%%%%%



\begin{theorem}
    \label{GenerationLemma}
    The following generation lemmata are valid
    \begin{enumerate}
    \item
        $$
        \ruleh {
            \Gamma \vdash s_1 : T
        }
        {
            \exists
                s_2.
                \left[
                \begin{array}{l}
                    \Gamma \vdash s_1: s_2
                    \\
                    s_2 \le T
                \end{array}
                \right]
        }
        $$

    \item
        $$
        \ruleh {
            \Gamma \vdash x : T
        }
        {
            \Gamma x \le T
        }
        $$
        where $\Gamma x$ is the type of $x$ in the context $\Gamma$.

    \item % product
        $$
        \ruleh {
            \Gamma \vdash (\Pi x^A. B) : T
        }
        {
            \exists
                s_1, s_2.
                \left[
                \begin{array}{l}
                    \Gamma \vdash A: s_1
                    \\
                    \Gamma, x^A \vdash B: s_2
                    \\
                    (s_2 = \Prop \lor s_1 = s_2) \land s_2 \le T
                \end{array}
                \right]
        }
        $$

    \item % abstraction
        $$
        \ruleh {
            \Gamma \vdash (\lambda x^A. e) : T
        }
        {
            \exists
                B, s.
                \left[
                \begin{array}{l}
                    \Gamma \vdash (\Pi x^A. B): s
                    \\
                    \Gamma, x^A \vdash e: B
                    \\
                    \Pi x^A.B \le T
                \end{array}
                \right]
        }
        $$

    \item % application
        $$
        \ruleh {
            \Gamma \vdash f a : T
        }
        {
            \exists
                A, B.
                \left[
                \begin{array}{l}
                    \Gamma \vdash f: \Pi x^A. B
                    \\
                    \Gamma \vdash a: A
                    \\
                    B[x:=a] \le T
                \end{array}
                \right]
        }
        $$
    \end{enumerate}

    \begin{proof}
        Premise in all lemmata is the validity of $\Gamma \vdash t: T$ where $t$
        is either a sort, a variable, an application, a product or an
        abstraction. The proof is done by induction on $\Gamma \vdash t: T$.

        Only one of the introduction rules is valid. For the corresponding
        introduction rule the proof of the goal is trivial.

        The weakening rule modifies neither the term $t$ nor its type $T$.
        Therefore the goal is proved by the corresponding induction hypothesis.

        The reduction and expansion rules modify the type from the type $T$ to
        another type $U$ with $T \betaeq U$. Since $T \betaeq U$ implies $T \le U$
        the goal can be concluded from the induction hypothesis and the
        transitivity of $\le$.

        The subtype rule modifies the type from the type $T$ to another type $U$
        with $T \prec_i U$. Since $T \prec_i U$ implies $T \le U$ the goal can be
        concluded from the induction hypothesis and the transitivity of $\le$.
    \end{proof}
\end{theorem}



\subsection{Type of types}

\begin{theorem}
    \label{TypeOfTypes}
    \emph{The type of a type is a sort}
    $$
    \ruleh {
        \Gamma \vdash t: T
    }
    {
        \exists s. \Gamma \vdash T : s
    }
    $$

    \begin{proof}
        By induction of $\Gamma \vdash t : T$.
        \begin{enumerate}
            \item Introduction rules:
            \begin{enumerate}
                \item Axiom
                $$
                \begin{array}{l|l}
                    [] \vdash \Any_i : \Any_{i+1}
                    &
                    \exists s. \Gamma \vdash \Any_{i+1}: s
                \end{array}
                $$
                Use $s = \Any_{i+2}$.

                \item Variable
                $$
                \begin{array}{l|l}
                    \Gamma \vdash A : s
                    \\
                    x \notin \Gamma
                    \\
                    \hline
                    \Gamma, x^A \vdash x: A
                    &
                    \exists s_2. \Gamma,x^A \vdash A: s_2
                \end{array}
                $$
                Use $s_2 = s$ and the weakening rule.

                \item Product
                $$
                \begin{array}{l|l}
                    \Gamma \vdash A : s_1
                    \\
                    \Gamma,x^A \vdash B: s_2
                    \\
                    s_1 = s_2 \lor s_2 = \Prop
                    \\
                    \hline
                    \Gamma \vdash \Pi x^A. B : s_2
                    &
                    \exists s_3. \Gamma \vdash s_2: s_3
                \end{array}
                $$
                For $s_2 = \Any_i$ use $s_3 = \Any_{i+1}$.

                \item Abstraction
                $$
                \begin{array}{l|l}
                    \Gamma \vdash \Pi x^A. B : s_1
                    \\
                    \Gamma,x^A \vdash e: B
                    \\
                    \hline
                    \Gamma \vdash \lambda x^A. e : \Pi x^A. B
                    &
                    \exists s. \Gamma \vdash \Pi x^A. B: s
                \end{array}
                $$
                Use $s = s_1$.

                \item Application
                $$
                \begin{array}{l|l}
                    \Gamma \vdash f: \Pi x^A. B
                    &
                    \exists s_\pi: \Gamma \vdash \Pi x^A. B: s_\pi
                    \\
                    \Gamma \vdash a: A
                    \\
                    \hline
                    \Gamma \vdash f a: B[x:=a]
                    &
                    \exists s. \Gamma \vdash B[x:=a]: s
                \end{array}
                $$
                From the induction hypothesis we get a sort $s_\pi$ which is the
                type of the product. Then we use the generation
                lemma~\ref{GenerationLemma} for the product to get $s_B$ which
                satisfies $\Gamma, x^A \vdash B: s_B$. Finally we use the
                substitution lemma~\ref{SubstitutionLemma} to infer $\Gamma
                \vdash B[x:=a] : s_B$ and use $s = s_B$.
            \end{enumerate}

            \item Structural rules:
            \begin{enumerate}
                \item Weaken:
                $$
                \begin{array}{l|l}
                    \Gamma \vdash t: T
                    &
                    \exists s_T: \Gamma \vdash T: s_T
                    \\
                    \Gamma \vdash A: s_A
                    \\
                    x \notin \Gamma
                    \\
                    \hline
                    \Gamma, x^A \vdash t: T
                    &
                    \exists s. \Gamma, x^A \vdash T: s
                \end{array}
                $$
                Use $s = s_T$ which exists by the induction hypothesis and lift
                    the typing judgement $\Gamma \vdash T : s_T$ using the
                    weakening rule into the context $\Gamma, x^A$.

                \item Reduction:
                $$
                \begin{array}{l|l}
                    \Gamma \vdash t : T
                    \\
                    \Gamma \vdash U: s_U
                    \\
                    T \reduce U
                    \\
                    \hline
                    \Gamma \vdash t: U
                    &
                    \exists s. \Gamma \vdash U: s
                \end{array}
                $$
                Use $s = s_U$.

                \item Expansion:
                $$
                \begin{array}{l|l}
                    \Gamma \vdash t : T
                    \\
                    \Gamma \vdash U: s_U
                    \\
                    U \reducestar T
                    \\
                    \hline
                    \Gamma \vdash t: U
                    &
                    \exists s. \Gamma \vdash U: s
                \end{array}
                $$
                Use $s = s_U$.

                \item Subtype:
                $$
                \begin{array}{l|l}
                    \Gamma \vdash t : T
                    \\
                    \Gamma \vdash U: s_U
                    \\
                    T \prec_i U
                    \\
                    \hline
                    \Gamma \vdash t: U
                    &
                    \exists s. \Gamma \vdash U: s
                \end{array}
                $$
                Use $s = s_U$.
            \end{enumerate}
        \end{enumerate}
    \end{proof}
\end{theorem}





\subsection{Subject Reduction Lemma}

\begin{theorem}
    \label{SubjectReduction}
    \emph{Subject reduction lemma} Reduction of a term does change its type.
    $$
    \rulev{
        \Gamma \vdash t: T
        \\
        t \reduce u
    }
    {
        \Gamma \vdash u: T
    }
    $$

    {
    \def\SRLeftPart#1#2#3#4{
        \left(
        \forall #1.
        \ruleh{#2 \reduce #1}{#3 \vdash #1: #4}
        \right)
    }
    \def\SRRightPart#1#2#3#4{
        \left(
        \forall #1.
        \ruleh{#2 \reduce #1}{#1 \vdash #3: #4}
        \right)
    }

    \begin{proof} In order to prove the subject reduction lemma we prove the
        more general lemma
        $$
        \ruleh{
            \Gamma \vdash t: T
        }
        {
            \SRLeftPart u t \Gamma T
            \land
            \SRRightPart \Delta \Gamma t T
        }
        $$
        where $\Gamma \reduce \Delta$ means that $\Delta$ is $\Gamma$ with one
        of the variable types replaced by a reduced type.

        We prove the more general lemma by induction on $\Gamma \vdash t: T$.
        \begin{enumerate}
            \item Introduction rules:
            \begin{enumerate}
                \item Sort: If $\Gamma$ is empty and $t$ is a sort, then the
                    goal is vacuously true because neither the empty context nor
                    a sort can reduce to anything (they are in normal form).

                \item Variable:
                    $$
                    \begin{array}{l|l}
                        \Gamma \vdash A : s
                        &
                        \SRLeftPart B A \Gamma s
                        \land
                        \SRRightPart {\Delta_0} \Gamma A s
                        \\
                        \hline
                        \Gamma, x^A \vdash x: A
                        &
                        \SRLeftPart u x {\Gamma,x^A} A
                        \land
                        \SRRightPart \Delta {\Gamma,x^A} x A
                    \end{array}
                    $$
                    The left part of the goal in the lower right corner is
                    vacously true because a variable is in normal form and there
                    is no term to which it reduces.

                    For the right part we assume $\Gamma,x^A \reduce \Delta$.
                    There are two possibilities:
                    \begin{enumerate}
                        \item $\Delta = \Delta_0,x^A$ where $\Gamma \reduce
                            \Delta_0$ for some $\Delta_0$:

                        In that case we get $\Delta_0 \vdash A: s$ from the
                            induction hypothesis which implies the goal
                            $\Delta_0,x^A \vdash x: A$.

                        \item $\Delta = \Gamma,x^B$ where $A \reduce B$:

                        In that case we get $\Gamma \vdash B : s$ from the
                            induction hypothesis which implies the goal
                            $\Gamma,x^B: x \vdash B$.
                    \end{enumerate}

                \item Product:
                $$
                \begin{array}{l|l}
                    \Gamma \vdash A: s_1
                    &
                    \SRLeftPart C A \Gamma {s_1}
                    \lor
                    \SRRightPart \Delta \Gamma A {s_1}
                    \\
                    \Gamma,x^A \vdash B: s_2
                    &
                    \SRLeftPart D B {\Gamma,x^A} {s_2}
                    \lor
                    \SRRightPart {\Delta'} {\Gamma,x^A} B {s_2}
                    \\
                    s_1 = s_2 \lor s_2 = \Prop
                    \\
                    \hline
                    \Gamma \vdash \Pi x^A. B : s_2
                    &
                    \SRLeftPart t {\Pi x^A.B} \Gamma {s_2}
                    \land
                    \SRRightPart \Delta \Gamma {\Pi x^A. B} {s_2}
                \end{array}
                $$
                \begin{enumerate}
                    \item Left part: We assume $\Pi x^A. B \reduce t$.

                    Since products are preserved under reduction
                        (lemma~\ref{ReductionProductAbstraction}) we have either
                        $t = \Pi x^C. B$ where $A \reduce C$ for some $C$ or $t
                        = \Pi x^A.D$ where $B \reduce D$ for some $D$.

                    In both cases we can derive from the induction hypotheses
                        either $\Gamma \vdash C: s_1$ or $\Gamma,x^A \vdash D:
                        s_2$. Therefore $\Gamma \vdash \Pi x^C. B: s_2$ or $\Gamma
                        \vdash \Pi x^A.D: s_2$ is valid trivially.

                    \item Right part: Assume $\Gamma \reduce \Delta$. From the
                        first induction hypothesis we get $\Delta \vdash A:
                        s_1$. From the second induction hypothesis we get
                        $\Delta, x^A \vdash B: s_2$ where we use $\Delta' =
                        \Delta, x^A$. These facts imply $\Delta
                        \vdash \Pi x^A. B : s_2$.
                \end{enumerate}

                \item Abstraction: Same reasoning as with product.

                \item Application:
                $$
                \begin{array}{l|l}
                    \Gamma \vdash f: \Pi x^A. B
                    &
                    \SRLeftPart g f \Gamma {\Pi x^A. B}
                    \land
                    \SRRightPart \Delta \Gamma f {\Pi x^A.B}
                    \\
                    \Gamma \vdash a: A
                    &
                    \SRLeftPart b a \Gamma A
                    \land
                    \SRRightPart \Delta \Gamma a A
                    \\
                    \hline
                    \Gamma \vdash f a: B[x:=a]
                    &
                    \SRLeftPart t {f a} \Gamma {B[x:=a]}
                    \land
                    \SRRightPart \Delta \Gamma {f a} {B[x:=a]}
                \end{array}
                $$
                \begin{enumerate}
                    \item Left part: Assume $f a \reduce t$. We have three cases
                        to consider.
                    \begin{enumerate}
                        \item $f a \reduce g a$ where $f \reduce g$: From the
                            first induction hypothesis we get that $g$ has the
                            same type as $f$ and therefore $g a: B[x:=a]$ is
                            valid.

                        \item $f a \reduce f b$ where $a \reduce b$:
                            From the second induction hypothesis we get that $b$
                            has the same type as $a$ and therefore $f b:
                            B[x:=b]$ is valid. ?????INCOMPLETE???

                        \item $(\lambda y^B. e) a \reduce e[y:=a]$:

                    \end{enumerate}

                    \item Right part: Immediate consequence of the induction
                        hypotheses.
                \end{enumerate}
            \end{enumerate}

            \item Structural rules:
            \begin{enumerate}
                \item Weaken:
            \end{enumerate}
        \end{enumerate}

        INCOMPLETE!!!
    \end{proof}
    }
\end{theorem}





\subsection{Uninhabited Types}
%%%%%%%%%%%%%%%%%%%%%%%%%%%%%%%%%%%%%%%%%%%%%%%%%%%%%%%%%%%%%%%%%%%%%%%%%%%%%%%%



There are types which cannot be inhabited by a term in normal form in the empty
context.

\begin{theorem}
    \emph{The type $\Pi X^s. X$ cannot be inhabited by a term in normal form
    (i.e. a term with no redexes) in the empty context.}
    $$
    \ruleh {
        [] \vdash t: \Pi X^s. X
        \\
        t \text{ is in normal form}
    }
    {
        \perp
    }
    $$

    \begin{proof}
        We prove this theorem by induction on the structure of $t$.
        \begin{enumerate}
        \item $t$ is the sort $s_1$:

            By the generation lemma we postulate the existence of a sort $s_2$
                with $[] \vdash s_1: s_2 \land s_2 \le \Pi X^s. X$. This is
                contradictory because $s_2 \le \Pi X^s.X$ is imposible.

        \item $t$ is a variable:
            This is impossible, because the context is empty.

        \item $t$ is a product:
            This is impossible. The type of a product is always a sort and never
                a product.

        \item $t$ is an application:

            If $t$ is an application it must have the form $h a_0 a_1 \ldots$
                where the head term $h$ is neither an application nor a sort nor
                a product. Therefore the head term must be either a variable or
                an abstraction. It cannot be a variable, because the context is
                empty. It cannot be an abstraction, because the term is in
                normal form and therefore cannot have the redex $h a_0$.

        \item $t$ is the abstraction $\lambda x^A. e$:

            By the generation lemma we postulate the existence of $s_2$ and $B$
                with
                $$
                \begin{array}{l}
                    [] \vdash \Pi x^A. B: s_2
                    \\
                    \text{$[x^A]$} \vdash e: B
                    \\
                    \Pi x^A. B \le \Pi X^s. X
                \end{array}
                $$

                Because of the last statement we conclude $x = X$, $A \equiv s$
                and $B \le X$ where the latter implies $B = X$. I.e. we have
                lead the existence of $e$ with $[X^A] \vdash e: X$ to a
                contradiction.

                We prove this claim by induction on the structure of $e$.
                \begin{enumerate}
                \item $e$ is a sort:

                    This is impossible, because the type of a sort cannot be a
                        variable.

                \item $e$ is a variable: The only variable in the context is
                    $X$. By the generation lemma we know that $A \le X$. By
                        definition of $\le$ this is possible only if $A \equiv
                        X$ which implies $X \equiv s$ which is contradictory.
                \end{enumerate}
        \end{enumerate}
    \end{proof}
\end{theorem}
