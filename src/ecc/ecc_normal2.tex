\section{Strong Normalization}


\subsection{Overview}

We want to prove that all welltyped terms $t$ i.e. terms which satisfy $\Gamma
\vdash t: T$ for some context $\Gamma$ and type $T$ are strongly normalizing.
The most straightforward way is to try to prove it by induction on $\Gamma
\vdash t: T$. The proof works for nearly all rules. Sorts and variables are
already in normal form. Abstractions and products can use the induction
hypothesis to prove the strong normalization by the strong normalization of its
components. The structural rules for weakening, subtyping and reduction
immediately prove the strong normalization from the induction hypothesis.

The only rule which fails is the rule for application. There is no way to prove
the strong normalization of the application $f a$ from the strong normalization
of its components $f$ and $a$. A simple counterexample: Use $\lambda x^A. x x$
for $f$ and $a$. The components are strongly normalizing as long as $A$ is
strongly normalizing. But we have $f a \reduce f a \reduce f a \ldots$ which is
the start of an infinite reduction sequence.



\subsubsection{Value Sets}

For each welltyped term $t$ i.e. a valid term in a context with a type ($\Gamma
\vdash t: T$) we define a value set (or a model set) $\nu t$. We use the fact
that there is a minimal type $T_\mu$ which can be put into quasinormal form
$T_{\mu q}$ i.e.  $\Gamma \vdash_{\mu q} t: T_{\mu q}$.

A type in quasinormal form is either a sort, a base term or a term of the form
$\Pi x^A. B$ where $A$ and $B$ are in quasinormal form.

A value set for a term $t$ with minimal quasinormal type $T_{\mu q}$ is either
$\set{\theta}$ for some special symbol $\theta$, the set of saturated sets for
the corresponding type $\satsets\, T_{\mu q}$ or a set of mathematical functions
mapping a term and a value set to a value set. The last form is chosen when
$T_{\mu q}$ is not a proposition and has the form $\Pi x^A. B$ i.e. the value
set of a term which is a function is a set of mathematical functions.

Remark: We use the term \emph{mathematical function} to distinguish them from
lambda terms which represent a function.





