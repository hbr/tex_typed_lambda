\section{Thoughts}


\subsection{Predicative Church Encodings}

Booleans:
$$
\begin{array}{llll}
    B_i:
    &\Any_{i+1}
    &:= &
    \Pi A^{\Any_i}. \underbrace{A \to A \to A}_{:\Any_i} 
    %
    \\
    \text{true}_i:
    & B_i
    &:=&
    \lambda A^{\Any_i} a^A b^A := a
    %
    \\
    \text{false}_i:
    & B_i
    &:=&
    \lambda A^{\Any_i} a^A b^A := b
    %
    \\
    \text{and}_i:
    & B_{i+1} \to B_i \to B_i
    &:=&
    \lambda a^{B_{i+1}} b^{B_i}. a B_i b \text{ false}_i
\end{array}
$$

Natural numbers:
$$
\begin{array}{llll}
    N_i:
    &\Any_{i+1}
    &:= &
    \Pi A^{\Any_i}. \underbrace{A \to (A \to A) \to A}_{:\Any_i} 
    %
    \\
    \text{zero}_i: & N_i &:= &\lambda A^{\Any_i} s^A f^{A \to A}. s
    %
    \\
    \text{succ}_i: & N_i \to N_i &:= &
    \lambda n^{N_i} A^{\Any_i} s^A f^{A \to A}. f (n A s f)
    \\
    %
    \text{plus}_i:
    & N_{i+1} \to N_i \to N_i
    &:=&
    \lambda a^{N_{i+1}} b^{N_i}. a N_i b \text{ succ}_i
    \\
    %
    \text{mult}_i:
    & N_{i+1} \to N_{i+1} \to N_i
    &:=&
    \lambda a^{N_{i+1}} b^{N_{i+1}}
    . a N_{i+1} \text{ zero}_i \underbrace{(\text{plus}_i b)}_{: N_i \to N_i}
\end{array}
$$


Ackermann Function:
$$
a m n :=
\dcases{
a 0 n &:=& 1 + n
\\
a (1+m) 0 &:=& a m 1
\\
a (1+m) (1+n) &:=& a m (a (1+m) n)
}
$$

It can be defined in the impredicative style:
$$
\begin{array}{llll}
    g:
    & (N \to N) \to (N \to N)
    &:=&
    \lambda f n. \text{ succ } n\, N (\text{succ zero}) f
    \\
    %
    a: & N \to N \to N
    &:=&
    \lambda m. m (N \to N) \text{ succ } g
\end{array}
$$









\subsection{STLC Weak Normalization}
%-------------------------------------------------------------------------------


In simply typed lambda calculus all products $\Pi x^A. B$ have the form $A \to
B$ i.e. $x \notin \FV(B)$. Therefore all types can be constructed by the grammar
$$
T ::= X \mid T \to T
$$
where $X$ ranges over (type) variables and $T$ over types. All types $T$ have a
degree $D(T)$ which is defined as
$$
D(T) := \dcases {
    D(X) &:=& 0
    \\
    D(A \to B) &:=& 1 + \text{max}(D(A), D(B))
}
$$

The degree of a redex is the degree of the type of its rator. The degree of a
redex is by definition greater than zero (the rator of a redex must have type $A
\to B$.


Any term $t$ not in normal form has some redexes. For a term we define the
measure
$$
n_{m-1} \ldots n_2 n_1 n_0
$$
where $n_i$ is the number of redexes of degree $i+1$ and $m$ is the greatest
degree if all redexes of the term. The lexicographic order measures is a
wellorder.

\begin{theorem}
    \emph{A rightmost reduction of a term reduces its measure.}

    \begin{proof}
        Let
        $$
        t = C[(\lambda x^A.e) a] \reduce C[e[a/x]]
        $$
        be a rightmost reduction i.e. $(\lambda x^A.e)a)$ is the rightmost redex
        of the term and has degree $k+1$.

        Since the contracted redex is the rightmost redex $e$ and $a$ must be
        in normal form. Therefore all redexes in $e[a/x]$ have to be new redexes
        which have not been present in the original term.

        All newly created redexes have a degree strictly smaller than $k+1$.
        Reason: There are only three possibilities to create new redexes and the
        degree of the new redexes is strictly smaller:
        \begin{enumerate}
            \item The reduction of the rightmost redex has the form
                $$
                (\lambda x^{A \to B}. \ldots x a \ldots ) (\lambda y^A.e)
                \reduce
                \ldots (\lambda y^A. e) a \ldots
                $$
                where $x a$ is in the head position of an application (otherwise
                it were not a new redex). The degree of the newly created redex
                is $D(A \to B)$. The degree of the original redex is $D((A \to
                B) \to C)$ for some type $C$ i.e. the degree of the original
                redex is striclty greater than the degree of the newly created
                redex.

            \item The reduction of the rightmost redex has the form
                $$
                (\lambda x^{A \to B}. x) (\lambda y^A. e)
                \reduce
                \lambda y^A. e
                $$
                and the redex in the original term is applied to an argument
                $a$. I.e. the newly created redex is $(\lambda y^A. e) a$ and
                has degree $D(A \to B)$. The original redex has degree $D((A \to
                B) \to (A \to B))$ which is strictly greater.


            \item The reduction of the rightmost redex has the form
                $$
                (\lambda x^A. (\lambda y^B. e)) a
                \reduce
                \lambda y^B. e[a/x]
                $$
                and the redex in the original term is applied to an argument
                $b$. I.e. the newly created redex is $(\lambda y^B. e[a/x]) b$
                and has degree $D(B \to C)$. The original redex has degree $D(A
                 \to (B \to C))$ which is strictly greater.
        \end{enumerate}

        Therefore the number of degree $k$ redexes decreases by one, the numbers
        of redexes with higher degree stay the same. I.e. the measure decreases.
    \end{proof}
\end{theorem}

\begin{remark}
    In the proof of the previous theorem the requirement that the redex is
    rightmost can be relaxed a little bit. It is at least necessary that
    the argument $a$ of the redex $(\lambda x^A.e) a$ only contains
    redexes of lower degree. Since the bound variable $x$ might be in the
    body more than once, the contractum $e[a/x]$ might multiply the number of
    redexes of $a$. As long as they are of lower degree, the measure decreases.
\end{remark}







\subsection{Barendregt-Geuvers-Klop Conjecture}
%-------------------------------------------------------------------------------

Proof goals:
\begin{enumerate}
    \item If a welltyped term has a normal form, then it can be found
        by a rightmost reduction.

    \item If a rightmost reduction of a term ends in a normal form, then
        the term is strongly normalizing.

    \item  If all subterms of a term are weakly normalizing, then the term
        is strongly normalizing. Or the other way round: If a term is not
        strongly normalizing, then it must have a subterm which is not
        normalizing.

        It might be necessary to make the premise stronger: If all subterms of
        any reduct of a term are weakly normalizing, then the term is strongly
        normalizing.

        This formulation might be better fitted to typed lambda calculi
        where we try to prove that all welltyped terms are weakly normalizing.
        Due to subject reduction all reducts of a welltyped term are welltyped
        as well.
\end{enumerate}

Candidates for intermediate lemmas:
\begin{enumerate}
    \item A rightmost reduction reduces all redexes of a term.

        Proof idea: A rightmost reduction step contracts the rightmost redex.
        The contraction only modifies subterms of all redexes
        in which is is contained. But the redexes remain. Finally it has to
        contract all of them.
\end{enumerate}


Concept: Terms with an infinite reduction sequence where all proper subterms are
strongly normalizing. Examples $(\lambda x. x x)(\lambda x. x x)$, $U U f$ where
$U = \lambda x y. y (x x y)$. Such terms must be applications with a redex in
the head position. Reason: All subterms can be converted to their normal form.
If the term were not an application with  a redex in the head position, it would
be normal and could not have an infinite reduction sequence.

Note: A rightmost reduction reduces all subterms of a redex before it contracts
a redex. I.e. a rightmost reduction will finally encounter all terms with an
infinite reduction sequence which have only strongly normalizing proper
subterms.

All terms with an infinite reduction sequence (i.e. not strongly normalizing)
have minimal substerms with an infinite reduction sequence where all proper
subterms have no infinite reduction sequence i.e. where all proper subterms are
strongly normalizing. I.e. we can concentrate on terms with an infinite
reduction sequence where all proper subterms are strongly normalizing. If such
welltyped terms do not exist, then all terms are strongly normalizing.


\begin{theorem} If a term $t$ has an infinite reduction sequence, then the
    rightmost reduction sequence is infinite.

    \begin{proof}
        Main idea: A rightmost reduction of a term with an infinite reduction
        sequence never ends. After a finite number of steps it reduces to a term
        where the rightmost reduction continues.

        $t$ has  a rightmost subterm $r$ which has an infinite reduction
        sequence. Thus all proper subterms of $r$ and all terms to the right of
        $r$ are strongly normalizing. The rightmost reduction sequence reduces
        all terms to the right of $r$ and all proper subterms of $r$ to their
        normal form.

        $r$ must be an application of the form $(\lambda x. e) a \vec b$. This
        is the only possible form where all proper subterms are normal and the
        term still reduces.

        The rightmost reduction reduces this term to $e[a/x] \vec b$. $e$, $a$
        and $\vec b$ are in normal form. Therefore the term $e[a/x] \vec b$ must
        contain newly created redexes.
    \end{proof}
\end{theorem}







\begin{definition}
    \emph{Call by value reduction} $a \betarv b$ defined by the compatible
    closure of the rule
    $$
    \ruleh{
        a \in \NF
    }
    {
        (\lambda x^A. e) a \betarv e[a/x]
    }
    $$
\end{definition}







\begin{definition}
    \emph{Rightmost reduction} $a \betarr b$ defined by the rules
    \begin{enumerate}
        \item Redex contraction
            $$
                \ruleh{
                    a \in \NF
                }
                {
                    (\lambda x^A. e) a \betarr e[a/x]
                }
            $$

        \item Product result type reduction
            $$
            \rulev{
                B \betarr B'
            }
            {
                \Pi x^A.B \betarr \Pi x^A. B'
            }
            $$

        \item Product argument type reduction
            $$
            \rulev{
                A \betarr A'
                \\
                B \in \NF
            }
            {
                \Pi x^A.B \betarr \Pi x^{A'}. B
            }
            $$

        \item Abstraction body reduction
            $$
            \rulev{
                e \betarr e'
            }
            {
                \lambda x^A.e \betarr \lambda x^A. e'
            }
            $$

        \item Abstraction argument type reduction
            $$
            \rulev{
                A \betarr A'
                \\
                e \in \NF
            }
            {
                \lambda x^A.e \betarr \lambda x^{A'}. e
            }
            $$

        \item Application argument reduction
            $$
            \rulev{
                a \betarr b
            }
            {
                fa \betarr fb
            }
            $$

        \item Application function reduction
            $$
            \rulev{
                f \betarr g
                \\
                a \in \NF
            }
            {
                fa \betarr ga
            }
            $$
    \end{enumerate}
\end{definition}





\begin{theorem}
    \label{thm:HeadRedexStep}
    $$
    \rulev{
        (\lambda x^A. e) a \vec b \in \SN
        \\
        c \in SN
    }
    {
        (\lambda x^A. e) a \vec b c \in \SN
    }
    $$
    \begin{proof}
        MISSING
    \end{proof}
\end{theorem}





\begin{theorem}
    \label{thm:StronglyNormalizingRedex}
    $$
    \rulev{
        A, e, a, e[a/x] \vec b  \in \SN
    }
    {
        (\lambda x^A. e) a \vec b \in \SN
    }
    $$
    \begin{proof} By induction on the length of $\vec b$.

        The base case $|\vec b| = 0$ is trivial.

        The inductive case is proved by the previous
        theorem~\ref{thm:HeadRedexStep}.

        MISSING: Induction step needs rework!!
    \end{proof}
\end{theorem}







\begin{definition}
    \emph{$n$-step relation}
    \begin{enumerate}
        \item $$ a \toreln r 0 a$$

        \item $$
            \ruleh{
                a \torel r b
                \\
                b \toreln r i c
            }
            {
                a \toreln r {i+1} c
            }
            $$
    \end{enumerate}
\end{definition}






\begin{definition}
    \emph{$n$-step beta reduction}
    \begin{enumerate}
        \item $$ a \betarn 0 a$$

        \item $$
            \ruleh{
                a \reduce b
                \\
                b \betarn i c
            }
            {
                a \betarn {i+1} c
            }
            $$
    \end{enumerate}
\end{definition}


\begin{theorem}
    $$
    \ruleh{
        \exists n. \lnot \exists b. a \betarn n b
    }
    {
        a \in \SN
    }
    $$
    \begin{proof}

        By induction on $a \betarn n b$.

        The base case is trivial, because the assumption $a \betarn 0 a \imp
        \perp$ is not satisfiable.

        MISSING
    \end{proof}
\end{theorem}







\begin{definition}
    \emph{Subterms of a term.}
    $$
    \subterms t :=
    \dcases {
        \subterms s &:=& \set{s}
        \\
        \subterms x &:=& \set{x}
        \\
        \subterms {(\Pi x^A.B)} &:=&
            \set{\Pi x^A. B} \cup \subterms A \cup \subterms B
        \\
        \subterms {(\lambda x^A.e)} &:=&
            \set{\lambda x^A. e} \cup \subterms A \cup \subterms e
        \\
        \subterms {(a b)} &:=&
            \set{ab} \cup \subterms a \cup \subterms b
    }
    $$
\end{definition}







\begin{definition}
    \emph{Diverging term.} A term $t$ is diverging if it is not strongly
    normalizing i.e. $t \notin \SN$.
\end{definition}





\begin{definition}
    \emph{Minimal diverging term.} A term $t$ is minimal diverging if it is not
    strongly normalizing i.e. $t \notin \SN$ and all its proper subterms are
    strongly normalizing.
\end{definition}




\begin{theorem}
    \label{thm:FormMinimalDiverging}
    \emph{A minimal diverging term $t$ has the form $(\lambda x^A.e) a \vec b$.}

    \begin{proof}
        This is evident by the following facts:
        \begin{enumerate}
        \item $t$ cannot be a sort or a variable because they are not diverging.

        \item $t$ cannot be a product or an abstraction because they are
            strongly normalizing if their proper subterms are strongly
            normalizing.

        \item Therefore it has to be an application. An application is strongly
            normalizing if its proper subterms are strongly normalizing unless
            its head term is an abstraction.
        \end{enumerate}
    \end{proof}
\end{theorem}

Note that it is perfectly possible that $\vec b$ is not empty in a minimal
diverging term. E.g.
$$
\begin{array}{lll}
    (\lambda x y z. x z) \omega a \omega
    &\reduce&
    (\lambda y z. \omega z) a \omega
    \\
    &\reduce&
    (\lambda z. \omega z) \omega
    \\
    &\reduce&
    \omega \omega
\end{array}
$$
where $\omega = \lambda x. x x$. Without the last argument the term would be
strongly normalizing with its unique normal form $\lambda z. \omega z$.




\begin{theorem}
    \emph{All one step reducts $u$ of a minimal diverging term $t$ are
    diverging.}
    $$
    \rulev{
        t \text{ is minimal diverging}
        \\
        t \reduce u
    }
    {
        u \text{ is diverging}
    }
    $$

    \begin{proof}
        From the theorem~\ref{thm:FormMinimalDiverging} we know that
        $$
        t = (\lambda x^A.e) a \vec b
        $$
        is valid for some strongly normalizing terms $A$, $e$, $a$ and $\vec b$.
        We assume $(\lambda x^A.e) a \vec b) \reduce u$ and distinguish two
        cases:
        \begin{enumerate}
            \item We reduce the redex i.e. $u = e[a/x] \vec b$: Assume by way of
                contradiction that $u$ is strongly normalizing. Then
                by~\ref{thm:StronglyNormalizingRedex} $t$ would be strongly
                normalizing as well. This constradicts the assumption that $t$
                is diverging.

            \item We reduce one of $A$, $e$, $a$ or $\vec b$: In that case we
                prove the stronger statement that all terms $u$ which can be
                reached via zero or more reduction steps by reducing one of
                these subterms are minimal diverging.

                Since $A$, $e$, $a$ and $\vec b$ are strongly normalizing there
                exists a number $n$ such that no reduction sequence starting at
                $t$ and reducing only one of these subterms can be longer that
                $n$. Let $n$ be this number.

                We prove the fact that all terms $u$ are minimal diverging by
                induction on $n$.
                \begin{enumerate}
                    \item $n=0$: In this case the goal is vacuously valid,
                        because the terms $A$, $e$, $a$ and $\vec b$ are in
                        normal form such that no $u$ can be reached by reducing
                        one of these subterms.

                    \item Induction step: The induction hypothesis states that
                        for all $A$, $e$, $a$ and $\vec b$ and any reduction
                        sequence of $(\lambda x^A. e) a \vec b$ transforms them
                        into normal form after at most $n$ reduction steps then
                        all these reducts are minimal diverging.

                        Now assume that $(\lambda x^A.e)a \vec b$ transforms
                        these subterms into normal form by reducing only these
                        subterms in at most $n+1$ steps. Any first reduction
                        step reducing one of these subterm maintains the form of
                        the term but now any reduction sequence transforms these
                        subterms into normal form after at most $n$ steps.

                        MISSING: NEEDS REWORK!!
                \end{enumerate}
        \end{enumerate}
    \end{proof}
\end{theorem}
