\section{Typed Lambda Calculus}
%%%%%%%%%%%%%%%%%%%%%%%%%%%%%%%%%%%%%%%%%%%%%%%%%%%%%%%%%%%%%%%%%%%%%%%%%%%%%%%%






\subsection{Sorts}
%%%%%%%%%%%%%%%%%%%%%%%%%%%%%%%%%%%%%%%%%%%%%%%%%%%%%%%%%%%%%%%%%%%%%%%%%%%%%%%%


In the extended calculus of constructions terms and types are in the same
syntactic category. All welldefined terms have types and therefore types must
have types as well. In order to have types for types we start with the
introduction of sorts which are the types of types.

\begin{definition}
    \emph{Sorts:}
    There is
    \begin{itemize}
        \item the propositional sort $\Prop$ (sometimes written as $\Any_{-1}$)
    \item and the countably infinte collection of
    non-propositional sorts $\Any_i$ for $0 \le i$.
    \end{itemize}

    Sorts or universes are the types of types. Sorts are usually abbreviated by
    the variable $s$.
\end{definition}





\subsection{Terms}
%%%%%%%%%%%%%%%%%%%%%%%%%%%%%%%%%%%%%%%%%%%%%%%%%%%%%%%%%%%%%%%%%%%%%%%%%%%%%%%%

\begin{definition}
    The \emph{terms} are defined by the grammar where $s$ ranges over sorts, $x$
    ranges over some countably infinite set of variables and $t$ ranges over
    terms.
    $$
    \begin{array}{llll}
        t

        &::=& s & \text{sorts}

        \\

        &\mid & x & \text{variable}

        \\

        &\mid & \Pi x^t. t & \text{product}

        \\

        &\mid & \lambda x^t. t & \text{abstraction}

        \\

        &\mid & t t & \text{application}
    \end{array}
    $$
\end{definition}







\subsection{Contexts}
%%%%%%%%%%%%%%%%%%%%%%%%%%%%%%%%%%%%%%%%%%%%%%%%%%%%%%%%%%%%%%%%%%%%%%%%%%%%%%%%


\begin{definition}
    A \emph{context} is a sequence of variables and their corresponding types.
    Contexts are usually abbreviated by upper greek letters and types are terms
    which are usually abbreviated by uppercase letters.

    Contexts are defined by the grammar
    $$
    \begin{array}{llll}
        \Gamma
        &:=& [] & \text{empty context}

        \\

        &\mid& \Gamma, x^A & \text{one more variable $x$ with its type $A$}
    \end{array}
    $$
\end{definition}







\subsection{Free Variables}
%%%%%%%%%%%%%%%%%%%%%%%%%%%%%%%%%%%%%%%%%%%%%%%%%%%%%%%%%%%%%%%%%%%%%%%%%%%%%%%%

\begin{definition}
    The set of \emph{free variables} $\FV t$ of a term $t$ is defined by the function
    $$
    \FV t :=
        \begin{cases}
            \FV s &:= \emptyset

            \\

            \FV x &:= \{ x \}

            \\

            \FV (a b) &:= \FV a \cup \FV b

            \\

            \FV (\lambda x^A. e) &:= \FV A \cup (\FV e - \{x\})

            \\

            \FV (\Pi x^A. B) &:= \FV A \cup (\FV B - \{x\})
        \end{cases}
    $$
    where $s$ ranges over sorts, $x$ ranges over variables and $a$, $b$,
    $e$, $A$ and $B$ range over arbitrary terms.
\end{definition}

A variable which is not free is called a bound variable. E.g. if $x$ is a free
variable in the term $e$, it is no longer free in $\lambda x^A. e$. Therefore we
call $\lambda x^A. e$ a binder, because it makes the variable $x$ bound. The
same applies to the term $\Pi x^A. B$  where the variable $x$ is bound, but can
appear free in $B$.


It is possible to rename bound variables within a term. The renaming of a bound
variable does not change the term. We consider two terms which only differ in
the name of bound variables as identical. Examples of some identical terms:
$$
\begin{array}{lll}
    \lambda x^\Prop. x  &=& \lambda y^\Prop. y

    \\

    \Pi x^{\Any_5}. x   &=& \Pi y^{\Any_5}. y
\end{array}
$$



\subsection{Substitution}
%%%%%%%%%%%%%%%%%%%%%%%%%%%%%%%%%%%%%%%%%%%%%%%%%%%%%%%%%%%%%%%%%%%%%%%%%%%%%%%%


\begin{definition}
    \emph{Substitution:}
    %%%%%%%%%%%%%%%%%%%%%%%%%%%%%%%%%%%%%%%%%%%%%%%%%%%%%%%%%%%%%%%%%%%%%%%%%%%%%%%%
    The term $t[y:=c]$ is the term $t$ where the term $c$ is substituted for all
    free occurrences of the variable $y$. It is defined as a recursive function
    which iterates over all subterms until variables or sorts are encountered.
    $$
    t[y:=c] :=
        \begin{cases}
            s[y:=c] &:= s

            \\

            y[y:=c] &:= c

            \\

            x[y:=c] &:= x \quad \text{if } x \ne y

            \\

            (a b)[y:=c] &:= a[y:=c] b[y:=c]

            \\

            (\lambda x^A . e)[y:=c]
            &:=
            \lambda x^{A[y:=c]}. e[y:=c]
                \quad \text{if } x \notin \FV c

            \\

            (\Pi x^A . B)[y:=c]
            &:=
            \Pi x^{A[y:=c]}. B[y:=c]
                \quad \text{if } x \notin \FV c
        \end{cases}
    $$
\end{definition}
The condition $x \notin \FV c$ is not a restriction, because the bound variable
$x$ can always be renamed to annother variable which does not occur free in the
term $c$.





\subsection{Beta Reduction}
%%%%%%%%%%%%%%%%%%%%%%%%%%%%%%%%%%%%%%%%%%%%%%%%%%%%%%%%%%%%%%%%%%%%%%%%%%%%%%%%

Like in untyped lambda calculus, computation is done via \emph{beta reduction}.
A beta redex has the form $(\lambda x^A. e) a$ which reduces to the reduct
$e[x:=a]$. Beta reduction can be done in any subterm of a term.

\begin{definition}
    \emph{Beta reduction}
    %%%%%%%%%%%%%%%%%%%%%%%%%%%%%%%%%%%%%%%%%%%%%%%%%%%%%%%%%%%%%%%%%%%%%%%%%%%%
    is a binary relation $a \reduce b$ where the term $a$
    reduces to the term $b$. It is defined by the rules
    \begin{description}

        \item[redex]
            $$
                (\lambda x^A. e) a \reduce e[x:=a]
            $$

        \item[reduce function]
            $$
                \ruleh{
                    f \reduce g
                }
                {
                    f a \reduce g a
                }
            $$

        \item[reduce argument]
            $$
                \ruleh{
                    a \reduce b
                }
                {
                    f a \reduce f b
                }
            $$

        \item[reduce abstraction type]
            $$
                \ruleh{
                    A \reduce B
                }
                {
                    \lambda x^A. e \reduce \lambda x^B . e
                }
            $$

        \item[reduce abstraction term]
            $$
                \ruleh{
                    e \reduce f
                }
                {
                    \lambda x^A. e \reduce \lambda x^A. f
                }
            $$

        \item[reduce product type]
            $$
                \ruleh{
                    A \reduce B
                }
                {
                    \Pi x^A. C \reduce \Pi x^B . C
                }
            $$

        \item[reduce product result]
            $$
                \ruleh{
                    B \reduce C
                }
                {
                    \Pi x^A. B \reduce \Pi x^A. C
                }
            $$
    \end{description}
\end{definition}



\begin{theorem}
    \label{DoubleSubstitution}
    %%%%%%%%%%%%%%%%%%%%%%%%%%%%%%%%%%%%%%%%%%%%%%%%%%%%%%%%%%%%%%%%%%%%%%%%%%%%%%%% 
    \emph{Double substitution lemma} If we change the order of two subsequent
    substitutions, then it is necessary to introduce a correction term.
    $$
    \ruleh{
        x \ne y
        \\
        y \notin \FV a
    }
    {
        t[y:=b][x:=a]
        =
        t[x:=a][y:=b[x:=a]]
    }
    $$
    \begin{proof}
        The intuitive proof is quite evident. If we substitute $b$ for the
        variable $y$ in the term $t$ and then substitute $a$ for the variable
        $x$ in the result we replace in the second substitution not only all
        variables $x$ originally contained in $t$ but all occurrences of $x$ in
        $b$ as well.

        If we do the substitution $[x:=a]$ on $t$, then all occurrences of $x$
        in $b$ have not yet been substituted. Therefore the correction term
        $b[x:=a]$ is necessary if the order is changed.

        The formal proof goes by induction on the structure of $t$.

        \begin{enumerate}

            \item If $t$ is a sort then the equality is trivial because a sort
                does not have any variables.

            \item If $t$ is an application, an abstraction or a product, then
                the goal follows immediately from the induction hypotheses.

            \item The only interesting case is when $t$ is a variable. Let's
                call the variable $z$. Then we have to distinguish the cases
                that $z$ is $x$ or $z$ is $y$ or $z$ is different from $x$ and
                $y$.

                \begin{enumerate}
                    \item Case $z = x$:
                        $$
                        \begin{array}{llll}
                            x[y:=b][x:=a]
                            &=& x[x:=a]
                            \\
                            &=& a
                            \\
                            \\
                            x[x:=a][y:=b[x:=a]]
                            &=& a[y:=b[x:=a]]
                            \\
                            &=& a & y \notin \FV a
                        \end{array}
                        $$

                    \item Case $z = y$:
                        $$
                        \begin{array}{llll}
                            y[y:=b][x:=a]
                            &=& b[x:=a]
                            \\
                            \\
                            y[x:=a][y:=b[x:=a]]
                            &=& y[y:=b[x:=a]]
                            \\
                            &=&b[x:=a]
                        \end{array}
                        $$

                    \item Case $z \ne x \land z \ne y$:
                        $$
                        \begin{array}{llll}
                            z[y:=b][x:=a]
                            &=& z
                            \\
                            \\
                            z[x:=a][y:=b[x:=a]]
                            &=& z
                        \end{array}
                        $$

                \end{enumerate}
        \end{enumerate}
    \end{proof}
\end{theorem}

The validity of the
\emph{double substitution lemma}~\ref{DoubleSubstitution}
is needed to make sure that \emph{beta
reduction} is confluent. E.g. if we have the term
$(\lambda x^A. (\lambda y^B. t) b) a$ then we can decide whether we reduce first
the inner redex and then the outer redex or the other way round. Because of
confluence both possibilities shall have the same result.
$$
\begin{array}{llll}
    (\lambda x^A. (\lambda y^B. t) b) a
    &\reduce& (\lambda x^A. t[y:=b]) a
    \\
    &\reduce& t[y:=b][x:=a]
    \\
    \\
    (\lambda x^A. (\lambda y^B. t) b) a
    &\reduce& (\lambda y^B. t[x:=a]) b[x:=a]
    \\
    &\reduce& t[x:=a][y:=b[x:=a]]
\end{array}
$$


\begin{theorem}
    \label{SubstituteReduction}
    %%%%%%%%%%%%%%%%%%%%%%%%%%%%%%%%%%%%%%%%%%%%%%%%%%%%%%%%%%%%%%%%%%%%%%%%%%%%%%%%
    \emph{Substitute Reduction} A reduction remains valid if we do the same
    substitution before and after the reduction.
    $$
    \ruleh{
        t \reduce u
    }
    {
        t[y:=v] \reduce u[y:=v]
    }
    $$

    \begin{proof}
        Proof by induction on $t \reduce u$.


        \begin{enumerate}

        \item Redex: We have to prove the goal
            $$
                ((\lambda x^A. e) a)[y:=v] \reduce e[x:=a][y:=v]
            $$

            We can see this by the sequence
            $$
            \begin{array}{lll}
                ((\lambda x^A. e) a)[y:=v]
                &=&
                (\lambda x^{A[y:=v]}. e[y:=v]) a[y:=v]
                \\
                &\reduce&
                e[y:=v][x:=a[y:=v]]
                \\
                &=& \text{by Double substitution lemma~\ref{DoubleSubstitution}}
                \\
                &&e[x:=a][y:=v]
            \end{array}
            $$

        \item Reduce function: We have to prove the goal
            $$
            \begin{array}{l|l}
                f \reduce g
                & f[y:=v] \reduce g[y:=v]
                \\
                \hline
                f a \reduce g a
                &(f a)[y:=v] \reduce (g a)[y:=v]
            \end{array}
            $$
            The validity of the final goal in the right lower corner can be seen
            by the following reasoning
            $$
            \begin{array}{lll}
                (f a)[y:=v]
                &=&
                f[y:=v] a[y:=v]
                \\
                &\reduce&
                g[y:=v] a[y:=v]
                \\
                &=&
                (g a)[y:=v]
            \end{array}
            $$

        \item Other rules: All other rules follow the same pattern as the proof
            of the rule \emph{reduce function}.
        \end{enumerate}
    \end{proof}
\end{theorem}





\subsection{Beta Equivalence}
%%%%%%%%%%%%%%%%%%%%%%%%%%%%%%%%%%%%%%%%%%%%%%%%%%%%%%%%%%%%%%%%%%%%%%%%%%%%%%%%


\begin{definition}
    We define \emph{beta equivalence} as a binary relation $a \equiv b$ between
    the terms $a$ and $b$ inductively by the rules
    \begin{enumerate}
    \item
        $$ a \equiv a$$

    \item
        $$
        \rulev
        {
            a \equiv b
            \\
            b \reduce c
        }
        {
            a \equiv c
        }
        $$

    \item
        $$
        \rulev
        {
            a \equiv b
            \\
            c \reduce b
        }
        {
            a \equiv c
        }
        $$
    \end{enumerate}

    In other words the beta equivalence relation $\equiv$ is the smallest
    equivalence relation which contains beta reduction $\reduce$.
\end{definition}













\subsection{Cumulativity / Subtyping}
%%%%%%%%%%%%%%%%%%%%%%%%%%%%%%%%%%%%%%%%%%%%%%%%%%%%%%%%%%%%%%%%%%%%%%%%%%%%%%%%

In this section we want to define a subtype relation $A \le B$ (read $A$ is a
subtype of $B$) which is a partial order (reflexive, transitive and
antisymmetric) with respect to beta equivalent terms.

Furthermore we want a proper subtype relation $A < B$ (read $A$ is a proper
subtype of $B$) which is a strict partial order (irreflexive and transitive)
with respect to beta equivalence.

In the partial orders we regard beta equivalent terms as the same terms.

In order to define the partial orders we first define the prececessor relation
$A \prec_i B$ saying that type $A$ is a predecessor of type $B$ and both have
the arity $i$. In the predecessor relation we consider only terms of the form
$\Pi \vec{x}^\vec{A}. s$ where $s$ is a sort i.e. it has the form $\Any_k$ with
$-1 \le k$. Only terms with identical arguments and same arity figure in the
predecessor relation i.e. $\Pi \vec{x}^\vec{A}. \Any_k \prec_i \Pi
\vec{x}^\vec{A}. \Any_l$ if $k < l$ and $i$ is the number of arguments.

Based on this we define the proper subtype relation $A < B$ saying that $A$ is a
proper subtype of $B$ and both are beta equivalent to some terms $A'$ and $B'$
which figure in the predecessor relation for some arity $i$ and are beta
equivalent to $A$ and $B$ respectively.

Finally the subtype relation $A \le B$ is defined by stating that either $A$ is
beta equivalent to $B$ or $A$ is a proper subtype of $B$.



\subsubsection{Predecessor Relation}

\begin{definition}
    We define the
    %
    \emph{predecessor relation}
    %%%%%%%%%%%%%%%%%%%%%%%%%%%
    %
    $A \prec_i B$ between the terms $A$
    and $B$ with the arity $i$ inductively by the following rules
    \begin{enumerate}
    \item
        $$
        \ruleh {
            -1 \le k < l
        }
        {
            \Any_k \prec_0 \Any_l
        }
        $$

    \item
        $$
        \ruleh {
            B \prec_i C
        }
        {
            \Pi x^A. B \prec_{i+1} \Pi x^A. C
        }
        $$
    \end{enumerate}
\end{definition}



\begin{lemma}
    \label{PrecedenceSubstitution}
    \emph{Substitution does not affect the predecessor relation}
    $$
    \ruleh{A \prec_i B}{A[y:=b] \prec_i B[y:=b]}
    $$

    \begin{proof}
        By induction on $A \prec_i B$.

        The first case is trivial because substitution does not modify a sort.

        The second case follows immediately from the induction hypothesis.
    \end{proof}
\end{lemma}



\begin{lemma}
    \label{PrecedenceSameArity}
    %%%%%%%%%%%%%%%%%%%%%%%%%%%
    \emph{Subsequent predecessor relations connected by equivalent terms have
    the same aritiy}.
    $$
    \ruleh{
        A \prec_i B
        \\
        C \prec_j D
        \\
        B \betaeq C
    }
    {
        i = j
    }
    $$

    \begin{proof}
        Proof by induction on $A \prec_i B$
        \begin{enumerate}
        \item
            $$
            \begin{array}{l|l}
                -1 \le k < l
                \\
                \hline
                \Any_k \prec_0 \Any_l
                &
                \forall C D.
                \left[
                \ruleh{
                    C \prec_j D
                    \\
                    \Any_l \betaeq C
                }
                {0 = j}
                \right]
            \end{array}
            $$
            We prove the goal in the right lower corner by making a case split
            on the construction of $C \prec_j D$. Only the first case
            is possible with the trivial goal $0 = 0$. The second case is
            impossible because $\Any_l \betaeq \Pi x^A. R$ cannot be the case.

        \item
        $$
            \begin{array}{l|l}
                B \prec_i C
                &
                \forall k F G.
                \left[
                \ruleh{F \prec_k G \\ C \betaeq F}{i = k}
                \right]
                \\
                \hline
                \Pi x^A.B \prec_{i+1} \Pi x^A. C
                &
                \forall j D E.
                \left[
                \ruleh{
                    D \prec_j E
                    \\
                    \Pi x^A. C \betaeq D
                }
                {i+1 = j}
                \right]
            \end{array}
        $$
        We prove the goal in the right lower corner by making a case split on
        the construction of $D \prec_j E$.

        The first case is impossible, because $\Pi x^A. B$ cannot be beta
        equivalent to a sort.

        In the second case we construct $D \prec_j E$ as
        $\Pi x^{A'}. F \prec_{k+1} \Pi x^{A'}. G$ where $j = k + 1$,
        $D = \Pi x^{A'}. F$ and $E = \Pi x^{A'}. G$
        under the premise $F \prec_k G$.

        The condition $\Pi x^A. C \betaeq D$ becomes $\Pi x^A.C \betaeq \Pi
        x^{A'}. F$ which implies $A \betaeq A'$ and $C \betaeq F$.

        From the induction hypothesis we infer $i = k$ which implies the final
        goal $i + 1 = j$.
        \end{enumerate}
    \end{proof}
\end{lemma}



\begin{lemma}
    \label{PrecedenceTransitive}
    \emph{The predecessor relation is transitive}
    $$
    \ruleh{
        A \prec_i B
        \\
        B \prec_i C
    }
    {
        A \prec_i C
    }
    $$
    \begin{proof}
        Trivial by induction on $A \prec_i B$.
    \end{proof}
\end{lemma}






\begin{lemma}
    \label{PrecedenceEqTransitive}
    \emph{Subsequent predecessor relations connected by equivalent terms can be
    joined to one predecessor relation}.
    $$
    \ruleh{
        A \prec_i B
        \\
        B' \prec_i C'
        \\
        B \betaeq B'
    }
    {
        \exists C. C \betaeq C' \land B \prec_i C
    }
    $$

    \begin{proof}
        By induction on $A \prec_i B$.
        \begin{enumerate}
        \item
            $$
            \begin{array}{l|l}
                -1 \le k < l
                \\
                \hline
                \Any_k \prec_0 \Any_l
                &
                \ruleh {
                    B' \prec_0 C'
                    \\
                    \Any_l \betaeq B'
                }
                {
                    \exists C. C \betaeq C' \land \Any_l \prec_0 C
                }
            \end{array}
            $$

            The only possibility to make both premises in the lower right corner
                valid is by
            $$
            \begin{array}{lll}
                B' &=& \Any_l
                \\
                C' &=& \Any_m
                \\
                l &<& m
            \end{array}
            $$
            The final goal is valid by setting
            $$
                C = \Any_m
            $$
            which trivially satisfies $\Any_m \betaeq \Any_m$ and $\Any_l
            \prec_0 \Any_m$


        \item
            $$
            \begin{array}{l|l}
                B \prec_i C
                &
                \forall C' D'.
                \left[
                    \ruleh {
                        C' \prec_i D'
                        \\
                        C \betaeq C'
                    }
                    {
                        \exists D. D \betaeq D' \land C \prec_i D
                    }
                \right]
                \\
                \hline
                \Pi x^A. B \prec_{i+1} \Pi x^A.C
                &
                \ruleh{
                    E' \prec_{i+1} F'
                    \\
                    \Pi x^A.C \betaeq E'
                }
                {
                    \exists F.
                    F \betaeq F'
                    \land
                    \Pi x^A. C \prec_{i+1} F
                }
            \end{array}
            $$
            From $E' \prec_{i+1} F'$ we conclude the existence of $A'$, $C'$ and
            $D'$ such that
            $$
            \begin{array}{lll}
                E' &=& \Pi x^{A'}. C'
                \\
                F' &=& \Pi x^{A'}. D'
                \\
                C' &\prec_i& D'
            \end{array}
            $$
            From $\Pi x^A.C \betaeq E'$ we infer
            $$
                A \betaeq A' \land C \betaeq C'
            $$
            Then we use the induction hypothesis to infer the existence of some
            $D$ which satisfies
            $$
                D \betaeq D' \land C \prec_i D
            $$
            Finally we use $F := \Pi x^A. D$ which satisfies the final goal.
        \end{enumerate}
    \end{proof}
\end{lemma}



\begin{lemma}
    \label{PrecedenceIrreflexive}
    \emph{The predecessor relation is irreflexive}.
    $$\ruleh {A \prec_i B \\ B \betaeq A}{\perp}$$

    \begin{proof}
        Proof by induction on $A \prec_i B$.

        \begin{enumerate}
        \item
        $$
            \begin{array}{l|l}
                -1 \le k < l
                \\
                \hline
                \Any_k \prec_0 \Any_l
                &
                \Any_l \betaeq \Any_k \imp \perp
            \end{array}
        $$
        $\Any_k$ and $\Any_l$ cannot be beta equivalent for different $k$ and $l$


        \item
        $$
            \begin{array}{l|l}
                B \prec_i C
                &
                C \betaeq B \imp \perp
                %
                \\
                \hline
                %
                \Pi x^A. B \prec_{i+1} \Pi x^A. C
                &
                \Pi x^A. C \betaeq \Pi x^A. B \imp \perp
            \end{array}
        $$
        To prove the goal in the right lower corner we assume $\Pi x^A. C
        \betaeq \Pi x^A.B$. This can be the case only if $C$ and $B$ are
        beta equivalent. From the induction hypothesis we immediately
        get the contradiction.
        \end{enumerate}
    \end{proof}
\end{lemma}








\subsubsection{Proper Subtype Relation}



\begin{definition}
    \emph{Proper Subtype}
    $A$ is a proper subtype of $B$ written as $A < B$
    of $A$ and $B$ if the following rule is satisfied
    $$
    \rulev{
        A \betaeq A'
        \\
        A' \prec_i B'
        \\
        B' \betaeq B
    }
    {
        A < B
    }
    $$
\end{definition}



\begin{theorem}
    \label{ProperSubtypeIrreflexive}
    \emph{The proper subtype relation is irreflexive}
    %%%%%%%%%%%%%%%%%%%%%%%%%%%%%%%%%%%%%%%%%%%%%%%%%
    $$
    \ruleh{
        A < B
        \\
        A \betaeq B
    }
    {\perp}
    $$

    \begin{proof}
        By definition of the proper subtype relation there exists an arity $i$ and
        some type $A'$ and $B'$ with $A \betaeq A'$, $B \betaeq B'$ and $A'
        \prec_i B'$.

        We conclude $A' \betaeq B'$ from $A \betaeq B$. The
        lemma~\ref{PrecedenceIrreflexive} leads immediately to the
        contradiciton.
    \end{proof}
\end{theorem}





\begin{theorem}
    \label{ProperSubtypeTransitive}
    \emph{The proper subtype relation is transitive}
    $$
    \ruleh{
        A < B
        \\
        B < C
    }
    {A < C}
    $$

    \begin{proof}
        By definition of the proper subtype relation there exist some $A'$,
        $B'$, $B''$, $C''$, $i$ and $j$ such that the following is valid
        $$
        \begin{array}{lll}
            A' &\prec_i& B'
            \\
            A &\betaeq& A'
            \\
            B &\betaeq& B'
            \\
            B'' &\prec_j& C''
            \\
            B &\betaeq& B''
            \\
            C &\betaeq& C''
        \end{array}
        $$
        The equivalences imply $B' \betaeq B''$ which by using
        lemma~\ref{PrecedenceSameArity} implies $i = j$.

        From lemma~\ref{PrecedenceEqTransitive} we infer the existence of some
        $C'$ with
        $$
            C' \betaeq C'' \land B' \prec_i C'
        $$
        which implies with lemma~\ref{PrecedenceTransitive}
        $$
            A' \prec_i C'
        $$
        By definition of the proper subtype relation the goal
        $$
            A < C
        $$
        is evident.
    \end{proof}
\end{theorem}








\subsubsection{Subtype Relation}


\begin{definition}
    \emph{Subtype}
    \begin{enumerate}
    \item
        $$
        \ruleh {
            A \betaeq B
        }
        {
            A \le B
        }
        $$

    \item
        $$
        \ruleh {
            A < B
        }
        {
            A \le B
        }
        $$
    \end{enumerate}
\end{definition}


\begin{theorem}
    \emph{The subtype relation is reflexive with respect to beta equivalence}
    $$
    \ruleh{
        A \betaeq B
    }
    {
        A \le B
    }
    $$
    \begin{proof} Trival by the first rule.
    \end{proof}
\end{theorem}






\begin{theorem}
    \emph{The subtype relation is transitive}
    $$
    \ruleh{
        A \le B
        \\
        B \le C
    }
    {
        A \le C
    }
    $$
    \begin{proof}
        We have to distinguish four cases
        \begin{enumerate}
        \item $A \betaeq B \land B \betaeq C$: Trivial by the transitivity of
            beta equivalence.

        \item $A < B \land B \betaeq C$: $A < C$ follows immediately from the
            definition of $<$.

        \item $A \betaeq B \land B < C$: $A < C$ follows immediately from the
            definition of $<$.

        \item $A < B \land B < C$: $A < C$ follows immediately from the
            transitivity of $<$~\ref{ProperSubtypeTransitive}
        \end{enumerate}
    \end{proof}
\end{theorem}


\begin{theorem}
    \emph{The subtype relation is antisymmetric with respect to beta
    equivalence}
    $$
    \ruleh{
        A \le B
        \\
        B \le A
    }
    {
        A \betaeq B
    }
    $$
    \begin{proof}
        We have to distinguish four cases:
        \begin{enumerate}
        \item $A \betaeq B \land B \betaeq A$: Trivial.

        \item $A < B \land B \betaeq A$: This case is not possible because
            the irreflexivity of the proper subtype
            relation~\ref{ProperSubtypeIrreflexive} implies a contradiction.

        \item $A \betaeq B \land B < A$: Same as the previous case.

        \item $A < B \land B < A$: By the transitivity of the proper subtype
            relation~\ref{ProperSubtypeTransitive} we infer $A < A$ which
            implies by the irreflexivity of the proper subtype
            relation~\ref{ProperSubtypeIrreflexive} immediately a
            contradiction.
        \end{enumerate}
    \end{proof}
\end{theorem}








\subsection{Typing Relation}
%%%%%%%%%%%%%%%%%%%%%%%%%%%%%%%%%%%%%%%%%%%%%%%%%%%%%%%%%%%%%%%%%%%%%%%%%%%%%%%%


\begin{definition}
    The ternary \emph{typing relation} $\Gamma \vdash t: T$ which says that in
    the context $\Gamma$ then term $t$ has type $T$ is defined inductively by
    the rules
    \begin{enumerate}
    \item Introduction rules:
        \begin{enumerate}
            \item Axiom:
                $$
                \ruleh
                {i \in \{-1, 0, 1, 2, 3, \ldots \}}
                { [] \vdash \Any_i : \Any_{i + 1}}
                $$

        \item Variable:
            $$
            \rulev {
                \Gamma \vdash A: s
                \\
                x \notin \Gamma
            }
            {
                \Gamma, x^A \vdash x: A
            }
            $$

        \item Product:
            $$
            \rulev {
                \Gamma \vdash A: s_1
                \\
                \Gamma, x^A \vdash B: s_2
                \\
                s_1 = s_2 \lor s_1 = \Prop
            }
            {
                \Gamma \vdash \Pi x^A.B : s_2
            }
            $$

        \item Abstraction:
            $$
            \rulev {
                \Gamma \vdash \Pi x^A. B: s
                \\
                \Gamma, x^A \vdash e: B
            }
            {
                \Gamma \vdash (\lambda x^A. e): \Pi x^A. B
            }
            $$

        \item Application:
            $$
            \rulev {
                \Gamma \vdash f: \Pi x^A. B
                \\
                \Gamma \vdash a: A
            }
            {
                \Gamma \vdash f a: B[x:=a]
            }
            $$
        \end{enumerate}


    \item Structural rules:
        \begin{enumerate}
        \item Weaken:
            $$
            \rulev {
                \Gamma \vdash t: T
                \\
                \Gamma \vdash A: s
                \\
                x \notin \Gamma
            }
            {
                \Gamma, x^A \vdash t: T
            }
            $$

        \item Reduction:
            $$
            \rulev {
                \Gamma \vdash t: T
                \\
                T \reduce U
            }
            {
                \Gamma \vdash t: U
            }
            $$

        \item Expansion:
            $$
            \rulev {
                \Gamma \vdash t: T
                \\
                \Gamma \vdash U: s
                \\
                U \reduce T
            }
            {
                \Gamma \vdash t: U
            }
            $$

        \item Subtype:
            $$
            \rulev {
                \Gamma \vdash t: T
                \\
                T \prec_i U
            }
            {
                \Gamma \vdash t: U
            }
            $$
        \end{enumerate}

    \end{enumerate}
\end{definition}
