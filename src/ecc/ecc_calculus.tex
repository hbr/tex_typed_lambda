\section{The Calculus}
%%%%%%%%%%%%%%%%%%%%%%%%%%%%%%%%%%%%%%%%%%%%%%%%%%%%%%%%%%%%%%%%%%%%%%%%%%%%%%%%






\subsection{Sorts}
%%%%%%%%%%%%%%%%%%%%%%%%%%%%%%%%%%%%%%%%%%%%%%%%%%%%%%%%%%%%%%%%%%%%%%%%%%%%%%%%


In the extended calculus of constructions terms and types are in the same
syntactic category. All welldefined terms have types and therefore types must
have types as well. In order to have types for types we start with the
introduction of sorts which are the types of types.

\begin{definition}
    \emph{Sorts:} There is a countably infinite collection of sorts $\Any_0,
    \Any_1, \Any_2 \ldots$. All sorts except $\Any_0$ are called
    non-propositional sorts. The sort $\Any_0$ is the only propositional sort
    and sometimes written as $\Prop$.

    Sorts or universes are the types of types. Sorts are usually abbreviated by
    the variable $s$.
\end{definition}





\subsection{Terms}
%%%%%%%%%%%%%%%%%%%%%%%%%%%%%%%%%%%%%%%%%%%%%%%%%%%%%%%%%%%%%%%%%%%%%%%%%%%%%%%%

\begin{definition}
    The \emph{terms} are defined by the grammar where $s$ ranges over sorts, $x$
    ranges over some countably infinite set of variables and $t$ ranges over
    terms.
    $$
    \begin{array}{llll}
        t

        &::=& s & \text{sorts}

        \\

        &\mid & x & \text{variable}

        \\

        &\mid & \Pi x^t. t & \text{product}

        \\

        &\mid & \lambda x^t. t & \text{abstraction}

        \\

        &\mid & t t & \text{application}
    \end{array}
    $$
\end{definition}







\subsection{Contexts}
%%%%%%%%%%%%%%%%%%%%%%%%%%%%%%%%%%%%%%%%%%%%%%%%%%%%%%%%%%%%%%%%%%%%%%%%%%%%%%%%


\begin{definition}
    A \emph{context} is a sequence of variables and their corresponding types.
    Contexts are usually abbreviated by upper greek letters and types are terms
    which are usually abbreviated by uppercase letters.

    Contexts are defined by the grammar
    $$
    \begin{array}{llll}
        \Gamma
        &:=& [] & \text{empty context}

        \\

        &\mid& \Gamma, x^A & \text{one more variable $x$ with its type $A$}
    \end{array}
    $$
\end{definition}







\subsection{Free Variables}
%%%%%%%%%%%%%%%%%%%%%%%%%%%%%%%%%%%%%%%%%%%%%%%%%%%%%%%%%%%%%%%%%%%%%%%%%%%%%%%%

\begin{definition}
    The set of \emph{free variables} $\FV t$ of a term $t$ is defined by the function
    $$
    \FV t :=
        \begin{cases}
            \FV s &:= \emptyset

            \\

            \FV x &:= \{ x \}

            \\

            \FV (a b) &:= \FV a \cup \FV b

            \\

            \FV (\lambda x^A. e) &:= \FV A \cup (\FV e - \{x\})

            \\

            \FV (\Pi x^A. B) &:= \FV A \cup (\FV B - \{x\})
        \end{cases}
    $$
    where $s$ ranges over sorts, $x$ ranges over variables and $a$, $b$,
    $e$, $A$ and $B$ range over arbitrary terms.
\end{definition}

A variable which is not free is called a bound variable. E.g. if $x$ is a free
variable in the term $e$, it is no longer free in $\lambda x^A. e$. Therefore we
call $\lambda x^A. e$ a binder, because it makes the variable $x$ bound. The
same applies to the term $\Pi x^A. B$  where the variable $x$ is bound, but can
appear free in $B$.


It is possible to rename bound variables within a term. The renaming of a bound
variable does not change the term. We consider two terms which only differ in
the name of bound variables as identical. Examples of some identical terms:
$$
\begin{array}{lll}
    \lambda x^\Prop. x  &=& \lambda y^\Prop. y

    \\

    \Pi x^{\Any_5}. x   &=& \Pi y^{\Any_5}. y
\end{array}
$$



\subsection{Substitution}
%%%%%%%%%%%%%%%%%%%%%%%%%%%%%%%%%%%%%%%%%%%%%%%%%%%%%%%%%%%%%%%%%%%%%%%%%%%%%%%%


\begin{definition}
    \emph{Substitution:}
    %%%%%%%%%%%%%%%%%%%%%%%%%%%%%%%%%%%%%%%%%%%%%%%%%%%%%%%%%%%%%%%%%%%%%%%%%%%%%%%%
    The term $t[y:=c]$ is the term $t$ where the term $c$ is substituted for all
    free occurrences of the variable $y$. It is defined as a recursive function
    which iterates over all subterms until variables or sorts are encountered.
    $$
    t[y:=c] :=
        \begin{cases}
            s[y:=c] &:= s

            \\

            y[y:=c] &:= c

            \\

            x[y:=c] &:= x \quad \text{if } x \ne y

            \\

            (a b)[y:=c] &:= a[y:=c] b[y:=c]

            \\

            (\lambda x^A . e)[y:=c]
            &:=
            \lambda x^{A[y:=c]}. e[y:=c]
                \quad \text{if } x \notin \FV c

            \\

            (\Pi x^A . B)[y:=c]
            &:=
            \Pi x^{A[y:=c]}. B[y:=c]
                \quad \text{if } x \notin \FV c
        \end{cases}
    $$
\end{definition}
The condition $x \notin \FV c$ is not a restriction, because the bound variable
$x$ can always be renamed to annother variable which does not occur free in the
term $c$.



\begin{lemma}
    \label{DoubleSubstitution}
    %%%%%%%%%%%%%%%%%%%%%%%%%%%%%%%%%%%%%%%%%%%%%%%%%%%%%%%%%%%%%%%%%%%%%%%%%%%%%%%% 
    \emph{Double substitution lemma} If we change the order of two subsequent
    substitutions, then it is necessary to introduce a correction term.
    $$
    \ruleh{
        x \ne y
        \\
        y \notin \FV a
    }
    {
        t[y:=b][x:=a]
        =
        t[x:=a][y:=b[x:=a]]
    }
    $$
    \begin{proof}
        The intuitive proof is quite evident. If we substitute $b$ for the
        variable $y$ in the term $t$ and then substitute $a$ for the variable
        $x$ in the result we replace in the second substitution not only all
        variables $x$ originally contained in $t$ but all occurrences of $x$ in
        $b$ as well.

        If we do the substitution $[x:=a]$ on $t$, then all occurrences of $x$
        in $b$ have not yet been substituted. Therefore the correction term
        $b[x:=a]$ is necessary if the order is changed.

        The formal proof goes by induction on the structure of $t$.

        \begin{enumerate}

            \item If $t$ is a sort then the equality is trivial because a sort
                does not have any variables.

            \item If $t$ is an application, an abstraction or a product, then
                the goal follows immediately from the induction hypotheses.

            \item The only interesting case is when $t$ is a variable. Let's
                call the variable $z$. Then we have to distinguish the cases
                that $z$ is $x$ or $z$ is $y$ or $z$ is different from $x$ and
                $y$.

                \begin{enumerate}
                    \item Case $z = x$:
                        $$
                        \begin{array}{llll}
                            x[y:=b][x:=a]
                            &=& x[x:=a]
                            \\
                            &=& a
                            \\
                            \\
                            x[x:=a][y:=b[x:=a]]
                            &=& a[y:=b[x:=a]]
                            \\
                            &=& a & y \notin \FV a
                        \end{array}
                        $$

                    \item Case $z = y$:
                        $$
                        \begin{array}{llll}
                            y[y:=b][x:=a]
                            &=& b[x:=a]
                            \\
                            \\
                            y[x:=a][y:=b[x:=a]]
                            &=& y[y:=b[x:=a]]
                            \\
                            &=&b[x:=a]
                        \end{array}
                        $$

                    \item Case $z \ne x \land z \ne y$:
                        $$
                        \begin{array}{llll}
                            z[y:=b][x:=a]
                            &=& z
                            \\
                            \\
                            z[x:=a][y:=b[x:=a]]
                            &=& z
                        \end{array}
                        $$

                \end{enumerate}
        \end{enumerate}
    \end{proof}
\end{lemma}

The validity of the
\emph{double substitution lemma}~\ref{DoubleSubstitution}
is needed to make sure that \emph{beta
reduction} (see definition in the next section) is confluent. E.g. if we have
the term $(\lambda x^A. (\lambda y^B. t) b) a$ then we can decide whether we
reduce first the inner redex and then the outer redex or the other way round.
Because of confluence both possibilities shall have the same result.
$$
\begin{array}{llll}
    (\lambda x^A. (\lambda y^B. t) b) a
    &\reduce& (\lambda x^A. t[y:=b]) a
    \\
    &\reduce& t[y:=b][x:=a]
    \\
    \\
    (\lambda x^A. (\lambda y^B. t) b) a
    &\reduce& (\lambda y^B. t[x:=a]) b[x:=a]
    \\
    &\reduce& t[x:=a][y:=b[x:=a]]
\end{array}
$$









\subsection{Beta Reduction}
%%%%%%%%%%%%%%%%%%%%%%%%%%%%%%%%%%%%%%%%%%%%%%%%%%%%%%%%%%%%%%%%%%%%%%%%%%%%%%%%

Like in untyped lambda calculus, computation is done via \emph{beta reduction}.
A beta redex has the form $(\lambda x^A. e) a$ which reduces to the reduct
$e[x:=a]$. Beta reduction can be done in any subterm of a term.

\begin{definition}
    \emph{Beta reduction}
    %%%%%%%%%%%%%%%%%%%%%%%%%%%%%%%%%%%%%%%%%%%%%%%%%%%%%%%%%%%%%%%%%%%%%%%%%%%%
    is a binary relation $a \reduce b$ where the term $a$
    reduces to the term $b$. It is defined by the rules
    \begin{enumerate}

        \item[redex]
            $$
                (\lambda x^A. e) a \reduce e[x:=a]
            $$

        \item Reduce function:
            $$
                \ruleh{
                    f \reduce g
                }
                {
                    f a \reduce g a
                }
            $$

        \item Reduce argument:
            $$
                \ruleh{
                    a \reduce b
                }
                {
                    f a \reduce f b
                }
            $$

        \item Reduce abstraction argument type:
            $$
                \ruleh{
                    A \reduce B
                }
                {
                    \lambda x^A. e \reduce \lambda x^B . e
                }
            $$

        \item Reduce abstraction inner term:
            $$
                \ruleh{
                    e \reduce f
                }
                {
                    \lambda x^A. e \reduce \lambda x^A. f
                }
            $$

        \item Reduce product argument type:
            $$
                \ruleh{
                    A \reduce B
                }
                {
                    \Pi x^A. C \reduce \Pi x^B . C
                }
            $$

        \item Reduce product result type
            $$
                \ruleh{
                    B \reduce C
                }
                {
                    \Pi x^A. B \reduce \Pi x^A. C
                }
            $$
    \end{enumerate}
\end{definition}



\begin{theorem}
    \label{SubstituteReduction}
    %%%%%%%%%%%%%%%%%%%%%%%%%%%%%%%%%%%%%%%%%%%%%%%%%%%%%%%%%%%%%%%%%%%%%%%%%%%%%%%%
    \emph{Substitution respects reduction} A reduction remains valid if we do
    the same substitution before and after the reduction.
    $$
    \ruleh{
        t \reduce u
    }
    {
        t[y:=v] \reduce u[y:=v]
    }
    $$
    (Note that iterated application of this lemma results in the more general
    statement $t \reducestar u \imp t[y:=v] \reducestar u[y:=v]$.)

    \begin{proof}
        Proof by induction on $t \reduce u$.


        \begin{enumerate}

        \item Redex: We have to prove the goal
            $$
                ((\lambda x^A. e) a)[y:=v] \reduce e[x:=a][y:=v]
            $$

            We can see this by the sequence
            $$
            \begin{array}{lll}
                ((\lambda x^A. e) a)[y:=v]
                &=&
                (\lambda x^{A[y:=v]}. e[y:=v]) a[y:=v]
                \\
                &\reduce&
                e[y:=v][x:=a[y:=v]]
                \\
                &=& \text{by Double substitution lemma~\ref{DoubleSubstitution}}
                \\
                &&e[x:=a][y:=v]
            \end{array}
            $$

        \item Reduce function: We have to prove the goal
            $$
            \begin{array}{l|l}
                f \reduce g
                & f[y:=v] \reduce g[y:=v]
                \\
                \hline
                f a \reduce g a
                &(f a)[y:=v] \reduce (g a)[y:=v]
            \end{array}
            $$
            The validity of the final goal in the right lower corner can be seen
            by the following reasoning
            $$
            \begin{array}{lll}
                (f a)[y:=v]
                &=&
                f[y:=v] a[y:=v]
                \\
                &\reduce&
                g[y:=v] a[y:=v]
                \\
                &=&
                (g a)[y:=v]
            \end{array}
            $$

        \item Other rules: All other rules follow the same pattern as the proof
            of the rule \emph{reduce function}.
        \end{enumerate}
    \end{proof}
\end{theorem}





\begin{lemma}
    \label{ReductionProductAbstraction}
    \emph{Product and abstraction are preserved under beta reduction}
    \begin{enumerate}
    \item
    $
        \ruleh{
            \Pi x^A.B \reduce t
        }
        {
            \left( \exists C. A \preduce C \land t = \Pi x^C. B \right)
            \lor
            \left( \exists D. B \preduce C \land t = \Pi x^A. D \right)
        }
    $
    \item
    $
        \ruleh{
            \lambda x^A.e \reduce t
        }
        {
            \left( \exists B. A \preduce B \land t = \lambda x^B. e \right)
            \lor
            \left( \exists f. e \preduce f \land t = \lambda x^A. f \right)
        }
    $
    \end{enumerate}

    \begin{proof}
        The proofs for product and abstraction follow the same pattern.
        Therefore we prove only the preservation of products.

        Assume $\Pi x^A. B \reduce t$ and do induction on it. Only the two
        product rules are syntactically possible. Each one guarantees one
        alternative of the goal.
    \end{proof}
\end{lemma}



\subsection{Beta Equivalence}
%%%%%%%%%%%%%%%%%%%%%%%%%%%%%%%%%%%%%%%%%%%%%%%%%%%%%%%%%%%%%%%%%%%%%%%%%%%%%%%%


\begin{definition}
    We define \emph{beta equivalence} as a binary relation $a \equiv b$ between
    the terms $a$ and $b$ inductively by the rules
    \begin{enumerate}
    \item
        $$ a \equiv a$$

    \item
        $$
        \rulev
        {
            a \equiv b
            \\
            b \reduce c
        }
        {
            a \equiv c
        }
        $$

    \item
        $$
        \rulev
        {
            a \equiv b
            \\
            c \reduce b
        }
        {
            a \equiv c
        }
        $$
    \end{enumerate}

    In other words the beta equivalence relation $\equiv$ is the smallest
    equivalence relation which contains beta reduction $\reduce$.
\end{definition}


\begin{theorem}
    \label{thm:SubstitutionRespectsEquivalence}
    \emph{Substitution respects beta equivalence}.
    $$
    \ruleh{
        t \betaeq u
    }
    { \sub a x t \betaeq \sub a x u}
    $$
    \begin{proof}
        By induction on $t \betaeq u$.

        The reflexive case is trivial. The forward and backward cases use the
        induction hypothesis and the fact that substitution respects beta
        reduction~\ref{SubstituteReduction}
    \end{proof}
\end{theorem}



\subsection{Confluence}
%----------------------------------------------------------------------

MISSING: Add the essence of confluence and remove the other chapter about
confluence.



\subsection{Subtype}
%----------------------------------------------------------------------



\begin{definition}
    \label{def:ProperSubtype}
    %-----------------------------
    The relation $A <_i B$ which says that $A$ is a \emph{proper subtype} of $B$
    with arity $i$ is defined inductively by the rules
    \begin{enumerate}
        \item $\ruleh{j < k} {\Any_j <_0 \Any_k}$

        \item $\ruleh{B <_i C} {\Pi x^A. B <_{i+1} \Pi x^A. C}$
    \end{enumerate}
\end{definition}


\begin{theorem}
    \label{thm:SubstitutionRespectsProperSubtype}
    \emph{Substitution respects proper subtypes}.
    %------------------------------------------------------------
    $$
    \ruleh{
        T <_i U
    }
    {
        \sub a x T <_i \sub a x U
    }
    $$
    \begin{proof}
        By induction on $T <_i U$. The base case is trivial. The inductive case
        is an immediate consequence of the induction hypothesis and the
        definition of substitution.
    \end{proof}
\end{theorem}


\begin{definition}
    \label{def:Subtype}
    %-----------------------------
    The relation $A \le B$ which says that $A$ is a \emph{subtype} of $B$ is
    defined inductively by the rules
    \begin{enumerate}
        \item $\ruleh{A \betaeq B}{A \le B}$

        \item $
            \ruleh
            {A_1 <_i B_1
            \quad A \reducestar A_1
            \quad B \reducestar B_1}
            {A \le B}
            $
    \end{enumerate}
\end{definition}


The definition of the subtype relation has been based on reduction. I could have
been defined based on betaequivalence (i.e. $A \betaeq A_1, B \betaeq A_2$) as
well. The following theorem shows that a definition based on equivalence would
not increase the relation.

\begin{theorem}
    \label {thm:SubtypeEquivalence}
    $$
    \ruleh{
        V_1 <_i V_2
        \\
        T \betaeq V_1
        \\
        U \betaeq V_2
    }
    {
        \exists V_1' V_2'.
        \bracklist{
            V_1' <_i V_2'
            \\
            T \reducestar V_1'
            \\
            U \reducestar V_2'
        }
    }
    $$
    \begin{proof}
        By induction on the definition of the proper subtype relation
        $<_i$.

        The base case is trivial, because a sort can only be betaequivalent with
        another term if it is a reduct of the other term.

        For the inductive step we assume $\Pi x^A. B_1 <_{i+1} \Pi x^A. B_2$
        because of $B_1 <_i B_2$ and $T \betaeq \Pi x^A. B_1$ and $U \betaeq \Pi
        x^A. B_2$.

        By using the Church-Rosser theorem we derive
        $$
        \begin{array}{lcl|lcl}
            T & \betaeq & \Pi x^A.B_1 &
            U & \betaeq & \Pi x^A. B_2
            \\
            \searrow\beta^* & & \swarrow\beta^* &
            \searrow\beta^* & & \swarrow\beta^*
            \\
            & \Pi x^{A_1'}.B_1' & &
            & \Pi x^{A_2'}.B_2'
            \\
            & \downarrow\beta^* & & & \downarrow\beta^*
            \\
            & \Pi x^{A''}. B_1' & &
            & \Pi x^{A''}. B_2'
        \end{array}
        $$

        Because of $B_1' \betaeq B_1$ and $B_2' \betaeq B_2$ we derive from the
        induction hypothesis the existence of $B_1''$ and
        $B_2''$ such that $B_1' \reducestar B_2''$, $B_2' \reducestar B_2''$ and
        $B_1'' <_i B_2''$ and therfore $\Pi x^{A''}.B_1'' <_{i+1} \Pi x^{A''}.
        B_2''$.

        Using $V_1' = \Pi x^{A''}. B_1''$ and $V_2' = \Pi x^{A''}. B_2''$ proves
        the goal.
    \end{proof}
\end{theorem}



\begin{theorem}
    \label{thm:SubstitutionRespectsSubtype}
    \emph{Substitution respects subtypes}.
    %------------------------------------------------------------
    $$
    \ruleh{
        T \le U
    }
    {
        \sub a x T \le \sub a x U
    }
    $$
    \begin{proof}
        By induction on $T \le U$.

        The equivalence case follows from the fact that substitution respects
        beta equivalence~\ref{thm:SubstitutionRespectsEquivalence}. Furthermore
        substitution respects proper
        subtypes~\ref{thm:SubstitutionRespectsProperSubtype}. This implies the
        second case.
    \end{proof}
\end{theorem}





\begin{theorem}
    \label{thm:ProductSubtype}
    %------------------------------------------------------------
    $$
    \ruleh{
        \Pi x^A. B \le \Pi x^{A'}. B'
    }
    {
        A \betaeq A' \land B \le B'
    }
    $$
    \begin{proof}
        MISSING!!
    \end{proof}
\end{theorem}





\begin{theorem}
    \label{thm:SubtypeAntisymmetric}
    %------------------------------------------------------------
    $$
    \ruleh{
        A \le B
        \\
        B \le A
    }
    {
        A \betaeq B
    }
    $$
    \begin{proof}
        MISSING!!
    \end{proof}
\end{theorem}
