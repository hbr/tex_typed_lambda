\section{Quasinormalization}





\subsection{Levels}

To motivate the definition of levels let's consider the following reduction
$$
    (\lambda x^{\Any_i}. x) \Prop \reduce \Prop
$$
where the left hand side and the right hand side have the minimal sorts
$$
    \begin{array}{lll}
        (\lambda x^{\Any_i}) \Prop &:& \Any_i
        \\
        \Prop &:& \Any_0
    \end{array}
$$
I.e. the minimal sort can go down via beta reduction. However we want to
characterize a type by a sort/universe which is independent of reduction.


\begin{definition}
    A \emph{level} $l$ is a set of sorts which is upper closed i.e.
    $$
    \rulev{s_1 \in l \\ s_1 \le s_2}{s_2 \in l}
    $$
\end{definition}

Because levels are upper closed, a level which is a superset of another level
must contain extra sorts which are lower than the other level.

We define the following levels
$$
\begin{array}{lll}
    l_\Prop &:=& \{\Prop, \Any_0, \Any_1, \ldots\}
    \\
    l_{\Any_0} &:=& \{ \Any_0, \Any_1, \Any_2 \ldots \}
    \\
    \ldots
    \\
    l_{\Any_i} &:=& \{ \Any_i, \Any_{i+1}, \Any_{i+2}, \ldots \}
\end{array}
$$
with
$$
    l_\Prop \supseteq \Any_0 \supseteq \Any_1 \ldots
$$


\begin{definition}
    The level $\Level T$ of a type $T$ is the set of all sorts/universes it
    lives in modulo reduction.

    The relation $\Level$ is defined inductively by the rule
    $$
    \rulev
    {
        \Gamma \vdash B: s_1
        \\
        \Gamma \vdash C: s
        \\
        B \reducestar C
    }
    {\Level B s}
    $$
    where $\Level T s$ means that $s$ in the level of the type $T$.
\end{definition}

The so defined level is really a level i.e. upperclosed because of the
cumulativity.


\begin{theorem}
    \emph{Typesafe substitution does not increase the level of a type}.
    $$
    \rulev
    {
        \Gamma \vdash a: A
        \\
        \Gamma, x^A, \Delta \vdash B: s
    }
    {\Level B \subseteq \Level B[x:=a]}
    $$


    \begin{proof}
    \begin{enumerate}
        \item We have to proof that all sorts $s$ with $\Level B s$ satisfy
        $\Level B[x:=a] s$.

        \item In order for $\Level B s$ to be valid there exists a type $C$ with $B
        \reducestar C$ and $\Gamma, x^A, \Delta \vdash B: s$.

        \item By the substition lemma we get $\Gamma, \Delta[x:=a] \vdash C[x:=a]: s$.

        \item Since reduction and substitution are compatible
            (Lemma~\ref{SubstituteReduction})
            we can conclude $B[x:=a]
        \reducestar C[x:=a]$  from $B \reducestar C$.

        \item Therefore by definition $\Level B[x:=a] s$ is valid.
    \end{enumerate}
    \end{proof}
\end{theorem}



\begin{definition}
    The level $\ell t$ of a term $t$ is the union of all the levels of its
    types. The term level is defined by the rule
    $$
    \rulev{
        \Gamma \vdash t: T
        \\
        \Level T s
    }
    {
        \ell t s
    }
    $$
\end{definition}

For term levels we use the same ordering as for type levels.


\begin{theorem}
    The term level $\ell f$ of the function term of an application $f a$ where
    $f a$ is a valid term in some context $\Gamma$ is a subset of the term level
    of the application
    $$
    \ell f \subseteq \ell (f a)
    $$

    \begin{proof}

        We have to prove that $\ell f s$ implies $\ell (f a) s$ for all sorts $s$.

        MISSING
    \end{proof}
\end{theorem}


\subsection{Quasinormalization}


\begin{theorem}
    \label{quasinormalization}
    \emph{Quasinormalization theorem}

    MISSING!!
\end{theorem}
