%-------------------------------------------------------------------------------
\section{Quasinormalization}
%-------------------------------------------------------------------------------


\subsection{Overview}
%-------------------------------------------------------------------------------


In this section we prove that all welltyped terms can be reduced to an $1$
normal form. A term is in $1$ normal form if it has no redexes of level
$1$ or higher.

The level of a type is the level of the lowest sort the type lives in up to
reduction. The level of a redex is the level of the type of its function term.

Here we give an overview of the proof. The needed definitions and theorems will
be made precise in the subsequent sections.

A term $t$ is welltyped if there is a context $\Gamma$ and a type term $T$ such
that $\Gamma \vdash t : T$ is satisfied. Without restriction of generality we
can assume that $\Gamma \vdash t : T$ has been derived with a finite set of
sorts $S_n$. To find $n$ just look into the derivation of $\Gamma \vdash t : T$
and find the highest sort $\Any_n$. Then it is guaranteed that the derivation is
possible with the finite sort set $S_n$.

If we restrive ourselves to the finite sort set $S_n$ then the highest level of
all types is $n$ and the highest level of any redex is $n$ as well.
I.e. all derivable terms are in $n+1$ normal form.

We want to prove that we can transform any welltyped term $t$ derived with the
finite sort set $S_n$ into $1$ normal form. We do this by downward
induction from $n+1$ to $1$.

The induction start is trivial, because all terms derived with the finite sort
set $S_n$ are in $n+1$ normal form.

For the induction step from $i+1$ down to $i$ we can assume the induction
hypothesis that we can reduce any welltyped term to $i+1$ normal form.




\begin{comment}

    Basic proof idea
    ============================================================

    Levels of a redex (via level of the type of the function term)
    ------------------------------------------------------------

    Each term which is not in normal form has a nonempty set of redexes. Each of
    these redexes has a function term and this function term has a minimal type.

    Each type has a level which is its lowest sort modulo reduction. Therefore
    we can say that each redex has a level (the level of the minimal type of its
    function term). The number of redexes in a term is finite. There is a
    highest level of all redexes a term.

    We say that a term is s-normal, if it has no redexes of level s or higher.
    If s is the highest level of a redex in a term, then the term is s+1 normal.


    Degree of a redex
    ------------------------------------------------------------

    For all terms which are s+1 normal it is possible the define a degree for
    all s level redexes.

    We can reduce each term where s is the highest level of all its redexes into
    s-normal form. We do this by repeatedly reducing the rightmost redex of
    level s.



\end{comment}



\begin{comment}
    Level:

        A set of sorts a type can live in, even if beta reduced.

        Note:
            Beta reduction can reduce the sort. E.g.

                (\ (A: Any (i+k+1)) := A) (Any i) ~> Any i

            where `Any i` has type  `Any (i+1)` and
            `(\ (A: Any (i+k+1)) := A) (Any i)` has type `Any (i+k+1)` because
            the result of the function term is `A` which has type `Any (i+k+1)`.


    Level of a redex:

        All redexes have a function term which has a minimal type of the form
        'all (x: A): B' (or beta equivalents). The level of this type is the
        level of a redex.


    j-normal term:

        Does not have redexes of level 'upper (Any j)' or subsets of it.

        0-normal terms are normal terms, because they contain no redexes.


    Proof idea:

        In ECCn a redex with sort set upper (Any n) is not possible. Therefore
        we have a function rn which reduces any term to its n-normal form. Since
        all terms are in n-normal form rn is the identity function.

        Induction step: Assume there is a reduction function ri+1, find a
        reduction function ri.

            We look into redexes whose sort set of its function term type is
            upper (Any i). We can transform all these types into i+1 normal form
            via ri+1. The normalized type is a sort or a product or a base term.
            We can compute the degree of all these types.

            We use the triple

                (i, d, n)

            where d is the maximal degree of all these types and n is the number
            of redexes which have this maximal degree.

            We reduce the rightmost redex of degree d and claim that this
            reduces n by one. We continue until n=0 i.e. there are no more
            redexes of degree d.

            We continue this process until there are no more redexes of level i.

            The i-normal term is unique because the rightmost redex is unique.
            Therefore we have found the reduction function ri.

        Crucial in the construction: The reduction of the rightmost redex of the
        highest degree does not make new redexes of this degree.
\end{comment}





\subsection{Sort Sets}
%----------------------------------------------------------------------

In the extended calculus of constructions there are functions which have
non-propositional types as arguments and non-propositional types as values. E.g.
the identity function $\lambda x^{\Any_i}. x$ with type $\Any_i \to \Any_i$ for
$i > 0$ can accept arguments $T$ of type $\Any_j$ with $j \le i$ and it returns
a value of type $\Any_i$.

However if we consider the beta reduction $(\lambda x^{\Any_i}. x) T \reduce T$
then the types of redex and the reduct are different.
$$
\begin{array}{lll}
    (\lambda x^{\Any_i}. x) T & : & \Any_i[T/x] = \Any_i
    \\
    T & : & \Any_j
\end{array}
$$

I.e. the minimal type of a type term is not invariant against beta reduction. We
define the \emph{level of type} as the level of the lowest sort it lives in up
to beta reduction.



\begin{definition}
    \label{def:UpperSet}
    %-----------------------------
    The \emph{upper set} $\upper s$ of a sort $s$ are all sorts
    $s_2$ where $s \le s_2$.
    $$
    \upper s := \set{s_2 \mid s \le s_2}
    $$
\end{definition}




\begin{definition}
    \label{def:SortSet}
    %-----------------------------
    The \emph{sort set $S_\Gamma(T)$ of a type $T$} is defined inductively by
    the rule
    $$
    \ruleh{
        \Gamma \vdash T : s_1\quad
        T \reducestar U\quad
        \Gamma \vdash U : s
    }
    {
        s \in S_\Gamma(T)
    }
    $$

    We often write $S(T)$ without the subscript $\Gamma$ when the context is
    clear.
\end{definition}





\begin{theorem}
    \label{thm:SortSetUpperClosed}
    %-----------------------------
    \emph{A sort set of a type is upper closed}.
    $$
        \ruleh{
            s_1 \in S_\Gamma(T)
            \\
            s_1 \le s_2
        }
        {
            s_2 \in S_\Gamma(T)
        }
    $$
    \begin{proof}
        By the cumulativity rule of the typing relation.
    \end{proof}
\end{theorem}


\begin{theorem}
    \label{thm:MinimalSortDecreases}
    \emph{The minimal sort of a type might decrease after reduction}
    $$
    \ruleh{
        A \reduce B
        \quad \Gamma \vdashmin A: s_1
        \quad \Gamma \vdashmin B: s_2
    }
    {
        s_2 \le s_1
    }
    $$

    \begin{proof}
        By subject reduction~\ref{SubjectReduction} we have $\Gamma \vdash B:
        s_1$. Since $s_2$ is the minimal type of $B$ we have $s_2 \le s_1$.
    \end{proof}

\end{theorem}




\begin{theorem}
    \emph{The sort set of a type has a unique minimal sort which we call the
    level of the type.}

    \begin{proof}
        If we have a reduction sequence of a type $A_0 \reduce A_1 \reduce
        \ldots$ we know from~\ref{thm:MinimalSortDecreases} that the
        corresponding sequence of minimal sorts $s_0 \ge s_1 \ge s_2 \ge
        \ldots$ is a decreasing sequence. Since the order of the sorts is
        wellfounded, the sequence must reach a minimal sort.
    \end{proof}
\end{theorem}


\begin{theorem}
    \label{thm:SortSetSubstitution}
    %-----------------------------
    \emph{Type safe substitution makes the sort set bigger}.
    $$
        \ruleh{
            \Gamma \vdash a : A \quad \Gamma,x^A \vdash B
        }
        {
            S_{\Gamma,x^A}(B) \subseteq S_\Gamma(B[a/x])
        }
    $$
    \begin{proof}
        Assume $s \in S_{\Gamma,x^A}(B)$.

        Then by the definition of sort sets~\ref{def:SortSet}
        there exists $C$ with $B \reducestar C$ and $\Gamma,x^A \vdash C: s$.
        This implies $B[a/x] \reducestar C[a/x]$ and $\Gamma \vdash C[a/x] : s$
        by the subject reduction~\ref{SubjectReduction} and the substitution
        theorem~\ref{SubstitutionLemma}.

        The definition of sort sets~\ref{def:SortSet} implies $s \in
        S_\Gamma(B[a/x])$ which completes the proof.
    \end{proof}
\end{theorem}






\begin{theorem}
    \label{thm:SortSetProduct}
    %-----------------------------
    \emph{The sort set of a product is either $\upper {s_\Prop}$ in case the
    product is a proposition (i.e. has type $s_\Prop$) or the intersection of
the sort sets of the argument and the result type.}
    $$
    \ruleh{
        \Gamma \vdash \Pi x^A. B : \dontcare
    }
    {
        S(\Pi x^A. B) =
        \dcases{
            \upper{s_\Prop} & \text{if }\Gamma \vdash \Pi x^A . B : s_\Prop
            \\
            S(A) \cap S(B) & \text{otherwise}
        }
    }
    $$

    This theorem has the immediate consequence $S(\Pi x^A. B) \subseteq S(B)$. In
    the propositional case this is trivial. In the non-propositional case it is
    a consequence of the intersection.

    \begin{proof}
        Now we prove the main theorem.

        If the type of the product is a propositional sort $s_\Prop$ then
        $\upper{s_\Prop}$ already is the greatest possible sort set for the
        product.

        If a propositional sort $s_\Prop$ cannot be the type of the product we
        have to prove both directions:
        \begin{enumerate}
            \item $S(\Pi x^A. B) \subseteq S(A) \cap S(B)$:
                Assume $s$ is in the sort set of the product. By definition
                there exist reducts $A'$ and $B'$ of $A$ and $B$ such that
                $\Gamma \vdash \Pi x^{A'}. B' : s$. By the generation
                lemma~\ref{GenerationLemma} for products there exist a sort
                $s_0$ which is a valid type of $A'$ and $B'$ with $s_0 \le s$.
                Therefore $s_0$ is in the sort sets of $A$ and $B$. Because sort
                sets are upper closed $s$ is in both sort sets as well.

            \item $S(A) \cap S(B) \subseteq S(\Pi x^A. B)$:
                Assume $s$ is in both sort sets of $A$ and $B$. By definition
                there exist reducts $A'$ and $B'$ such that $s$ is a type of
                $A'$ and $B'$. The introduction rule for products in the typing
                relation says that $s$ is a type of the product $\Pi x^{A'}.
                B'$. By definition of the sort set $s$ is in the sort set of
                $\Pi x^A. B$.
        \end{enumerate}
    \end{proof}
\end{theorem}






\begin{theorem}
    \label{thm:SortSetsSubtype}
    %-----------------------------
    \emph{Let $A$ and $B$ be some welltyped terms in the same context. Then the
    following is valid:}
    \begin{enumerate}

        \item $A \betaeq B \imp S(A) = S(B)$

        \item $A <_i B \imp S(B) \subset S(A)$
    \end{enumerate}
    \begin{proof}

        \ \begin{enumerate}

            \item Because of confluence there exists a common reduct $C$ of $A$
                and $B$. Then by definition of the sort set $S(A) = S(C) =
                S(B)$.

            \item By induction on $ A <_i B$.
                \begin{enumerate}
                    \item The base case is trivial.

                    \item In the induction step we have $\Pi x^A . B_1 <_{i+1}
                        \Pi x^A. B_2$ where $S(B_2) \subset S(B_1)$ is given by
                        the induction hypothesis.

                        In case that $\Pi x^A. B_1$ is a proposition the goal is
                        trivial because $\upper\Prop$ is the largest possible
                        sort set.

                        It remains the case that both are not propositions ($\Pi
                        x^A. B_2$ cannot be a proposition because it is a proper
                        supertype).

                        Then we have
                        $$
                        S(\Pi x^A. B_2) = S(A) \cap S(B_2) \subset S(A) \cap
                        S(B_2) = S(\Pi x^A. B_1)
                        $$
                        by using the induction hypothesis
                        and~\ref{thm:SortSetProduct}.
                \end{enumerate}
        \end{enumerate}
    \end{proof}
\end{theorem}






\begin{theorem}
    \label{thm:SortSetApplication}
    %-----------------------------
    \emph{Sort sets of applications} Let $fa$ be an application where $F$ is a
    minimal type of $f$ and $R$ is a minimal type of $fa$. Then the sort set of
    $F$ is a subset of the sort set of $R$.
    $$
    \rulev{
        \Gamma \vdashmin f : F
        \\
        \Gamma \vdashmin fa : R
    }
    {
        S(F) \subseteq S(R)
    }
    $$
    \begin{proof}

        By~\ref{thm:MinimalTypeFunctionTerm} there exist the types $A$ and $B$
        such that $\Pi x^A. B$ is a minimal type of $f$. Therefore $F$ and $\Pi
        x^A.B$ are beta equivalent and have the same sort sets.

        The generation lemma~\ref{GenerationLemma} for application guarantees
        the existence of $A'$ and $B'$ such that $\Pi x^{A'}.B'$ is a valid type
        of $f$ and $A'$ is a valid type of $a$ and $\sub a x {B'} \le R$.

        Since $\Pi x^A. B$ is minimal we have $A \betaeq A'$ and $B \le B'$
        by~\ref{thm:ProductSubtype}.

        The minimality of $R$, the antisymmetry of
        subtypes~\ref{thm:SubtypeAntisymmetric} and the fact that
        substitution respects subtypes~\ref{thm:SubstitutionRespectsSubtype} we
        have $\sub a x B \betaeq R$.

        In combination with \ref{thm:SortSetsSubtype} and
        \ref{thm:SortSetProduct} and \ref{thm:SortSetSubstitution} we get
        $$
        S(F) = S(\Pi x^A. B)
            \subseteq S(B)
            \subseteq S(\sub a x B)
            = S(R)
        $$
    \end{proof}
\end{theorem}






\begin{theorem}
    \label{thm:SortSetApplications}
    %-----------------------------
    \emph{Sort sets of generalized applications} This is a generalization of the
    previous theorem~\ref{thm:SortSetApplication}.

    Let $fa \vec b$ be an application where
    $F$ is a minimal type of $f$ and $R$ is a minimal type of $fa \vec b$ where
    $\vec b$ is a possibly empty sequence of arguments $b_1, b_2, \ldots, b_n$.
    Then the sort set of $F$ is a subset of the sort set of $R$.
    $$
    \rulev{
        \Gamma \vdashmin f : F
        \\
        \Gamma \vdashmin f a \vec b : R
    }
    {
        S(F) \subseteq S(R)
    }
    $$
    \begin{proof}
        We prove the goal by induction on the length $n$ of $\vec b$.
        \begin{enumerate}
            \item $n=0$: Immediate consequence of the previous
                theorem~\ref{thm:SortSetApplication}.

            \item Induction step from $n$ to $n+1$:
                Assume $S(F) \subseteq S(R)$ for $n = |\vec b|$. We have to
                prove the goal
                $$
                \rulev{
                    \Gamma \vdashmin f a \vec b b_{n+1} : R'
                }
                {
                    S(F) \subseteq S(R')
                }
                $$
                By using the the induction hypothesis and the previous
                theorem~\ref{thm:SortSetApplication} we can prove the goal
                $$
                    S(F) \subseteq S(R) \subseteq S(R')
                $$
        \end{enumerate}
    \end{proof}
\end{theorem}






\subsection{Level}
%------------------------------------------------------------


\begin{definition}
    The \emph{Level} $\Level(T)$ of a type T is the level of the minimal sort
    the type lives in up to reduction.
\end{definition}


\begin{theorem}
    \label{thm:LevelTypeSafeSubstitution}
    %------------------------------------------------------------
    Type safe substitution decreases the level of a type.
    $$
    \rulev{
        \Gamma \vdash a : A
        \\
        \Gamma, x^A \vdash B: s
    }
    {
        \Level(B[a/x]) \le \Level(B)
    }
    $$
    \begin{proof}
        Let's say the level of $B$ is $i$. Then there exists by definition of
        the level a reduct $C$ of $B$ such that $\Gamma, x^A \vdash C: \Any_i$
        is valid. By the substitution lemma~\ref{SubstitutionLemma} we get
        $\Gamma \vdash C[a/x]: \Any_i$. Since $C[a/x]$ lives in the sort
        $\Any_i$ and is a reduct of $B[a/x]$ the level of $B[a/x]$ is less or
        equal $i$.
    \end{proof}
\end{theorem}




\begin{theorem}
    \label{thm:LevelProduct}
    %------------------------------------------------------------
    The level of a nonpropositional product is the maximal level of the argument
    type and the result type.
    $$
    \ruleh{
        \Level(B) > 0
    }
    {
        \Level(\Pi x^A.B) = \max(\Level(A), \Level(B))
    }
    $$
    \begin{proof}
        Because of cumulativity we have
        $$
        \rulev{
            \Gamma \vdash A: \Any_i
            \\
            \Gamma,x^A \vdash B: \Any_j
            \\
            i,j > 0
        }
        {
            \Gamma \vdash \Pi x^A. B : \Any_{\max(i,j)}
        }
        $$
        Since reduction of a product does not change its shape, the levels of
        the argument and result type determine uniquely the level of the
        product.
    \end{proof}
\end{theorem}




\begin{theorem}
    \label{thm:LevelSubtypeEquivalence}
    %------------------------------------------------------------
    The level of a type respects subtyping and betaequivalence.
    \begin{enumerate}
        \item $ A \betaeq B \imp \Level(A) = \Level(B)$

        \item $ A <_i B \imp \Level(A) \le \Level(B) $
    \end{enumerate}

    \begin{proof}
        \ \begin{enumerate}
            \item In case of betaequivalence $A$ and $B$ have a common reduct
                $C$ with $\Level(A) = \Level(C) = \Level(B)$.

            \item By induction on $<_i$.
                \begin{enumerate}
                    \item The base case is trivial.

                    \item For the induction step we assume $B_1 <_i B_2 \imp
                        \Level(B_1) \le \Level(B_2)$ and have to prove
                        $$
                            \Level(\Pi x^A.B_1) \le \Level(\Pi x^A. B_2)
                        $$
                        under the assumption $ \Pi x^A. B_1 <_{i+1} \Pi x^A.
                        B_2$ (which implies $\Level(B_1) \le \Level(B_2)$ by the
                        induction hypothesis.

                        If $B_1$ is a proposition i.e. $\Level(B_1) = 0$, then
                        the goal is trivial.

                        So we assume $B_1$ is not a
                        proposition which implies that $B_2$ is not a
                        proposition either. With the help of
                        theorem~\ref{thm:LevelProduct} we derive
                        $$
                        \begin{array}{lll}
                            \Level(\Pi x^A. B_1)
                            & = & \max(\Level(A), \Level(B_1)
                            \\
                            & \le & \max(\Level(A), \Level(B_2))
                            \\
                            & = & \Level(\Pi x^A. B_2)
                        \end{array}
                        $$
                \end{enumerate}
        \end{enumerate}
    \end{proof}
\end{theorem}



\begin{theorem}
    \label{thm:LevelApplication}
    %------------------------------------------------------------
    Let $f a$ be an application where $F$ is a minimal type of $f$ and $R$ is a
    minimal type of $f a$. Then the level of $F$ is greater equal the level of
    $R$.
    $$
    \rulev{
        \Gamma \vdashmin f : F
        \\
        \Gamma \vdashmin f a: R
    }
    {
        \Level(F) \ge \Level(R)
    }
    $$
    \begin{proof}
        Without loss of generality we assume that $F$ is not a proposition,
        because for propositions the goal is trivial.

        Since $f$ is a function positition according
        to~\ref{thm:MinimalTypeFunctionTerm} there is a minimal type of the form
        $\Pi x^A.B$ betaequivalent to $F$.
        $$
        \begin{array}{lllll}
            \Level(F)
            & = & \Level(\Pi x^A.B) &~\ref{thm:LevelSubtypeEquivalence}
            \\
            & \ge  & \Level(B)
            &~\ref{thm:LevelProduct}
            \\
            & \ge & \Level(B[a/x]
            &~\ref{thm:LevelTypeSafeSubstitution}
            \\
            & = & \Level(R)
        \end{array}
        $$
    \end{proof}
\end{theorem}


\begin{theorem}
    \label{thm:LevelRepeatedApplication}
    %------------------------------------------------------------
    Let $f a_1 \ldots a_n$ is an application where $F$ is a minimal type of $f$
    and $R_i$ is a minimal type of $f a_1 \ldots a_i$ with $i \le n$. Then the
    level of $F$ is greater equal to the level of $R_n$.

    \begin{proof}
        We prove this fact by repeated application of the previous
        theorem~\ref{thm:LevelApplication}.
        $$
        \begin{array}{lll}
            \Level(F) & \ge & \Level(R_1)
            \\
            & \ge & \Level(R_2)
            \\
            \ldots
            \\
            & \ge & \Level(R_n)
        \end{array}
        $$
    \end{proof}
\end{theorem}








\subsection{$i$-Normal Terms}
%------------------------------------------------------------

With the help of the level of a type we define the level of a redex by
\begin{definition}
    \emph{Level of a redex}
    %------------------------------------------------------------

    The level of a redex $(\lambda x^A. e) a$ is the level of the minimal type of
    the function term $\lambda x^A. e$.
\end{definition}

The level of a redex is welldefined because all minimal types are betaequivalent
and due to~\ref{thm:LevelSubtypeEquivalence} all betaequivalent types have the
same level.



\begin{definition}
    \emph{Welltyped $i$-normal terms}

    A welltyped term $t$ is $i$-normal if it contains no redexes with a level of
    $i$ or higher.
\end{definition}




\begin{theorem}
    \label{thm:NormalBaseTerm}
    %-----------------------------
    \emph{Let $fa$ be a welltyped $i+1$-normal type whose level is $i$ or lower
    i.e. $\Level(f a) \le i$.
    In that case $f a$ is a base term.}

    \begin{proof}
        We prove the more general theorem using $f a \vec b$ as the welltyped
        $i+1$-normal term with $\Level(f a \vec b) \le i$ where $\vec b$ is a
        possibly empty sequence of additional arguments.
        We prove that $f a \vec b$ is a base term.

        Since $f a \vec b$ is a type, its type has to be a sort i.e. we have
        $\Gamma \vdashmin f a \vec b: \Any_j$ for some $\Gamma$ and $j$. Since the
        level of the type is $i$ or lower, we have $i \le j$.

        The proof goes by induction on the structure of the function term $f$.
        \begin{enumerate}

            \item $f$ can neither be a sort nor a product. This would contradict
                the fact that $f b \vec c$ is welltyped.

            \item If $f$ is a variable then $f a \vec b$ is by definition a base
                term.

            \item $f$ is the application $ga_0$: In that case $ga_0 a \vec b$ is
                a welltyped $i+1$-normal type of level $i$ or lower.
                From the induction hypothesis for $g$ we conclude that $g a_0
                a \vec b = f a \vec b$ is a base term.

            \item $f$ is an abstraction:
                Say $F$ is a minimal type of $f$.
                By~\ref{thm:LevelRepeatedApplication} we have $\Level(F) \ge
                \Level(\Any_j) = j + 1 > i$. This contradicts the assumption that
                $f a \vec b$ is $i+1$ normal i.e. does not have redexes of level
                $i + 1$ or higher. Therefore $f$ cannot be an abstraction.
        \end{enumerate}
    \end{proof}
\end{theorem}


\begin{theorem}
    \label{thm:FormNormalTypes}
    %------------------------------------------------------------
    \emph{Form of Normal Types} A welltyped $i+1$-normal type $T$ whose level is
    $i$ or higher is either a sort, a product or a base term.


    \begin{proof}
        By induction of the structure of $T$.
        \begin{itemize}

            \item If $T$ is a sort or a product or a variable then the goal is
                trivial.

            \item $T$ cannot be an abstraction because the type of an
                abstraction cannot be a sort i.e. it cannot be a welltyped type.

            \item If $T$ is an application it is a base term. This is an
                immediate consequence of~\ref{thm:NormalBaseTerm}.
        \end{itemize}
    \end{proof}
\end{theorem}



\subsection{Degree}
%------------------------------------------------------------

\begin{definition}
    \label{def:TypeDegree}
    \emph{The degree of a welltyped $s+1$-normal type is defined recursively by}

    $$
    D_s(T) :=
    \dcases{
        D_s(T) &:=& 0 \quad\text{if } \upper s \subset S(T)
        \\
        D_s(s') &:=& 0
        \\
        D_s(x \vec a) &:=& 0
        \\
        D_s(\Pi x^A.B) &:=&
        1 + D_s(A) +  D_s(B)
    }
    $$

    Note that the degree is welldefined by~\ref{thm:FormNormalTypes} which
    asserts that the type $T$ is either a sort or a base term or a product if it
    is $s+1$-normal and
    its sort set is a subset of $\upper s$.
    Furthermore subexpressions of $s+1$ normal types are $s+1$ normal as well.
\end{definition}



\begin{definition}
    \label{def:RedexDegree}
    \emph{Degree of a redex} Assume we have a function $r$ to transform any
    welltyped term $t$ into $s+1$-normal form $r(t)$ by a finite sequence of
    beta reductions. Assume further that the term $t$ is $s+1$-normal (if not,
    we can use $r$ to reduce it into $s+1$-normal form). Then by definition the
    term $t$ only has redexes whose sort set is a subset of $\upper s$.

    We define the degree of a redex $(\lambda x^A.e)a$ in $t$ by $D_s(r(T))$
    where $T$ is a minimal type of its function term $\lambda x^A.e$.
\end{definition}







\subsection{Normalization}
%------------------------------------------------------------


\begin{definition}
    \label{def:NormalizationStep}
    \emph{Normalization step of an $s+1$-normal term}
    Let $r$ be a function over welltyped terms $t$ such that $r(t)$ is
    $s+1$-normal. Assume that the term $a$ is $s+1$ normal and has redexes with
    sort set $\upper s$. Each of these redexes has a degree according to
    the definition~\ref{def:RedexDegree}.

    We define $a \torel {q_s} b$ as the reduction relation
    which reduces the rightmost redex of sort set $\upper s$.

    This definition has the following important consequences:
    \begin{itemize}
        \item
            If the highest degree $d$ of the redexes in the term $t$ is zero,
            then the term is $s$-normal.

            Reason: There is always a minimal type in $s+1$-normal form of the
            function term of the redex of the form $\Pi x^A. B$. By definition
            of the degree of a type~\ref{def:TypeDegree} its degree is at least
            $1$ as long as its sort set is a subset of $\upper s$. If the
            degree is zero, then its sort set is a proper superset of $\upper
            s$. Therefore $t$ has no redexes whose sort set is a subset of
            $\upper s$ and thus by definition is $s$-normal.

        \item
            If $a \torel {q_s} b$ is a normalization step of the $s+1$ normal
            term $a$, then the term $b$ is uniquely defined because the
            rightmost redex of degree with sort set $\upper s$ is unique.

        \item The reduction $a \torel {q_s} b$ does not copy any existing
            redexes with sort set $\upper s$, because the contracted redex is
            the rightmost redex.
    \end{itemize}
\end{definition}


\begin{theorem}
    \label{thm:NewRedexCreation}
    \emph{Creation of new redexes during beta reduction} There are only three
    possibilities to create new redexes during a beta reduction which does the
    reduction $(\lambda x^A.e) a \reduce \sub a x e$ on any subterm of a term.

    $$
    \begin{array}{lll}
        (\lambda x^A. \ldots xb \ldots) (\lambda y^B. e)
        &\reduce&
        \ldots (\lambda y^B.e) b \ldots
        %
        \\
        %
        (\lambda x^A. x)(\lambda y^B. e) b
        &\reduce&
        (\lambda y^B. e) b
        %
        \\
        %
        (\lambda x^A . \lambda y^B. e) a b
        &\reduce&
        (\lambda y^{B[a/x]}. e[a/x]) b
    \end{array}
    $$
    where $xb$ in the first case is not the argument of an application i.e. $xb$
    is in a head position.

    \begin{proof}
        A redex consists of an abstraction applied to an argument. Both must
        have been in the original expression and come together after the
        reduction. If the abstraction is in the argument of the redex then the
        argument must be the abstraction. The abstraction replaces the bound
        variable of the original redex.

        If the original redex has $x b$ in a head position, then the
        substitution of the abstraction for $x$ creates a new redex. This is the
        first case.

        If the original redex has only the bound variable $x$ but not applied to
        an argument, then the body of the original redex must be $x$ and the
        argument comes from an argument in the original expression which follows
        the original redex. This is the second case.

        The third possibility has an abstraction without argument (otherwise it
        were already a redex) as the body of the original abstraction and the
        argument of the newly created redex is the argument following the
        argument of the original redex.
    \end{proof}
\end{theorem}



\begin{theorem}
    \emph{A sequence of normalization steps
        according to
        definition~\ref{def:NormalizationStep}
        of an $s+1$-normal term $t$
        when $s$
        is not a propositional sort
        ends after a finite number of steps
        in an $s$-normal unique term.}

    \begin{proof}

        Let
        $$
            n_d n_{d-1} \ldots n_2 n_1
        $$
        be a sequence of numbers where $d$ is the maximal degree of all redexes
        in $t$ with sort set $\upper s$ where $s$ is not a propositional sort.
        The number $n_i$ is the number of redexes in the term $t$ of degree $i$
        and with sort set $\upper s$.

        Then in the reduction sequence
        $$
            t = t_0 \torel {q_s} t_1 \torel{q_s} t_2 \torel{q_s} \ldots
        $$
        each reduction step decreases the number sequence in the lexicographic
        order. Since the lexicographic order is a wellorder, the reduction
        sequence ends after a finite number of steps.
    \end{proof}
\end{theorem}
