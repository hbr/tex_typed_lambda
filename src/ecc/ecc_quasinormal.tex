%-------------------------------------------------------------------------------
\section{Quasinormalization}
%-------------------------------------------------------------------------------

\begin{comment}
    Level:

        A set of sorts a type can live in, even if beta reduced.

        Note:
            Beta reduction can reduce the sort. E.g.

                (\ (A: Any (i+k+1)) := A) (Any i) ~> Any i

            where `Any i` has type  `Any (i+1)` and
            `(\ (A: Any (i+k+1)) := A) (Any i)` has type `Any (i+k+1)` because
            the result of the function term is `A` which has type `Any (i+k+1)`.


    Level of a redex:

        All redexes have a function term which has a minimal type of the form
        'all (x: A): B' (or beta equivalents). The level of this type is the
        level of a redex.


    j-Quasinormal term:

        Does not have redexes of level 'upper (Any j)' or subsets of it.

        "Quasi" is not necessary. We can speak of j-normal terms. 0-normal terms
        are normal terms, because they contain no redexes.


    Proof idea:

        In ECCn a redex of level n is not possible. Therefore we have a function
        rn which reduces any term to its n-normal form. Since all terms are in
        n-normal form rn is the identity function.

        Induction step: Assume there is a reduction function ri+1, find a
        reduction function ri.

            We have the term t and reduce it via ri+1 to its i+1 normal form.
            This term has redexes of level i and lower. We want to remove all
            level i redexes.

            The type of the function term of an i level redex can be transformed
            via ri+1 into i+1 normal form.


        Needed theorem: If T is the type of the function term of a redex in a
        i+1 normal term, then T is either a base term or a sort or a product.

        Or maybe more general: If T is a type in i+1 normal form, then it is
        either a base term or a sort or a product.

            It might contain redexes of level i, but not of level i+1.
\end{comment}








\subsection{Sort Sets, $i$-Normal Terms}
%----------------------------------------------------------------------


\begin{definition}
    \label{def:UpperSet}
    %-----------------------------
    The \emph{upper set} $\upper {\Any_i}$ of a sort $\Any_i$ are all sorts
    $\Any_j$ where $i \le j$.
    $$
    \upper {\Any_i} := \set{\Any_j \mid i \le j}
    $$
\end{definition}




\begin{definition}
    \label{def:SortSet}
    %-----------------------------
    The \emph{sort set $S_\Gamma(T)$ of a type $T$} is defined inductively by
    the rule
    $$
    \ruleh{
        \Gamma \vdash T : s_1\quad
        T \reducestar U\quad
        \Gamma \vdash U : s
    }
    {
        s \in S_\Gamma(T)
    }
    $$

    We often write $S(T)$ without the subscript $\Gamma$ when the context is
    clear.
\end{definition}





\begin{theorem}
    \label{thm:SortSetUpperClosed}
    %-----------------------------
    \emph{A sort set is upper closed}.
    $$
        \ruleh{
            \Any_i \in S_\Gamma(T)
            \\
            i \le j
        }
        {
            \Any_j \in S_\Gamma(T)
        }
    $$
    \begin{proof}
        By the cumulativity rule of the typing relation.
    \end{proof}
\end{theorem}






\begin{theorem}
    \label{thm:SortSetSubstitution}
    %-----------------------------
    \emph{Type safe substitution makes the sort set bigger}.
    $$
        \ruleh{
            \Gamma \vdash a : A \quad \Gamma,x^A \vdash B
        }
        {
            S_{\Gamma,x^A}(B) \subseteq S_\Gamma(B[a/x])
        }
    $$
    \begin{proof}
        Assume $s \in S_{\Gamma,x^A}(B)$.

        Then by the definition of sort sets~\ref{def:SortSet}
        there exists $C$ with $B \reducestar C$ and $\Gamma,x^A \vdash C: s$.
        This implies $B[a/x] \reducestar C[a/x]$ and $\Gamma \vdash C[a/x] : s$
        by the subject reduction~\ref{SubjectReduction} and the substitution
        theorem~\ref{SubstitutionLemma}.

        The definition of sort sets~\ref{def:SortSet} implies $s \in
        S_\Gamma(B[a/x])$ which completes the proof.
    \end{proof}
\end{theorem}






\begin{theorem}
    \label{thm:SortSetProduct}
    %-----------------------------
    \emph{The sorts set of a product is either $\upper\Prop$ in case the product
    is a proposition or the intersection of the sort sets of the argument and
    the result type.}
    $$
    \ruleh{
        \Gamma \vdash \Pi x^A. B : \dontcare
    }
    {
        S(\Pi x^A. B) =
        \dcases{
            \upper\Prop & \text{if }\Gamma \vdash \Pi x^A . B : \Prop
            \\
            S(A) \cap S(B) & \text{otherwise}
        }
    }
    $$

    This theorem has the immediate consequence $S\Pi x^A. B \subseteq S(B)$. In
    the propositional case this is trivial. In the non-propositional case it is
    a consequence of the intersection.

    \begin{proof}
        Now we prove the main theorem.

        If the product is a proposition the proof is trivial because
        $\upper\Prop$ is the biggest possible sort set. For the
        non-propositional case we have to prove both directions:
        \begin{enumerate}
            \item $S(\Pi x^A. B) \subseteq S(A) \cap S(B)$:
                Assume $s$ is in the sort set of the product. By definition
                there exist reducts $A'$ and $B'$ of $A$ and $B$ such that
                $\Gamma \vdash \Pi x^{A'}. B' : s$. By the generation
                lemma~\ref{GenerationLemma} for products there exist a sort
                $s_0$ which is a valid type of $A'$ and $B'$ with $s_0 \le s$.
                Therefore $s_0$ is in the sort sets of $A$ and $B$. Because sort
                sets are upper closed $s$ is in both sort sets as well.

            \item $S(A) \cap S(B) \subseteq S(\Pi x^A. B)$:
                Assume $s$ is in both sort sets of $A$ and $B$. By definition
                there exist reducts $A'$ and $B'$ such that $s$ is a type of
                $A'$ and $B'$. The introduction rule for products in the typing
                relation says that $s$ is a type of the product $\Pi x^{A'}.
                B'$. By definition of the sort set $s$ is in the sort set of
                $\Pi x^A. B$.
        \end{enumerate}
    \end{proof}
\end{theorem}






\begin{theorem}
    \label{thm:SortSetsSubtype}
    %-----------------------------
    \emph{Let $A$ and $B$ be some welltyped terms in the same context. Then the
    following is valid:}
    \begin{enumerate}

        \item $A \betaeq B \imp S(A) = S(B)$

        \item $A <_i B \imp S(B) \subset S(A)$
    \end{enumerate}
    \begin{proof}

        \ \begin{enumerate}

            \item Because of confluence there exists a common reduct $C$ of $A$
                and $B$. Then by definition of the sort set $S(A) = S(C) =
                S(B)$.

            \item By induction on $ A <_i B$.
                \begin{enumerate}
                    \item The base case is trivial.

                    \item In the induction step we have $\Pi x^A . B_1 <_{i+1}
                        \Pi x^A. B_2$ where $S(B_2) \subset S(B_1)$ is given by
                        the induction hypothesis.

                        In case that $\Pi x^A. B_1$ is a proposition the goal is
                        trivial because $\upper\Prop$ is the largest possible
                        sort set.

                        It remains the case that both are not propositions ($\Pi
                        x^A. B_2$ cannot be a proposition because it is a proper
                        supertype).

                        Then we have
                        $$
                        S(\Pi x^A. B_2) = S(A) \cap S(B_2) \subset S(A) \cap
                        S(B_2) = S(\Pi x^A. B_1)
                        $$
                        by using the induction hypothesis
                        and~\ref{thm:SortSetProduct}.
                \end{enumerate}
        \end{enumerate}
    \end{proof}
\end{theorem}






\begin{theorem}
    \label{thm:SortSetApplication}
    %-----------------------------
    \emph{Sort sets of applications} Let $fa$ be an application where $F$ is a
    minimal type of $f$ and $R$ is a minimal type of $fa$. Then the sort set of
    $F$ is a subset of the sort set of $R$.
    $$
    \rulev{
        \Gamma \vdashmin f : F
        \\
        \Gamma \vdashmin fa : R
    }
    {
        S(F) \subseteq S(R)
    }
    $$
    \begin{proof}

        By~\ref{thm:MinimalTypeFunctionTerm} there exist the types $A$ and $B$
        such that $\Pi x^A. B$ is a minimal type of $f$. Therefore $F$ and $\Pi
        x^A.B$ are beta equivalent and have the same sort sets.

        The generation lemma~\ref{GenerationLemma} for application guarantees
        the existence of $A'$ and $B'$ such that $\Pi x^{A'}.B'$ is a valid type
        of $f$ and $A'$ is a valid type of $a$ and $\sub a x {B'} \le R$.

        Since $\Pi x^A. B$ is minimal we have $A \betaeq A'$ and $B \le B'$
        by~\ref{thm:ProductSubtype}.

        The minimality of $R$, the antisymmetry of
        subtypes~\ref{thm:SubtypeAntisymmetric} and the fact that
        substitution respects subtypes~\ref{thm:SubstitutionRespectsSubtype} we
        have $\sub a x B \betaeq R$.

        In combination with \ref{thm:SortSetsSubtype} and
        \ref{thm:SortSetProduct} and \ref{thm:SortSetSubstitution} we get
        $$
        S(F) = S(\Pi x^A. B)
            \subseteq S(B)
            \subseteq S(\sub a x B)
            = S(R)
        $$
    \end{proof}
\end{theorem}






\begin{theorem}
    \label{thm:SortSetApplications}
    %-----------------------------
    \emph{Sort sets of generalized applications} This is a generalization of the
    previous theorem~\ref{thm:SortSetApplication}.

    Let $fb \vec c$ be an application where
    $F$ is a minimal type of $f$ and $R$ is a minimal type of $fb \vec c$ where
    $\vec b$ is a possibly empty sequence of arguments $b_1, b_2, \ldots, b_n$.
    Then the sort set of $F$ is a subset of the sort set of $R$.
    $$
    \rulev{
        \Gamma \vdashmin f : F
        \\
        \Gamma \vdashmin f a \vec b : R
    }
    {
        S(F) \subseteq S(R)
    }
    $$
    \begin{proof}
        We prove the goal by induction on the length $n$ of $\vec b$.
        \begin{enumerate}
            \item $n=0$: Immediate consequence of the previous
                theorem~\ref{thm:SortSetApplication}.

            \item Induction step from $n$ to $n+1$:
                Assume $S(F) \subseteq S(R)$ for $n = |\vec b|$. We have to
                prove the goal
                $$
                \rulev{
                    \Gamma \vdashmin f a \vec b b_{n+1} : R'
                }
                {
                    S(F) \subseteq S(R')
                }
                $$
                By using the the induction hypothesis and the previous
                theorem~\ref{thm:SortSetApplication} we can prove the goal
                $$
                    S(F) \subseteq S(R) \subseteq S(R')
                $$
        \end{enumerate}
    \end{proof}
\end{theorem}






\begin{definition}
    \label{def:RedexSortSet}
    %-----------------------------
    \emph{Sort set of a redex}

    The sort set of a redex $(\lambda x^A. e) a$ is the sort set of a
    minimal type of its function term $\lambda x^A. e$.

    This set is welldefined because a minimal type of a term is unique up to
    beta equivalence and due to~\ref{thm:SortSetsSubtype} all beta equivalent
    types have the same sort set.
\end{definition}




\begin{definition}
    \label{def:INormalTerm}
    %-----------------------------
    \emph{Welltyped $i$-normal terms}

    A welltyped term $t$ is $i$-normal if it contains no redexes whose sort set
    is a subset of $\upper {\Any_i}$.
\end{definition}






\begin{theorem}
    \label{thm:NormalBaseTerm}
    %-----------------------------
    \emph{Let $fa$ be a welltyped $i+1$-normal term whose sort set is $\upperany
    i$.  Then $fa$ is a base term.}

    \begin{proof}
        We proof the more general theorem using $f b \vec c$ as the welltyped
        $i+1$-normal term whose sort set is $\upperany i$ and prove that $f b
        \vec c$ is a base term. $\vec c$ is a possibly empty sequence $c_1, c_2,
        \ldots, c_n$ of arguments.

        The proof goes by induction on the structure of the function term $f$.
        \begin{enumerate}

            \item $f$ can neither be a sort nor a product. This would contradict
                the fact that $f b \vec c$ is welltyped.

            \item If $f$ is a variable then $f b \vec c$ is by definition a base
                term.

            \item $f$ is the application $ga$: In that case $gab\vec c$ is a
                welltyped $i+1$-normal type whose sort set is $\upperany i$ and
                by the induction hypothesis for $g$ we conclude that $g a b \vec
                c = f b \vec c$ is a base term.

            \item $f$ is the abstraction $\lambda x^A. e$:

                MISSING
        \end{enumerate}
    \end{proof}
\end{theorem}








\subsection{Overview}

The goal of this section is to prove that any welltyped term can be reduced to a
\emph{quasinormal form} where quasinormal means that the type contains only
redexes whose function term has a propositional type. If the welltyped term is a
type (i.e. a term whose type is a sort), then a quasinormal form of this type
has the structure
$$
    \Pi
    x_0^{A_0} x_1^{A_1} \ldots x_{n-1}^{A_{n-1}}
    .
    \left\{
        \begin{array}{l}
            \Any_i
            \\
            y \vec{a}
        \end{array}
    \right .
$$
with $0 \le n$. I.e. a type in quasinormal form is a function type with zero or
more arguments and the result type is either a sort or a base term. The argument
types $A_0$, $A_1$, $\ldots$ are in quasinormal form as well.
with $0 \le n$. I.e. a type in quasinormal form is a function type with zero or
more arguments and the result type is either a sort or a base term. The argument
types $A_0$, $A_1$, $\ldots$ are in quasinormal form as well.






\subsection{Levels}

To motivate the definition of levels let's consider the following reduction
$$
    (\lambda x^{\Any_i}. x) \Prop \reduce \Prop
$$
where the left hand side and the right hand side have the minimal sorts
$$
    \begin{array}{lll}
        (\lambda x^{\Any_i} . x) \Prop &:& \Any_i
        \\
        \Prop &:& \Any_0
    \end{array}
$$
I.e. the minimal sort can go down via beta reduction. However we want to
characterize a type by a sort/universe which is independent of reduction.




\noindent OLD MATERIAL
%---------------------------------------------------------------

\begin{definition}
    A \emph{level} $l$ is a set of sorts which is upper closed i.e.
    $$
    \rulev{s_1 \in l \\ s_1 \le s_2}{s_2 \in l}
    $$
\end{definition}

Because levels are upper closed, a level which is a superset of another level
must contain extra sorts which are lower than the other level.

We define the following levels
$$
\begin{array}{lll}
    l_\Prop &:=& \{\Prop, \Any_0, \Any_1, \ldots\}
    \\
    l_{\Any_0} &:=& \{ \Any_0, \Any_1, \Any_2 \ldots \}
    \\
    \ldots
    \\
    l_{\Any_i} &:=& \{ \Any_i, \Any_{i+1}, \Any_{i+2}, \ldots \}
\end{array}
$$
with
$$
    l_\Prop \supseteq \Any_0 \supseteq \Any_1 \ldots
$$


\begin{definition}
    The level $\Level T$ of a type $T$ is the set of all sorts/universes it
    lives in modulo reduction.

    The relation $\Level$ is defined inductively by the rule
    $$
    \rulev
    {
        \Gamma \vdash B: s_1
        \\
        \Gamma \vdash C: s
        \\
        B \reducestar C
    }
    {\Level B s}
    $$
    where $\Level T s$ means that $s$ in the level of the type $T$.
\end{definition}

The so defined level is really a level i.e. upperclosed because of the
cumulativity.


\begin{theorem}
    \emph{Typesafe substitution does not increase the level of a type}.
    $$
    \rulev
    {
        \Gamma \vdash a: A
        \\
        \Gamma, x^A, \Delta \vdash B: s
    }
    {\Level B \subseteq \Level B[x:=a]}
    $$


    \begin{proof}
    \begin{enumerate}
        \item We have to proof that all sorts $s$ with $\Level B s$ satisfy
        $\Level B[x:=a] s$.

        \item In order for $\Level B s$ to be valid there exists a type $C$ with $B
        \reducestar C$ and $\Gamma, x^A, \Delta \vdash B: s$.

        \item By the substition lemma we get $\Gamma, \Delta[x:=a] \vdash C[x:=a]: s$.

        \item Since reduction and substitution are compatible
            (Lemma~\ref{SubstituteReduction})
            we can conclude $B[x:=a]
        \reducestar C[x:=a]$  from $B \reducestar C$.

        \item Therefore by definition $\Level B[x:=a] s$ is valid.
    \end{enumerate}
    \end{proof}
\end{theorem}



\begin{definition}
    The level $\ell t$ of a term $t$ is the union of all the levels of its
    types. The term level is defined by the rule
    $$
    \rulev{
        \Gamma \vdash t: T
        \\
        \Level T s
    }
    {
        \ell t s
    }
    $$
\end{definition}

For term levels we use the same ordering as for type levels.


\begin{theorem}
    The term level $\ell f$ of the function term of an application $f a$ where
    $f a$ is a valid term in some context $\Gamma$ is a subset of the term level
    of the application
    $$
    \ell f \subseteq \ell (f a)
    $$

    \begin{proof}

        We have to prove that $\ell f s$ implies $\ell (f a) s$ for all sorts $s$.

        MISSING
    \end{proof}
\end{theorem}


\subsection{Quasinormalization}


\begin{theorem}
    \label{quasinormalization}
    \emph{Quasinormalization theorem}

    MISSING!!
\end{theorem}
