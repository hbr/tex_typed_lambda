%-------------------------------------------------------------------------------
\section{Quasinormalization}
%-------------------------------------------------------------------------------


\subsection{Overview}
%-------------------------------------------------------------------------------


In this section we prove that all welltyped terms can be reduced to an $1$
normal form. A term is in $1$ normal form if it has no redexes of level
$1$ or higher.

The level of a type is the level of the lowest sort the type lives in up to
reduction. The level of a redex is the level of the type of its function term.

Here we give an overview of the proof. The needed definitions and theorems will
be made precise in the subsequent sections.

A term $t$ is welltyped if there is a context $\Gamma$ and a type term $T$ such
that $\Gamma \vdash t : T$ is satisfied. Without restriction of generality we
can assume that $\Gamma \vdash t : T$ has been derived with a finite set of
sorts $S_n$. To find $n$ just look into the derivation of $\Gamma \vdash t : T$
and find the highest sort $\Any_n$. Then it is guaranteed that the derivation is
possible with the finite sort set $S_n$.

If we restrive ourselves to the finite sort set $S_n$ then the highest level of
all types is $n$ and the highest level of any redex is $n$ as well.
I.e. all derivable terms are in $n+1$ normal form.

We want to prove that we can transform any welltyped term $t$ derived with the
finite sort set $S_n$ into $1$ normal form. We do this by downward
induction from $n+1$ to $1$.

The induction start is trivial, because all terms derived with the finite sort
set $S_n$ are in $n+1$ normal form.

For the induction step from $i+1$ down to $i$ we can assume the induction
hypothesis that we can reduce any welltyped term to $i+1$ normal form.




\begin{comment}

    Basic proof idea
    ============================================================

    Levels of a redex (via level of the type of the function term)
    ------------------------------------------------------------

    Each term which is not in normal form has a nonempty set of redexes. Each of
    these redexes has a function term and this function term has a minimal type.

    Each type has a level which is its lowest sort modulo reduction. Therefore
    we can say that each redex has a level (the level of the minimal type of its
    function term). The number of redexes in a term is finite. There is a
    highest level of all redexes a term.

    We say that a term is s-normal, if it has no redexes of level s or higher.
    If s is the highest level of a redex in a term, then the term is s+1 normal.


    Degree of a redex
    ------------------------------------------------------------

    For all terms which are s+1 normal it is possible the define a degree for
    all s level redexes.

    We can reduce each term where s is the highest level of all its redexes into
    s-normal form. We do this by repeatedly reducing the rightmost redex of
    level s.



\end{comment}



\begin{comment}
    Level:

        A set of sorts a type can live in, even if beta reduced.

        Note:
            Beta reduction can reduce the sort. E.g.

                (\ (A: Any (i+k+1)) := A) (Any i) ~> Any i

            where `Any i` has type  `Any (i+1)` and
            `(\ (A: Any (i+k+1)) := A) (Any i)` has type `Any (i+k+1)` because
            the result of the function term is `A` which has type `Any (i+k+1)`.


    Level of a redex:

        All redexes have a function term which has a minimal type of the form
        'all (x: A): B' (or beta equivalents). The level of this type is the
        level of a redex.


    j-normal term:

        Does not have redexes of level 'upper (Any j)' or subsets of it.

        0-normal terms are normal terms, because they contain no redexes.


    Proof idea:

        In ECCn a redex with sort set upper (Any n) is not possible. Therefore
        we have a function rn which reduces any term to its n-normal form. Since
        all terms are in n-normal form rn is the identity function.

        Induction step: Assume there is a reduction function ri+1, find a
        reduction function ri.

            We look into redexes whose sort set of its function term type is
            upper (Any i). We can transform all these types into i+1 normal form
            via ri+1. The normalized type is a sort or a product or a base term.
            We can compute the degree of all these types.

            We use the triple

                (i, d, n)

            where d is the maximal degree of all these types and n is the number
            of redexes which have this maximal degree.

            We reduce the rightmost redex of degree d and claim that this
            reduces n by one. We continue until n=0 i.e. there are no more
            redexes of degree d.

            We continue this process until there are no more redexes of level i.

            The i-normal term is unique because the rightmost redex is unique.
            Therefore we have found the reduction function ri.

        Crucial in the construction: The reduction of the rightmost redex of the
        highest degree does not make new redexes of this degree.
\end{comment}







\subsection{Level}
%------------------------------------------------------------


\begin{definition}
    The \emph{Level} $\Level(T)$ of a type T is the level of the minimal sort
    the type lives in up to reduction.
\end{definition}


\begin{theorem}
    \label{thm:LevelTypeSafeSubstitution}
    %------------------------------------------------------------
    Type safe substitution decreases the level of a type.
    $$
    \rulev{
        \Gamma \vdash a : A
        \\
        \Gamma, x^A \vdash B: s
    }
    {
        \Level(B[a/x]) \le \Level(B)
    }
    $$
    \begin{proof}
        Let's say the level of $B$ is $i$. Then there exists by definition of
        the level a reduct $C$ of $B$ such that $\Gamma, x^A \vdash C: \Any_i$
        is valid. By the substitution lemma~\ref{SubstitutionLemma} we get
        $\Gamma \vdash C[a/x]: \Any_i$. Since $C[a/x]$ lives in the sort
        $\Any_i$ and is a reduct of $B[a/x]$ the level of $B[a/x]$ is less or
        equal $i$.
    \end{proof}
\end{theorem}




\begin{theorem}
    \label{thm:LevelProduct}
    %------------------------------------------------------------
    The level of a nonpropositional product is the maximal level of the argument
    type and the result type.
    $$
    \ruleh{
        \Level(B) > 0
    }
    {
        \Level(\Pi x^A.B) = \max(\Level(A), \Level(B))
    }
    $$
    \begin{proof}
        Because of cumulativity we have
        $$
        \rulev{
            \Gamma \vdash A: \Any_i
            \\
            \Gamma,x^A \vdash B: \Any_j
            \\
            i,j > 0
        }
        {
            \Gamma \vdash \Pi x^A. B : \Any_{\max(i,j)}
        }
        $$
        Since reduction of a product does not change its shape, the levels of
        the argument and result type determine uniquely the level of the
        product.
    \end{proof}
\end{theorem}




\begin{theorem}
    \label{thm:LevelSubtypeEquivalence}
    %------------------------------------------------------------
    The level of a type respects subtyping and betaequivalence.
    \begin{enumerate}
        \item $ A \betaeq B \imp \Level(A) = \Level(B)$

        \item $ A <_i B \imp \Level(A) \le \Level(B) $
    \end{enumerate}

    \begin{proof}
        \ \begin{enumerate}
            \item In case of betaequivalence $A$ and $B$ have a common reduct
                $C$ with $\Level(A) = \Level(C) = \Level(B)$.

            \item By induction on $<_i$.
                \begin{enumerate}
                    \item The base case is trivial.

                    \item For the induction step we assume $B_1 <_i B_2 \imp
                        \Level(B_1) \le \Level(B_2)$ and have to prove
                        $$
                            \Level(\Pi x^A.B_1) \le \Level(\Pi x^A. B_2)
                        $$
                        under the assumption $ \Pi x^A. B_1 <_{i+1} \Pi x^A.
                        B_2$ (which implies $\Level(B_1) \le \Level(B_2)$ by the
                        induction hypothesis.

                        If $B_1$ is a proposition i.e. $\Level(B_1) = 0$, then
                        the goal is trivial.

                        So we assume $B_1$ is not a
                        proposition which implies that $B_2$ is not a
                        proposition either. With the help of
                        theorem~\ref{thm:LevelProduct} we derive
                        $$
                        \begin{array}{lll}
                            \Level(\Pi x^A. B_1)
                            & = & \max(\Level(A), \Level(B_1)
                            \\
                            & \le & \max(\Level(A), \Level(B_2))
                            \\
                            & = & \Level(\Pi x^A. B_2)
                        \end{array}
                        $$
                \end{enumerate}
        \end{enumerate}
    \end{proof}
\end{theorem}



\begin{theorem}
    \label{thm:LevelApplication}
    %------------------------------------------------------------
    Let $f a$ be an application where $F$ is a minimal type of $f$ and $R$ is a
    minimal type of $f a$. Then the level of $F$ is greater equal the level of
    $R$.
    $$
    \rulev{
        \Gamma \vdashmin f : F
        \\
        \Gamma \vdashmin f a: R
    }
    {
        \Level(F) \ge \Level(R)
    }
    $$
    \begin{proof}
        Without loss of generality we assume that $F$ is not a proposition,
        because for propositions the goal is trivial.

        Since $f$ is a function positition according
        to~\ref{thm:MinimalTypeFunctionTerm} there is a minimal type of the form
        $\Pi x^A.B$ betaequivalent to $F$.
        $$
        \begin{array}{lllll}
            \Level(F)
            & = & \Level(\Pi x^A.B) &~\ref{thm:LevelSubtypeEquivalence}
            \\
            & \ge  & \Level(B)
            &~\ref{thm:LevelProduct}
            \\
            & \ge & \Level(B[a/x]
            &~\ref{thm:LevelTypeSafeSubstitution}
            \\
            & = & \Level(R)
        \end{array}
        $$
    \end{proof}
\end{theorem}


\begin{theorem}
    \label{thm:LevelRepeatedApplication}
    %------------------------------------------------------------
    Let $f a_1 \ldots a_n$ is an application where $F$ is a minimal type of $f$
    and $R_i$ is a minimal type of $f a_1 \ldots a_i$ with $i \le n$. Then the
    level of $F$ is greater equal to the level of $R_n$.

    \begin{proof}
        We prove this fact by repeated application of the previous
        theorem~\ref{thm:LevelApplication}.
        $$
        \begin{array}{lll}
            \Level(F) & \ge & \Level(R_1)
            \\
            & \ge & \Level(R_2)
            \\
            \ldots
            \\
            & \ge & \Level(R_n)
        \end{array}
        $$
    \end{proof}
\end{theorem}








\subsection{$i$-Normal Terms}
%------------------------------------------------------------

With the help of the level of a type we define the level of a redex by
\begin{definition}
    \emph{Level of a redex}
    %------------------------------------------------------------

    The level of a redex $(\lambda x^A. e) a$ is the level of the minimal type of
    the function term $\lambda x^A. e$.
\end{definition}

The level of a redex is welldefined because all minimal types are betaequivalent
and due to~\ref{thm:LevelSubtypeEquivalence} all betaequivalent types have the
same level.



\begin{definition}
    \emph{Welltyped $i$-normal terms}

    A welltyped term $t$ is $i$-normal if it contains no redexes with a level of
    $i$ or higher.
\end{definition}




\begin{theorem}
    \label{thm:NormalBaseTerm}
    %-----------------------------
    \emph{Let $fa$ be a welltyped $i+1$-normal type whose level is $i$ or lower
    i.e. $\Level(f a) \le i$.
    In that case $f a$ is a base term.}

    \begin{proof}
        We prove the more general theorem using $f a \vec b$ as the welltyped
        $i+1$-normal term with $\Level(f a \vec b) \le i$ where $\vec b$ is a
        possibly empty sequence of additional arguments.
        We prove that $f a \vec b$ is a base term.

        Since $f a \vec b$ is a type, its type has to be a sort i.e. we have
        $\Gamma \vdashmin f a \vec b: \Any_j$ for some $\Gamma$ and $j$. Since the
        level of the type is $i$ or lower, we have $i \le j$.

        The proof goes by induction on the structure of the function term $f$.
        \begin{enumerate}

            \item $f$ can neither be a sort nor a product. This would contradict
                the fact that $f b \vec c$ is welltyped.

            \item If $f$ is a variable then $f a \vec b$ is by definition a base
                term.

            \item $f$ is the application $ga_0$: In that case $ga_0 a \vec b$ is
                a welltyped $i+1$-normal type of level $i$ or lower.
                From the induction hypothesis for $g$ we conclude that $g a_0
                a \vec b = f a \vec b$ is a base term.

            \item $f$ is an abstraction:
                Say $F$ is a minimal type of $f$.
                By~\ref{thm:LevelRepeatedApplication} we have $\Level(F) \ge
                \Level(\Any_j) = j + 1 > i$. This contradicts the assumption that
                $f a \vec b$ is $i+1$ normal i.e. does not have redexes of level
                $i + 1$ or higher. Therefore $f$ cannot be an abstraction.
        \end{enumerate}
    \end{proof}
\end{theorem}


\begin{theorem}
    \label{thm:FormNormalTypes}
    %------------------------------------------------------------
    \emph{Form of Normal Types} A welltyped $i+1$-normal type $T$ whose level is
    $i$ or lower is either a sort, a product or a base term.


    \begin{proof}
        By induction of the structure of $T$.
        \begin{itemize}

            \item If $T$ is a sort or a product or a variable then the goal is
                trivial.

            \item $T$ cannot be an abstraction because the type of an
                abstraction cannot be a sort i.e. it cannot be a welltyped type.

            \item If $T$ is an application it is a base term. This is an
                immediate consequence of~\ref{thm:NormalBaseTerm}.
        \end{itemize}
    \end{proof}
\end{theorem}



\subsection{Degree}
%------------------------------------------------------------

\begin{definition}
    \label{def:TypeDegree}
    \emph{The degree of a welltyped $i+1$-normal type is defined recursively by}

    $$
    D_i(T) :=
    \dcases{
        D_i(T) &:=& 0 \quad\text{if } \Level(T) \ne i
        \\
        D_i(s) &:=& 0
        \\
        D_i(x \vec a) &:=& 0
        \\
        D_i(\Pi x^A.B) &:=&
        1 + D_i(A) +  D_i(B)
    }
    $$

    Note that the degree is welldefined by~\ref{thm:FormNormalTypes} which
    asserts that a level $i$ type $T$ is either a sort or a base term or a
    product if it is $i+1$-normal.
\end{definition}



\begin{theorem}
    Welltyped betaequivalent types which are $i+1$-normal have the same degree.
    \label{thm:BetaEquivalentSameDegree}
    %----------------------------------------------------------------------
    \begin{proof}
        Assume that $T_1$ and $T_2$ are betaequivalent and $i+1$-normal.
        By~\ref{thm:LevelSubtypeEquivalence} they have the same level. If the
        level is not $i$, then the goal is trivial. Therefore we assume that the
        level is $i$.

        We prove $\Level(T_1) = \Level(T_2)$ by induction on the structure of
        $T_1$.

        By~\ref{thm:FormNormalTypes} we known that both are either sorts, base
        terms or products. Welltyped sorts, base terms and products can only be
        betaequivalent if they are both sorts, base terms or products respectively.


        For sorts and base terms the goal is evident by definition that
        they have the same degree (namely 0). If both are products the goal
        follows from the induction hypothesis.
    \end{proof}
\end{theorem}





\begin{definition}
    \label{def:RedexDegree}
    The \emph{degree of a redex} in an $i+1$-normal term is defined as the
    degree of a minimal type of the function term.

    The degree of a redex is welldefined, because betaequivalent types have the
    same degree by~\ref{thm:BetaEquivalentSameDegree}.
\end{definition}







\subsection{Normalization}
%------------------------------------------------------------


\begin{definition}
    \label{def:NormalizationStep}
    \emph{Normalization step of an $s+1$-normal term}
    Let $r$ be a function over welltyped terms $t$ such that $r(t)$ is
    $s+1$-normal. Assume that the term $a$ is $s+1$ normal and has redexes with
    sort set $\upper s$. Each of these redexes has a degree according to
    the definition~\ref{def:RedexDegree}.

    We define $a \torel {q_s} b$ as the reduction relation
    which reduces the rightmost redex of sort set $\upper s$.

    This definition has the following important consequences:
    \begin{itemize}
        \item
            If the highest degree $d$ of the redexes in the term $t$ is zero,
            then the term is $s$-normal.

            Reason: There is always a minimal type in $s+1$-normal form of the
            function term of the redex of the form $\Pi x^A. B$. By definition
            of the degree of a type~\ref{def:TypeDegree} its degree is at least
            $1$ as long as its sort set is a subset of $\upper s$. If the
            degree is zero, then its sort set is a proper superset of $\upper
            s$. Therefore $t$ has no redexes whose sort set is a subset of
            $\upper s$ and thus by definition is $s$-normal.

        \item
            If $a \torel {q_s} b$ is a normalization step of the $s+1$ normal
            term $a$, then the term $b$ is uniquely defined because the
            rightmost redex of degree with sort set $\upper s$ is unique.

        \item The reduction $a \torel {q_s} b$ does not copy any existing
            redexes with sort set $\upper s$, because the contracted redex is
            the rightmost redex.
    \end{itemize}
\end{definition}


\begin{theorem}
    \label{thm:NewRedexCreation}
    \emph{Creation of new redexes during beta reduction} There are only three
    possibilities to create new redexes during a beta reduction which does the
    reduction $(\lambda x^A.e) a \reduce \sub a x e$ on any subterm of a term.

    $$
    \begin{array}{lll}
        (\lambda x^A. \ldots xb \ldots) (\lambda y^B. e)
        &\reduce&
        \ldots (\lambda y^B.e) b \ldots
        %
        \\
        %
        (\lambda x^A. x)(\lambda y^B. e) b
        &\reduce&
        (\lambda y^B. e) b
        %
        \\
        %
        (\lambda x^A . \lambda y^B. e) a b
        &\reduce&
        (\lambda y^{B[a/x]}. e[a/x]) b
    \end{array}
    $$
    where $xb$ in the first case is not the argument of an application i.e. $xb$
    is in a head position.

    \begin{proof}
        A redex consists of an abstraction applied to an argument. Both must
        have been in the original expression and come together after the
        reduction. If the abstraction is in the argument of the redex then the
        argument must be the abstraction. The abstraction replaces the bound
        variable of the original redex.

        If the original redex has $x b$ in a head position, then the
        substitution of the abstraction for $x$ creates a new redex. This is the
        first case.

        If the original redex has only the bound variable $x$ but not applied to
        an argument, then the body of the original redex must be $x$ and the
        argument comes from an argument in the original expression which follows
        the original redex. This is the second case.

        The third possibility has an abstraction without argument (otherwise it
        were already a redex) as the body of the original abstraction and the
        argument of the newly created redex is the argument following the
        argument of the original redex.
    \end{proof}
\end{theorem}



\begin{theorem}
    \emph{A sequence of normalization steps
        according to
        definition~\ref{def:NormalizationStep}
        of an $s+1$-normal term $t$
        when $s$
        is not a propositional sort
        ends after a finite number of steps
        in an $s$-normal unique term.}

    \begin{proof}

        Let
        $$
            n_d n_{d-1} \ldots n_2 n_1
        $$
        be a sequence of numbers where $d$ is the maximal degree of all redexes
        in $t$ with sort set $\upper s$ where $s$ is not a propositional sort.
        The number $n_i$ is the number of redexes in the term $t$ of degree $i$
        and with sort set $\upper s$.

        Then in the reduction sequence
        $$
            t = t_0 \torel {q_s} t_1 \torel{q_s} t_2 \torel{q_s} \ldots
        $$
        each reduction step decreases the number sequence in the lexicographic
        order. Since the lexicographic order is a wellorder, the reduction
        sequence ends after a finite number of steps.
    \end{proof}
\end{theorem}
