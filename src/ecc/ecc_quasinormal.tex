%-------------------------------------------------------------------------------
\section{Quasinormalization}
%-------------------------------------------------------------------------------


In this section we prove that all welltyped terms can be reduced to a
quasinormal form where a term in quasinormal form has only propositional
redexes.

Outline of the proof:
\begin{itemize}
    \item First we introduce the level of a type as the smallest sort it lives
        in up to reduction. As opposed to the minimal type the level is
        invariant for betaequivalent types. A propositional type has level 0.
        All predicative types i.e. non-propositional types have a level greater
        than 0.

    \item We use the level of a type to assign a level to a redex which is the
        level of the minimal type of its function term (i.e. its abstraction).
        This allows us to define $i$-normality of a term which means that the
        term does not have redexes of level $i$ or higher.

    \item For $i+1$ normal types we can define an $i$-degree which is some kind
        of arity of the type.

    \item With the help of $i$-degrees we can define for $i+1$ normal terms an
        $i$-normalization step which decreases in each step the complexity
        measure to end finally in a term which is $i$-normal. Therefore we can
        reduce each $i+1$-normal term into an $i$-normal term.

    \item For all welltyped terms derivable with in infinite sort set $S_\infty$
        there exists a finite sort set $S_n$ such that the welltyped term can be
        derived with the finite sort set. All welltyped terms derived with a
        finite sort set $S_n$ are trivially $n+1$-normal because no types of
        level $n+1$ or higher can be derived with the finite sort set. Therefore
        all welltyped terms (which are $n+1$-normal for some $n$) can be
        stepwise reduced to 1-normal terms which are quasinormal and contain
        only propositional redexes.
\end{itemize}






\subsection{Level}
%------------------------------------------------------------


\begin{definition}
    The \emph{Level} $\Level(T)$ of a type T is the level of the minimal sort
    the type lives in up to reduction.
\end{definition}


\begin{theorem}
    \label{thm:LevelTypeSafeSubstitution}
    %------------------------------------------------------------
    Type safe substitution does not increase the level of a type.
    $$
    \rulev{
        \Gamma \vdash a : A
        \\
        \Gamma, x^A \vdash B: s
    }
    {
        \Level(B[a/x]) \le \Level(B)
    }
    $$
    \begin{proof}
        Let's say the level of $B$ is $i$. Then there exists by definition of
        the level a reduct $C$ of $B$ such that $\Gamma, x^A \vdash C: \Any_i$
        is valid. By the substitution lemma~\ref{SubstitutionLemma} we get
        $\Gamma \vdash C[a/x]: \Any_i$. Since $C[a/x]$ lives in the sort
        $\Any_i$ and is a reduct of $B[a/x]$ the level of $B[a/x]$ is less or
        equal $i$.
    \end{proof}
\end{theorem}




\begin{theorem}
    \label{thm:LevelProduct}
    %------------------------------------------------------------
    The level of a nonpropositional product is the maximal level of the argument
    type and the result type.
    $$
    \ruleh{
        \Level(B) > 0
    }
    {
        \Level(\Pi x^A.B) = \max(\Level(A), \Level(B))
    }
    $$
    \begin{proof}
        Because of cumulativity we have
        $$
        \rulev{
            \Gamma \vdash A: \Any_i
            \\
            \Gamma,x^A \vdash B: \Any_j
            \\
            i,j > 0
        }
        {
            \Gamma \vdash \Pi x^A. B : \Any_{\max(i,j)}
        }
        $$
        Since reduction of a product does not change its shape, the levels of
        the argument and result type determine uniquely the level of the
        product.
    \end{proof}

    Note the restriction to nonpropositional products. The level of a
    propositional product is 0 regardless of the level of the argument type.
\end{theorem}




\begin{theorem}
    \label{thm:LevelSubtypeEquivalence}
    %------------------------------------------------------------
    The level of a type respects subtyping and betaequivalence.
    \begin{enumerate}
        \item $ A \betaeq B \imp \Level(A) = \Level(B)$

        \item $ A <_i B \imp \Level(A) \le \Level(B) $
    \end{enumerate}

    \begin{proof}
        \ \begin{enumerate}
            \item In case of betaequivalence $A$ and $B$ have a common reduct
                $C$ with $\Level(A) = \Level(C) = \Level(B)$.

            \item By induction on $<_i$.
                \begin{enumerate}
                    \item The base case is trivial.

                    \item For the induction step we assume $B_1 <_i B_2 \imp
                        \Level(B_1) \le \Level(B_2)$ and have to prove
                        $$
                            \Level(\Pi x^A.B_1) \le \Level(\Pi x^A. B_2)
                        $$
                        under the assumption $ \Pi x^A. B_1 <_{i+1} \Pi x^A.
                        B_2$ (which implies $\Level(B_1) \le \Level(B_2)$ by the
                        induction hypothesis).

                        If $B_1$ is a proposition i.e. $\Level(B_1) = 0$, then
                        the goal is trivial.

                        So we assume $B_1$ is not a
                        proposition which implies that $B_2$ is not a
                        proposition either. With the help of
                        theorem~\ref{thm:LevelProduct} we derive
                        $$
                        \begin{array}{lll}
                            \Level(\Pi x^A. B_1)
                            & = & \max(\Level(A), \Level(B_1)
                            \\
                            & \le & \max(\Level(A), \Level(B_2))
                            \\
                            & = & \Level(\Pi x^A. B_2)
                        \end{array}
                        $$
                \end{enumerate}
        \end{enumerate}
    \end{proof}
\end{theorem}



\begin{theorem}
    \label{thm:LevelApplication}
    %------------------------------------------------------------
    Let $f a$ be an application where $F$ is a minimal type of $f$ and $R$ is a
    minimal type of $f a$. Then the level of $F$ is greater equal the level of
    $R$.
    $$
    \rulev{
        \Gamma \vdashmin f : F
        \\
        \Gamma \vdashmin f a: R
    }
    {
        \Level(F) \ge \Level(R)
    }
    $$
    \begin{proof}
        Without loss of generality we assume that $F$ is not a proposition,
        because for propositions the goal is trivial.

        Since $f$ is a function positition according
        to~\ref{thm:MinimalTypeFunctionTerm} there is a minimal type of the form
        $\Pi x^A.B$ betaequivalent to $F$.
        $$
        \begin{array}{lllll}
            \Level(F)
            & = & \Level(\Pi x^A.B) &~\ref{thm:LevelSubtypeEquivalence}
            \\
            & \ge  & \Level(B)
            &~\ref{thm:LevelProduct}
            \\
            & \ge & \Level(B[a/x]
            &~\ref{thm:LevelTypeSafeSubstitution}
            \\
            & = & \Level(R)
        \end{array}
        $$
    \end{proof}
\end{theorem}


\begin{theorem}
    \label{thm:LevelRepeatedApplication}
    %------------------------------------------------------------
    Let $f a_1 \ldots a_n$ is an application where $F$ is a minimal type of $f$
    and $R_i$ is a minimal type of $f a_1 \ldots a_i$ with $i \le n$. Then the
    level of $F$ is greater equal to the level of $R_n$.

    \begin{proof}
        We prove this fact by repeated application of the previous
        theorem~\ref{thm:LevelApplication}.
        $$
        \begin{array}{lll}
            \Level(F) & \ge & \Level(R_1)
            \\
            & \ge & \Level(R_2)
            \\
            \ldots
            \\
            & \ge & \Level(R_n)
        \end{array}
        $$
    \end{proof}
\end{theorem}








\subsection{$i$-Normal Terms}
%------------------------------------------------------------

With the help of the level of a type we define the level of a redex by
\begin{definition}
    \emph{Level of a redex}
    %------------------------------------------------------------

    The level of a redex $(\lambda x^A. e) a$ is the level of the minimal type of
    the function term $\lambda x^A. e$.
\end{definition}

The level of a redex is welldefined because all minimal types are betaequivalent
and due to~\ref{thm:LevelSubtypeEquivalence} all betaequivalent types have the
same level.



\begin{definition}
    \emph{Welltyped $i$-normal terms}

    A welltyped term $t$ is $i$-normal if it contains no redexes with a level of
    $i$ or higher.
\end{definition}




\begin{theorem}
    \label{thm:NormalBaseTerm}
    %-----------------------------
    \emph{Let $fa$ be a welltyped $i+1$-normal type whose level is $i$ or lower
    i.e. $\Level(f a) \le i$.
    In that case $f a$ is a base term.}

    \begin{proof}
        We prove the more general theorem using $f a \vec b$ as the welltyped
        $i+1$-normal term with $\Level(f a \vec b) \le i$ where $\vec b$ is a
        possibly empty sequence of additional arguments.
        We prove that $f a \vec b$ is a base term.

        Since $f a \vec b$ is a type, its type has to be a sort i.e. we have
        $\Gamma \vdashmin f a \vec b: \Any_j$ for some $\Gamma$ and $j$. Since the
        level of the type is $i$ or lower, we have $i \le j$.

        The proof goes by induction on the structure of the function term $f$.
        \begin{enumerate}

            \item $f$ can neither be a sort nor a product. This would contradict
                the fact that $f b \vec c$ is welltyped.

            \item If $f$ is a variable then $f a \vec b$ is by definition a base
                term.

            \item $f$ is the application $ga_0$: In that case $ga_0 a \vec b$ is
                a welltyped $i+1$-normal type of level $i$ or lower.
                From the induction hypothesis for $g$ we conclude that $g a_0
                a \vec b = f a \vec b$ is a base term.

            \item $f$ is an abstraction:
                Say $F$ is a minimal type of $f$.
                By~\ref{thm:LevelRepeatedApplication} we have $\Level(F) \ge
                \Level(\Any_j) = j + 1 > i$. This contradicts the assumption that
                $f a \vec b$ is $i+1$ normal i.e. does not have redexes of level
                $i + 1$ or higher. Therefore $f$ cannot be an abstraction.
        \end{enumerate}
    \end{proof}
\end{theorem}


\begin{theorem}
    \label{thm:FormNormalTypes}
    %------------------------------------------------------------
    \emph{Form of Normal Types} A welltyped $i+1$-normal type $T$ whose level is
    $i$ or lower is either a sort, a product or a base term.


    \begin{proof}
        By induction of the structure of $T$.
        \begin{itemize}

            \item If $T$ is a sort or a product or a variable then the goal is
                trivial.

            \item $T$ cannot be an abstraction because the type of an
                abstraction cannot be a sort i.e. it cannot be a welltyped type.

            \item If $T$ is an application it is a base term. This is an
                immediate consequence of~\ref{thm:NormalBaseTerm}.
        \end{itemize}
    \end{proof}
\end{theorem}








\subsection{Degree}
%------------------------------------------------------------

\begin{definition}
    \label{def:TypeDegree}
    \emph{The $i$-degree of a welltyped $i+1$-normal type is defined recursively
    by}

    $$
    \Degree_i(T) :=
    \dcases{
        \Degree_i(T) &:=& 0 \quad\text{if } \Level(T) \ne i
        \\
        \Degree_i(s) &:=& 0
        \\
        \Degree_i(x \vec a) &:=& 0
        \\
        \Degree_i(\Pi x^A.B) &:=&
        1 + \Degree_i(A) +  \Degree_i(B)
    }
    $$

    Note that the degree is welldefined by~\ref{thm:FormNormalTypes} which
    asserts that a level $i$ type $T$ is either a sort or a base term or a
    product if it is $i+1$-normal.
\end{definition}



\begin{theorem}
    Welltyped betaequivalent types which are $i+1$-normal have the same degree.
    \label{thm:BetaEquivalentSameDegree}
    %----------------------------------------------------------------------
    \begin{proof}
        Assume that $T_1$ and $T_2$ are betaequivalent and $i+1$-normal.
        By~\ref{thm:LevelSubtypeEquivalence} they have the same level. If the
        level is not $i$, then the goal is trivial. Therefore we assume that the
        level is $i$.

        We prove $\Level(T_1) = \Level(T_2)$ by induction on the structure of
        $T_1$.

        By~\ref{thm:FormNormalTypes} we know that both are either sorts, base
        terms or products. Welltyped sorts, base terms and products can only be
        betaequivalent if they are both sorts, base terms or products respectively.


        For sorts and base terms the goal is evident by definition that
        they have the same degree (namely 0). If both are products the goal
        follows from the induction hypothesis.
    \end{proof}
\end{theorem}



\begin{theorem}
    Subtyping of welltyped $i+1$-normal types of level $i$ respects the degree.
    \label{thm:SubtypeRespectsDegree}
    $$
    \rulev{
        T \le U
        \\
        \Level(T) = \Level(U) = i
    }
    {
        \Degree_i(T) \le \Degree_i(U)
    }
    $$
    \begin{proof}
        $T$ and $U$ are $i+1$-normal and of level $i$. Therefore both are either
        sorts, baseterms or products by~\ref{thm:FormNormalTypes}.

        In case $T$ and $U$ are betaequivalent, the goal follows immediately
        from \ref{thm:BetaEquivalentSameDegree}.

        In the other case there exist some $k$, $V_1$ and $V_2$ such that $V_1
        <_k V_2$, $T \reducestar V_1$ and $U \reducestar V_2$ are valid.

        We prove the remaining case by induction on the structure of $T$ where
        we have to consider only the cases that $T$ is either a sort, a base
        term or a product.

        \begin{enumerate}
            \item $T$ is a sort (say $s_T$): This is possible only if $k=0$. This
                implies that $U$ is a sort (say $s_U$) as well with $s_T <_0
                s_U$ which implies the goal. 

            \item $T$ is a base term: This case is impossible, because reduction
                does not change the structure of a base term and $V_1$
                cannot be a base term.

            \item $T$ is a product (say T = $\Pi x^{A_1}. B_1$): This implies that
                $V_1$ and $V_2$ are products as well and $k > 0$. Therefore $U$ has
                to by a product as well, say $U = \Pi x^{A_2}. B_2$.

                Furthermore from $V_1 <_k V_2$ we infer than $A_1 \betaeq A_2$
                and $B_1 \le B_2$. From~\ref{thm:BetaEquivalentSameDegree} and
                the induction hypothesis we infer $\Degree_i(A_1) =
                \Degree_i(A_2)$ and $\Degree_i(B_1) \le \Degree_i(B_2)$ which
                implies by the definition of the degree $\Degree_i(T) \le
                \Degree_i(U)$.
        \end{enumerate}
    \end{proof}
\end{theorem}



\begin{theorem}
    Typesafe substitution of an $i+1$ normal type of level $i$ does not increase
    the $i$-degree of the type as long as the level of the minimal type of the
    replacement term is $i$ or lower.
    \label{thm:DegreeTypesafeSubstitution}
    $$
    \rulev{
        \Gamma \vdashmin a : A
        \\
        \Gamma,x^A \vdashmin T: \Any_k
        \\
        \Level(T) = i   \quad \text{$T$ is $i+1$-normal}
        \\
        \text{$T[a/x]$ has a betaequivalent $i+1$-normal form}
    }
    {
        \Degree_i(T[a/x]) \le \Degree_i(T)
    }
    $$
    \begin{proof}
        By induction on the structure of $T$. Because of~\ref{thm:FormNormalTypes}
        $T$ is either a
        sort, a base term or a product.
        \begin{enumerate}
            \item $T$ is a sort: Trivial

            \item $T$ is a base term (say $y \vec b$):
                \begin{itemize}
                    \item $x \ne y$: Then $T[a/x]$ and all its betaequivalent
                        $i+1$-normal forms are base terms with degree 0.

                    \item $x = y$: In this case we have $T[a/x] = a \vec b :
                        \Any_k$. We can derive
                        $$
                        \begin{array}{llll}
                            i &=& \Level(A)
                            \\
                            &\ge& \Level(\Any_k)
                            &\text{\ref{thm:LevelRepeatedApplication}}
                            \\
                            &=& k + 1
                            \\
                            &>& i
                        \end{array}
                        $$
                        which states the contradiction $i > i$. Therefore this
                        case is not possible.
                \end{itemize}

            \item $T$ is a product (say $\Pi y^B. C$): The goal is a consequence
                of the induction hypotheses for $B$ and $C$.
        \end{enumerate}
    \end{proof}
\end{theorem}



\subsection{Normalization}
%------------------------------------------------------------

In this section we prove that all welltyped terms have a 1-normal form. Without
loss of generality we restrict ourselves to welltyped terms derivable with a
finite sort set $S_n$.

We prove the more general statement that all welltyped terms derivable with the
finite sort set $S_n = \set{\Any_0, \Any_1, \Any_2, \ldots, \Any_n}$ have an
$i$-normal form for all $0 < i \le n + 1$. The proof goes by induction from $i =
n+1$ down to $1$. The base case $i = n + 1$ is trivial because there are no
$n+1$-level types and therefore no $n+1$-level redexes. All terms are
$n+1$-normal.

In the following we concentrate on the induction step assuming that there exists
an $i+1$-normal form for all welltyped terms and construct from that an
$i$-normal form of the term.

\begin{definition}
    The \emph{$i$-degree of a redex}
    %------------------------------------------------------------
    in an $i+1$ normal term under the assumption
    that every welltyped term is reducible to an $i+1$ normal form is defined as
    the $i$-degree of the $i+1$-normalized minimal type of the function term of
    the redex.

    Note that the notion of the $i$-degree of a redex is welldefined, because
    all minimal types of the function term are betaequivalent and
    by~\ref{thm:BetaEquivalentSameDegree} all betaequivalent types have the same
    degree.
\end{definition}



\begin{definition}
    The \emph{complexity measure of an $i+1$-normal term $t$}
    %------------------------------------------------------------
    is defined as pair of numbers
    $$
        (d, n_d)
    $$
    where $d$ is the highest $i$-degree of a $i$-level redex in $t$ and $n_d$ is the
    number of $i$-level redexes with $i$-degree $d$.
\end{definition}


\begin{definition}
    An \emph{i-normalization step} $t \torel{i} u$
    %------------------------------------------------------------
    of an $i+1$-normal welltyped
    term $t$ consists of reducing the rightmost redex of level $i$ and $i$-degree
    $d$ in the term where $d$ is the highest $i$-degree of all $i$ level redexes in
    the term.

    If there are no $i$-level redexes, then the term is already in $i$-normal
    form.
\end{definition}






\begin{theorem}
    Each $i$-normalization step of an $i+1$-normal term $t$ reduces the
    complexity measure $(d, n_d)$ lexicographically.
    \label{thm:INormalizationStepReduceComplexity}

    \begin{proof}
        An $i$-normalization step reduces the rightmost redex of level $i$ and
        $i$-degree $d$ where $d$ is the highest degree of $i$-level redexes in
        $t$.

        The reduction step certainly reduces the number of $i$-degree $d$
        redexes by one because it reduces one of them. We have to make sure that
        it cannot increase the number.

        Since it reduces the rightmost redex of $i$-degree $d$, the argument of
        the redex
        does not contain any redexes of $i$-degree $d$. Therefore the reduction
        step does not increase the number of $i$-degree $d$ redexes by copying
        them.

        There are three possiblities that a reduction can create new redexes. We
        assume that the newly created redexs have level $i$, otherwise they are
        not critical.
        \begin{enumerate}
            \item The reduction step is
                $$
                (\underbrace{\lambda x^A. f(xa)}_{:\, \Pi x^A. B})
                (\underbrace{\lambda z^C. e}_{:\, A' \le A})
                \reduce
                f((\lambda z^C.e)a)
                $$
                where $f$ is a term with zero or more occurrences of the subterm
                $x a$ and the annotated types are minimal and have $i+1$-normal
                form. The reduced term may have zero or more occurrences of the
                redex $(\lambda z^C.e) a$.

                For the $i$-degree of $A'$ we get
                $$
                \begin{array}{llll}
                    \Degree_i(A')
                    &<& 1 + \Degree_i(A') + \Degree_i(B)
                    \\
                    &\le & 1 + \Degree_i(A) + \Degree_i(B)
                    & \text{\ref{thm:SubtypeRespectsDegree}}
                    \\
                    & = & \Degree_i(\Pi x^A. B) &\text{by definition}
                \end{array}
                $$
                which has lower degree.


            \item The reduction step is
                $$
                (\underbrace{\lambda x^A. x}_{:\, \Pi x^A. A})
                (\underbrace{\lambda z^C. e}_{:\, A' \le A})
                a
                \reduce
                (\lambda z^C. e) a
                $$

                By the same reasoning as above we get
                $$
                \begin{array}{llll}
                    \Degree_i(A')
                    &<& 1 + \Degree_i(A') + \Degree_i(A)
                    \\
                    &\le & 1 + \Degree_i(A) + \Degree_i(A)
                    & \text{\ref{thm:SubtypeRespectsDegree}}
                    \\
                    & = & \Degree_i(\Pi x^A. A) &\text{by definition}
                \end{array}
                $$


            \item The reduction step is
                $$
                (\underbrace{\lambda x^A.
                \underbrace{\lambda z^C.e}_{: B}}_{:\, \Pi x^A. B})
                a
                b
                \reduce
                (\underbrace{\lambda z^{C[a/x]}.  e[a/x]}_{: B[a/x]})
                b
                $$
                We can derive
                $$
                \begin{array}{llll}
                    \Degree_i(B[a/x])
                    &\le& \Degree_i(B)
                    &\text{\ref{thm:DegreeTypesafeSubstitution}}
                    \\
                    &<& 1 + \Degree_i(A) + \Degree_i(B)
                    \\
                    &=& \Degree_i(\Pi x^A.B)
                    &\text{definition}
                \end{array}
                $$
                Therefore the newly created redex has a strict lower degree than
                the reduced redex.
        \end{enumerate}
    \end{proof}
\end{theorem}


Since each $i$-normalization step reduces the complexity measure
lexicographically it is guaranteed that repeated $i$-normalization steps
terminate after a finite number of steps with an $i$-normal term.

This proves the validity of the induction step from $i+1$-normal terms to
$i$-normal terms. Therefore after finitely many induction steps we can generate
for all welltyped terms an 1-normal term i.e. a term which contains only
propositional redexes.
