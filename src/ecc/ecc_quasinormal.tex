%-------------------------------------------------------------------------------
\section{Quasinormalization}
%-------------------------------------------------------------------------------

\begin{comment}
    Level:

        A set of sorts a type can live in, even if beta reduced.

        Note:
            Beta reduction can reduce the sort. E.g.

                (\ (A: Any (i+k+1)) := A) (Any i) ~> Any i

            where `Any i` has type  `Any (i+1)` and
            `(\ (A: Any (i+k+1)) := A) (Any i)` has type `Any (i+k+1)` because
            the result of the function term is `A` which has type `Any (i+k+1)`.


    Level of a redex:

        All redexes have a function term which has a minimal type of the form
        'all (x: A): B' (or beta equivalents). The level of this type is the
        level of a redex.


    j-normal term:

        Does not have redexes of level 'upper (Any j)' or subsets of it.

        0-normal terms are normal terms, because they contain no redexes.


    Proof idea:

        In ECCn a redex with sort set upper (Any n) is not possible. Therefore
        we have a function rn which reduces any term to its n-normal form. Since
        all terms are in n-normal form rn is the identity function.

        Induction step: Assume there is a reduction function ri+1, find a
        reduction function ri.

            We look into redexes whose sort set of its function term type is
            upper (Any i). We can transform all these types into i+1 normal form
            via ri+1. The normalized type is a sort or a product or a base term.
            We can compute the degree of all these types.

            We use the triple

                (i, d, n)

            where d is the maximal degree of all these types and n is the number
            of redexes which have this maximal degree.

            We reduce the rightmost redex of degree d and claim that this
            reduces n by one. We continue until n=0 i.e. there are no more
            redexes of degree d.

            We continue this process until there are no more redexes of level i.

            The i-normal term is unique because the rightmost redex is unique.
            Therefore we have found the reduction function ri.

        Crucial in the construction: The reduction of the rightmost redex of the
        highest degree does not make new redexes of this degree.
\end{comment}








\subsection{Sort Sets}
%----------------------------------------------------------------------


\begin{definition}
    \label{def:UpperSet}
    %-----------------------------
    The \emph{upper set} $\upper {\Any_i}$ of a sort $\Any_i$ are all sorts
    $\Any_j$ where $i \le j$.
    $$
    \upper {\Any_i} := \set{\Any_j \mid i \le j}
    $$
\end{definition}




\begin{definition}
    \label{def:SortSet}
    %-----------------------------
    The \emph{sort set $S_\Gamma(T)$ of a type $T$} is defined inductively by
    the rule
    $$
    \ruleh{
        \Gamma \vdash T : s_1\quad
        T \reducestar U\quad
        \Gamma \vdash U : s
    }
    {
        s \in S_\Gamma(T)
    }
    $$

    We often write $S(T)$ without the subscript $\Gamma$ when the context is
    clear.
\end{definition}





\begin{theorem}
    \label{thm:SortSetUpperClosed}
    %-----------------------------
    \emph{A sort set is upper closed}.
    $$
        \ruleh{
            \Any_i \in S_\Gamma(T)
            \\
            i \le j
        }
        {
            \Any_j \in S_\Gamma(T)
        }
    $$
    \begin{proof}
        By the cumulativity rule of the typing relation.
    \end{proof}
\end{theorem}






\begin{theorem}
    \label{thm:SortSetSubstitution}
    %-----------------------------
    \emph{Type safe substitution makes the sort set bigger}.
    $$
        \ruleh{
            \Gamma \vdash a : A \quad \Gamma,x^A \vdash B
        }
        {
            S_{\Gamma,x^A}(B) \subseteq S_\Gamma(B[a/x])
        }
    $$
    \begin{proof}
        Assume $s \in S_{\Gamma,x^A}(B)$.

        Then by the definition of sort sets~\ref{def:SortSet}
        there exists $C$ with $B \reducestar C$ and $\Gamma,x^A \vdash C: s$.
        This implies $B[a/x] \reducestar C[a/x]$ and $\Gamma \vdash C[a/x] : s$
        by the subject reduction~\ref{SubjectReduction} and the substitution
        theorem~\ref{SubstitutionLemma}.

        The definition of sort sets~\ref{def:SortSet} implies $s \in
        S_\Gamma(B[a/x])$ which completes the proof.
    \end{proof}
\end{theorem}






\begin{theorem}
    \label{thm:SortSetProduct}
    %-----------------------------
    \emph{The sort set of a product is either $\upper\Prop$ in case the product
    is a proposition or the intersection of the sort sets of the argument and
    the result type.}
    $$
    \ruleh{
        \Gamma \vdash \Pi x^A. B : \dontcare
    }
    {
        S(\Pi x^A. B) =
        \dcases{
            \upper\Prop & \text{if }\Gamma \vdash \Pi x^A . B : \Prop
            \\
            S(A) \cap S(B) & \text{otherwise}
        }
    }
    $$

    This theorem has the immediate consequence $S(\Pi x^A. B) \subseteq S(B)$. In
    the propositional case this is trivial. In the non-propositional case it is
    a consequence of the intersection.

    \begin{proof}
        Now we prove the main theorem.

        If the product is a proposition the proof is trivial because
        $\upper\Prop$ is the biggest possible sort set. For the
        non-propositional case we have to prove both directions:
        \begin{enumerate}
            \item $S(\Pi x^A. B) \subseteq S(A) \cap S(B)$:
                Assume $s$ is in the sort set of the product. By definition
                there exist reducts $A'$ and $B'$ of $A$ and $B$ such that
                $\Gamma \vdash \Pi x^{A'}. B' : s$. By the generation
                lemma~\ref{GenerationLemma} for products there exist a sort
                $s_0$ which is a valid type of $A'$ and $B'$ with $s_0 \le s$.
                Therefore $s_0$ is in the sort sets of $A$ and $B$. Because sort
                sets are upper closed $s$ is in both sort sets as well.

            \item $S(A) \cap S(B) \subseteq S(\Pi x^A. B)$:
                Assume $s$ is in both sort sets of $A$ and $B$. By definition
                there exist reducts $A'$ and $B'$ such that $s$ is a type of
                $A'$ and $B'$. The introduction rule for products in the typing
                relation says that $s$ is a type of the product $\Pi x^{A'}.
                B'$. By definition of the sort set $s$ is in the sort set of
                $\Pi x^A. B$.
        \end{enumerate}
    \end{proof}
\end{theorem}






\begin{theorem}
    \label{thm:SortSetsSubtype}
    %-----------------------------
    \emph{Let $A$ and $B$ be some welltyped terms in the same context. Then the
    following is valid:}
    \begin{enumerate}

        \item $A \betaeq B \imp S(A) = S(B)$

        \item $A <_i B \imp S(B) \subset S(A)$
    \end{enumerate}
    \begin{proof}

        \ \begin{enumerate}

            \item Because of confluence there exists a common reduct $C$ of $A$
                and $B$. Then by definition of the sort set $S(A) = S(C) =
                S(B)$.

            \item By induction on $ A <_i B$.
                \begin{enumerate}
                    \item The base case is trivial.

                    \item In the induction step we have $\Pi x^A . B_1 <_{i+1}
                        \Pi x^A. B_2$ where $S(B_2) \subset S(B_1)$ is given by
                        the induction hypothesis.

                        In case that $\Pi x^A. B_1$ is a proposition the goal is
                        trivial because $\upper\Prop$ is the largest possible
                        sort set.

                        It remains the case that both are not propositions ($\Pi
                        x^A. B_2$ cannot be a proposition because it is a proper
                        supertype).

                        Then we have
                        $$
                        S(\Pi x^A. B_2) = S(A) \cap S(B_2) \subset S(A) \cap
                        S(B_2) = S(\Pi x^A. B_1)
                        $$
                        by using the induction hypothesis
                        and~\ref{thm:SortSetProduct}.
                \end{enumerate}
        \end{enumerate}
    \end{proof}
\end{theorem}






\begin{theorem}
    \label{thm:SortSetApplication}
    %-----------------------------
    \emph{Sort sets of applications} Let $fa$ be an application where $F$ is a
    minimal type of $f$ and $R$ is a minimal type of $fa$. Then the sort set of
    $F$ is a subset of the sort set of $R$.
    $$
    \rulev{
        \Gamma \vdashmin f : F
        \\
        \Gamma \vdashmin fa : R
    }
    {
        S(F) \subseteq S(R)
    }
    $$
    \begin{proof}

        By~\ref{thm:MinimalTypeFunctionTerm} there exist the types $A$ and $B$
        such that $\Pi x^A. B$ is a minimal type of $f$. Therefore $F$ and $\Pi
        x^A.B$ are beta equivalent and have the same sort sets.

        The generation lemma~\ref{GenerationLemma} for application guarantees
        the existence of $A'$ and $B'$ such that $\Pi x^{A'}.B'$ is a valid type
        of $f$ and $A'$ is a valid type of $a$ and $\sub a x {B'} \le R$.

        Since $\Pi x^A. B$ is minimal we have $A \betaeq A'$ and $B \le B'$
        by~\ref{thm:ProductSubtype}.

        The minimality of $R$, the antisymmetry of
        subtypes~\ref{thm:SubtypeAntisymmetric} and the fact that
        substitution respects subtypes~\ref{thm:SubstitutionRespectsSubtype} we
        have $\sub a x B \betaeq R$.

        In combination with \ref{thm:SortSetsSubtype} and
        \ref{thm:SortSetProduct} and \ref{thm:SortSetSubstitution} we get
        $$
        S(F) = S(\Pi x^A. B)
            \subseteq S(B)
            \subseteq S(\sub a x B)
            = S(R)
        $$
    \end{proof}
\end{theorem}






\begin{theorem}
    \label{thm:SortSetApplications}
    %-----------------------------
    \emph{Sort sets of generalized applications} This is a generalization of the
    previous theorem~\ref{thm:SortSetApplication}.

    Let $fa \vec b$ be an application where
    $F$ is a minimal type of $f$ and $R$ is a minimal type of $fa \vec b$ where
    $\vec b$ is a possibly empty sequence of arguments $b_1, b_2, \ldots, b_n$.
    Then the sort set of $F$ is a subset of the sort set of $R$.
    $$
    \rulev{
        \Gamma \vdashmin f : F
        \\
        \Gamma \vdashmin f a \vec b : R
    }
    {
        S(F) \subseteq S(R)
    }
    $$
    \begin{proof}
        We prove the goal by induction on the length $n$ of $\vec b$.
        \begin{enumerate}
            \item $n=0$: Immediate consequence of the previous
                theorem~\ref{thm:SortSetApplication}.

            \item Induction step from $n$ to $n+1$:
                Assume $S(F) \subseteq S(R)$ for $n = |\vec b|$. We have to
                prove the goal
                $$
                \rulev{
                    \Gamma \vdashmin f a \vec b b_{n+1} : R'
                }
                {
                    S(F) \subseteq S(R')
                }
                $$
                By using the the induction hypothesis and the previous
                theorem~\ref{thm:SortSetApplication} we can prove the goal
                $$
                    S(F) \subseteq S(R) \subseteq S(R')
                $$
        \end{enumerate}
    \end{proof}
\end{theorem}






\subsection{$i$-Normal Terms}
%------------------------------------------------------------


\begin{definition}
    \label{def:RedexSortSet}
    %-----------------------------
    \emph{Sort set of a redex}

    The sort set of a redex $(\lambda x^A. e) a$ is the sort set of a
    minimal type of its function term $\lambda x^A. e$.

    This set is welldefined because a minimal type of a term is unique up to
    beta equivalence and due to~\ref{thm:SortSetsSubtype} all beta equivalent
    types have the same sort set.
\end{definition}




\begin{definition}
    \label{def:INormalTerm}
    %-----------------------------
    \emph{Welltyped $i$-normal terms}

    A welltyped term $t$ is $i$-normal if it contains no redexes whose sort set
    is a subset of $\upper {\Any_i}$.
\end{definition}






\begin{theorem}
    \label{thm:NormalBaseTerm}
    %-----------------------------
    \emph{Let $fa$ be a welltyped $i+1$-normal type whose sort set is a subset
    of $\upperany i$.  Then $fa$ is a base term.}

    \begin{proof}
        We proof the more general theorem using $f a \vec b$ as the welltyped
        $i+1$-normal term whose sort set is a subset of  $\upperany i$ and prove
        that $f a \vec b$ is a base term. $\vec b$ is a possibly empty sequence
        $b_1, b_2, \ldots, b_n$ of arguments.

        The proof goes by induction on the structure of the function term $f$.
        \begin{enumerate}

            \item $f$ can neither be a sort nor a product. This would contradict
                the fact that $f b \vec c$ is welltyped.

            \item If $f$ is a variable then $f a \vec b$ is by definition a base
                term.

            \item $f$ is the application $ga_0$: In that case $ga_0 a \vec b$ is
                a welltyped $i+1$-normal type whose sort set is $\upperany i$
                and by the induction hypothesis for $g$ we conclude that $g a_0
                a \vec b = f a \vec b$ is a base term.

            \item $f$ is an abstraction:

                By the existence of minimal types~\ref{MinimalType} there exist
                the types $F$ and $\Any_j$ which are the minimal types of $f$
                and $f b \vec c$ respectively. The minimal type of $f b \vec c$
                must be a sort, because the term is a type.
                By~\ref{thm:SortSetApplications} we infer $S(F) \subseteq
                S(\Any_j)$.

                By definition of a sort set $\Any_j$ must be in the sort set of
                $f a \vec b$ which by assumption is a subset of $\upper \Any_i$
                $ \Any_j \subseteq S(f a \vec b) \subseteq \upper {\Any_i}$.
                This is possible only if $i \le j$. This implies $S(\Any_j) =
                \set{\Any_{j+1}, \Any_{j+2}, \ldots} \subset \upper{\Any_i}$.

                Combining the two facts we get $S(F) \subseteq S(\Any_j) \subset
                \upper{\Any_i}$ or better $S(F) \subseteq \upper{\Any_{i+1}}$.

                $f$ is an abstraction, therefore $f a$ is a redex whose minimal
                type is $F$. This implies that $f a \vec b$ contains a redex
                whose sort set is contained in $\upper {\Any_{i+1}}$ and
                contradicts the assumption that $f a \vec b$ is $i+1$-normal.
                Therefore $f$ cannot be an abstraction.
        \end{enumerate}
    \end{proof}
\end{theorem}



\begin{theorem}
    \label{thm:FormNormalTypes}
    %------------------------------------------------------------
    \emph{Form of Normal Types} A welltyped $i+1$-normal type $T$ whose sort set
    is a subset of $\upper{\Any_i}$ is either a sort or a product or a base
    term.


    \begin{proof}
        By induction of the structure of $T$.
        \begin{itemize}

            \item If $T$ is a sort or a product or a variable then the goal is
                trivial.

            \item $T$ cannot be an abstraction because the type of an
                abstraction cannot be a sort i.e. it cannot be a welltyped type.

            \item If $T$ is an application it is a base term. This is an
                immediate consequence of~\ref{thm:NormalBaseTerm}.
        \end{itemize}
    \end{proof}
\end{theorem}



\subsection{Degree}
%------------------------------------------------------------

\begin{definition}
    \label{def:TypeDegree}
    \emph{The degree of a welltyped $i+1$-normal type is defined recursively by}

    $$
    D_i(T) :=
    \dcases{
        D_i(T) &:=& 0 \quad\text{if } \upper{\Any_i} \subset S(T)
        \\
        D_i(s) &:=& 0
        \\
        D_i(x \vec a) &:=& 0
        \\
        D_i(\Pi x^A.B) &:=&
        1 + \text{max}(D_i(A), D_i(B))
    }
    $$

    Note that the degree is welldefined by~\ref{thm:FormNormalTypes} which
    asserts that the type $T$ is either a sort or a base term or a product if
    its sort set is a subset of $\upper{\Any_i}$.
    Furthermore subexpressions of $i+1$ normal types are $i+1$ normal as well.
\end{definition}



\begin{definition}
    \label{def:RedexDegree}
    \emph{Degree of a redex} Assume we have a function $r$ to transform any
    welltyped term $t$ into $i+1$-normal form $r(t)$ by a finite sequence of
    beta reductions. Assume further that the term $t$ is $i+1$-normal (if not,
    we can use $r$ to reduce it into $i+1$-normal form). Then by definition the
    term $t$ only has redexes whose sort set is a subset of $\upper \Any_i$.

    We define the degree of a redex $(\lambda x^A.e)a$ in $t$ by $D_i(r(T))$
    where $T$ is a minimal type of its function term $\lambda x^A.e$.
\end{definition}







\subsection{Normalization}
%------------------------------------------------------------


\begin{definition}
    \label{def:NormalizationStep}
    \emph{Normalization step of an $i+1$-normal term}
    Let $r$ be a function over welltyped terms $t$ such that $r(t)$ is
    $i+1$-normal. Assume that the term $a$ is $i+1$ normal and has redexes with
    sort set $\upper{\Any_i}$. Each of these redexes has a degree according to
    the definition~\ref{def:RedexDegree}. Let $d$ be the highest degree of all
    these redexes. Then $a \torel {q_i} b$ reduces the rightmost redex of degree
    $d$ when $d > 0$.

    This definition has the following important consequences:
    \begin{itemize}
        \item
            If the highest degree $d$ of the redexes in the term $t$ is zero,
            then the term is $i$-normal.

            Reason: There is always a minimal type in $i+1$-normal form of the
            function term of the redex of the form $\Pi x^A. B$. By definition
            of the degree of a type~\ref{def:TypeDegree} its degree is at least
            $1$ as long as its sort set is a subset of $\upper \Any_i$. If the
            degree is zero, then its sort set is a proper superset of $\upper
            \Any_i$. Therefore $t$ has no redexes whose sort set is a subset of
            $\upper \Any_i$ and thus by definition is $i$-normal.

        \item
            If $a \torel {q_i} b$ is a normalization step of the $i+1$ normal
            term $a$, then the term $b$ is uniquely defined.

            Reason: The contracted redex is unique because it is the rightmost
            redex of degree $d$.
    \end{itemize}
\end{definition}


\begin{theorem}
    \label{thm:NewRedexCreation}
    \emph{Creation of new redexes during beta reduction} There are only three
    possibilities to create new redexes during a beta reduction which does the
    reduction $(\lambda x^A.e) a \reduce \sub a x e$ on any subterm of a term.

    $$
    \begin{array}{lll}
        (\lambda x^A. \ldots xb \ldots) (\lambda y^B. e)
        &\reduce&
        \ldots (\lambda y^B.e) b \ldots
        %
        \\
        %
        (\lambda x^A. x)(\lambda y^B. e) b
        &\reduce&
        (\lambda y^B. e) b
        %
        \\
        %
        (\lambda x^A . \lambda y^B. e) a b
        &\reduce&
        (\lambda y^{B[a/x]}. e[a/x]) b
    \end{array}
    $$
    where $xb$ in the first case is not the argument of an application i.e. $xb$
    is in a head position.

    \begin{proof}
        A redex consists of an abstraction applied to an argument. Both must
        have been in the original expression and come together after the
        reduction. If the abstraction is in the argument of the redex then the
        argument must be the abstraction. The abstraction replaces the bound
        variable of the original redex.

        If the original redex has $x b$ in a head position, then the
        substitution of the abstraction for $x$ creates a new redex. This is the
        first case.

        If the original redex has only the bound variable $x$ but not applied to
        an argument, then the body of the original redex must be $x$ and the
        argument comes from an argument in the original expression which follows
        the original redex. This is the second case.

        The third possibility has an abstraction without argument (otherwise it
        were already a redex) as the body of the original abstraction and the
        argument of the newly created redex is the argument following the
        argument of the original redex.
    \end{proof}
\end{theorem}



\begin{theorem}
    \emph{A sequence of normalization steps of an $i+1$-normal term $t$ with $i
    > 0$ according to definition~\ref{def:NormalizationStep} ends after a finite
    number of steps in an $i$-normal unique term.}

    \begin{proof}
        MISSING
    \end{proof}
\end{theorem}
