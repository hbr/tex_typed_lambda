\section{The Calculus}
%%%%%%%%%%%%%%%%%%%%%%%%%%%%%%%%%%%%%%%%%%%%%%%%%%%%%%%%%%%%%%%%%%%%%%%%%%%%%%%%

In this section we introduce the terms of the calculus of constructions and the
way to do computations with them.

Lambda terms in untyped lambda calculus are either variables $x,y,z,\ldots$,
abstractions $\lambda x.e$ or applications $ab$.

The calculus of constructions is a typed lambda calculus where terms and types
are expressed within the same syntax. Since all welltyped terms must have a type
we need types of types. The types of types in the calculus of constructions are
the two sorts $\Prop$ and $\Any$.

Since the calculus is typed, all variables must have a type. Therefore in
binders like abstractions the variables have types $\lambda x^A. e$.

Furthermore it is necessary to express functions $A \to B$ from one type $A$ to
another type $B$. The calculus of constructions includes dependent types, i.e.
the result type $B$ of a function might depend on the argument. In order to
express that a product type of the form $\Pi x^A. B$ is used which describes the
type of a function from an argument of type $A$ to the result of type $B$ which
might depend on the argument.




\subsection{Sorts}
%%%%%%%%%%%%%%%%%%%%%%%%%%%%%%%%%%%%%%%%%%%%%%%%%%%%%%%%%%%%%%%%%%%%%%%%%%%%%%%%


In the calculus of constructions terms and types are in the same
syntactic category. All welldefined terms have types and therefore types must
have types as well. In order to have types for types we start with the
introduction of sorts which are the types of types.

\begin{definition}
    \emph{Sorts:} There are the two sorts $\Prop$ and $\Any$ in the calculus of
    constructions.

    Sorts or universes are the types of types. Sorts are usually abbreviated by
    the variable $s$.
\end{definition}





\subsection{Terms}
%%%%%%%%%%%%%%%%%%%%%%%%%%%%%%%%%%%%%%%%%%%%%%%%%%%%%%%%%%%%%%%%%%%%%%%%%%%%%%%%

\begin{definition}
    The \emph{terms} are defined by the grammar where $s$ ranges over sorts, $x$
    ranges over some countably infinite set of variables and $t$ ranges over
    terms.
    $$
    \begin{array}{llll}
        t

        &::=& s & \text{sorts}

        \\

        &\mid & x & \text{variable}

        \\

        &\mid & \Pi x^t. t & \text{product}

        \\

        &\mid & \lambda x^t. t & \text{abstraction}

        \\

        &\mid & t t & \text{application}
    \end{array}
    $$
\end{definition}








\subsection{Free Variables}
%%%%%%%%%%%%%%%%%%%%%%%%%%%%%%%%%%%%%%%%%%%%%%%%%%%%%%%%%%%%%%%%%%%%%%%%%%%%%%%%

\begin{definition}
    The set of \emph{free variables} $\FV(t)$ of a term $t$ is defined by the
    function
    $$
    \FV(t) :=
        \begin{cases}
            \FV(s) &:= \emptyset

            \\

            \FV(x) &:= \{ x \}

            \\

            \FV(a b) &:= \FV(a) \cup \FV(b)

            \\

            \FV(\lambda x^A. e) &:= \FV(A) \cup (\FV(e) - \{x\})

            \\

            \FV(\Pi x^A. B) &:= \FV(A) \cup (\FV(B) - \{x\})
        \end{cases}
    $$
    where $s$ ranges over sorts, $x$ ranges over variables and $a$, $b$,
    $e$, $A$ and $B$ range over arbitrary terms.
\end{definition}

A variable which is not free is called a bound variable. E.g. if $x$ is a free
variable in the term $e$, it is no longer free in $\lambda x^A. e$. Therefore we
call $\lambda x^A. e$ a binder, because it makes the variable $x$ bound. The
same applies to the term $\Pi x^A. B$  where the variable $x$ is bound, but can
appear free in $B$.

Note that the binders $\lambda x^A. e$ and $\Pi x^A. B$ make the variable $x$
bound only in the subterms $e$ and $B$, but not in the subterm $A$.

It is possible to rename bound variables within a term. The renaming of a bound
variable does not change the term. We consider two terms which only differ in
the name of bound variables as identical. Examples of some identical terms:
$$
\begin{array}{lll}
    \lambda x^\Prop. x  &=& \lambda y^\Prop. y

    \\

    \Pi x^z. x   &=& \Pi y^z. y
\end{array}
$$






\subsection{Contexts}
%%%%%%%%%%%%%%%%%%%%%%%%%%%%%%%%%%%%%%%%%%%%%%%%%%%%%%%%%%%%%%%%%%%%%%%%%%%%%%%%


All bound variables get their types by their corresponding binders. For free
variables we need types as well. In order to assign types to free variables we
define contexts.

\begin{definition}
    A \emph{context} is a sequence of variables and their corresponding types.
    Contexts are usually abbreviated by upper greek letters and types are terms
    which are usually abbreviated by uppercase letters.

    Contexts are defined by the grammar
    $$
    \begin{array}{llll}
        \Gamma
        &:=& [] & \text{empty context}

        \\

        &\mid& \Gamma, x^A & \text{one more variable $x$ with its type $A$}
    \end{array}
    $$
\end{definition}







\subsection{Substitution}
%%%%%%%%%%%%%%%%%%%%%%%%%%%%%%%%%%%%%%%%%%%%%%%%%%%%%%%%%%%%%%%%%%%%%%%%%%%%%%%%


\begin{definition}
    \emph{Substitution:}
    %%%%%%%%%%%%%%%%%%%%%%%%%%%%%%%%%%%%%%%%%%%%%%%%%%%%%%%%%%%%%%%%%%%%%%%%%%%%%%%%
    The term $t[y:=u]$ is the term $t$ where the term $u$ is substituted for all
    free occurrences of the variable $y$. It is defined as a recursive function
    which iterates over all subterms until variables or sorts are encountered.
    $$
    t[y:=u] :=
        \begin{cases}
            s[y:=u] &:= s

            \\

            x[y:=u] &:=
                \begin{cases}
                    u & y = x
                    \\
                    x & y \ne x
                \end{cases}

            \\

            (a b)[y:=u] &:= a[y:=u] b[y:=u]

            \\

            (\lambda x^A . e)[y:=u]
            &:=
            \lambda x^{A[y:=u]}.e[y:=u]
                \quad y \ne x, x \notin \FV(u)

            \\

            (\Pi x^A . B)[y:=u]
            &:=
            \Pi x^{A[y:=u]}. B[y:=u]
                \quad y \ne x, x \notin \FV(u)
        \end{cases}
    $$
\end{definition}

Remark:
        The conditions $y \ne x$ and $x \notin \FV(u)$ are not a restriction,
        because the bound variable $x$ can always be renamed to annother
        variable which is different from $y$ and does not occur free in the term
        $u$.



\begin{lemma}
    \label{SubstitutionToSort}
    \emph{Substitution to sort lemma:}
    If the result of a substitution is a sort then either the term in
    which the substitution occurs is the term or the term is the variable to be
    replaced and the substitution term is the sort.
    $$
        \rulev{
            a[x:=b] = s
        }
        {
            a = s \lor (a = x \land b = s)
        }
    $$
    \begin{proof}
        From the definition of the substitution it is evident that only the
        first two cases can result in a sort. All other cases cannot
        syntactically result in a sort. The first two cases of the definition
        prove exactly the goal.
    \end{proof}
\end{lemma}


\begin{lemma}
    \label{DoubleSubstitution}
    %%%%%%%%%%%%%%%%%%%%%%%%%%%%%%%%%%%%%%%%%%%%%%%%%%%%%%%%%%%%%%%%%%%%%%%%%%%%%%%%
    \emph{Double substitution lemma} If we change the order of two subsequent
    substitutions, then it is necessary to introduce a correction term.
    $$
    \ruleh{
        x \ne y
        \\
        y \notin \FV(a)
    }
    {
        t[y:=b][x:=a]
        =
        t[x:=a][y:=b[x:=a]]
    }
    $$
    \begin{proof}
        The intuitive proof is quite evident. If we substitute $b$ for the
        variable $y$ in the term $t$ and then substitute $a$ for the variable
        $x$ in the result we replace in the second substitution not only all
        variables $x$ originally contained in $t$ but all occurrences of $x$ in
        $b$ as well.

        If we do the substitution $[x:=a]$ on $t$, then all occurrences of $x$
        in $b$ have not yet been substituted. Therefore the correction term
        $b[x:=a]$ is necessary if the order is changed.

        The formal proof goes by induction on the structure of $t$.

        \begin{enumerate}

            \item If $t$ is a sort then the equality is trivial because a sort
                does not have any variables.

            \item If $t$ is an application, an abstraction or a product, then
                the goal follows immediately from the induction hypotheses.

            \item The only interesting case is when $t$ is a variable. Let's
                call the variable $z$. Then we have to distinguish the cases
                that $z$ is $x$ or $z$ is $y$ or $z$ is different from $x$ and
                $y$.

                \begin{enumerate}
                    \item Case $z = x$:
                        $$
                        \begin{array}{llll}
                            x[y:=b][x:=a]
                            &=& x[x:=a]
                            \\
                            &=& a
                            \\
                            \\
                            x[x:=a][y:=b[x:=a]]
                            &=& a[y:=b[x:=a]]
                            \\
                            &=& a & y \notin \FV(a)
                        \end{array}
                        $$

                    \item Case $z = y$:
                        $$
                        \begin{array}{llll}
                            y[y:=b][x:=a]
                            &=& b[x:=a]
                            \\
                            \\
                            y[x:=a][y:=b[x:=a]]
                            &=& y[y:=b[x:=a]]
                            \\
                            &=&b[x:=a]
                        \end{array}
                        $$

                    \item Case $z \ne x \land z \ne y$:
                        $$
                        \begin{array}{llll}
                            z[y:=b][x:=a]
                            &=& z
                            \\
                            \\
                            z[x:=a][y:=b[x:=a]]
                            &=& z
                        \end{array}
                        $$

                \end{enumerate}
        \end{enumerate}
    \end{proof}
\end{lemma}

The validity of the
\emph{double substitution lemma}~\ref{DoubleSubstitution}
is needed to make sure that \emph{beta
reduction} (see definition in the next section) is confluent. E.g. if we have
the term $(\lambda x^A. (\lambda y^B. t) b) a$ then we can decide whether we
reduce first the inner redex and then the outer redex or the other way round.
Because of confluence both possibilities shall have the same result.
$$
\begin{array}{llll}
    (\lambda x^A. (\lambda y^B. t) b) a
    &\reduce& (\lambda x^A. t[y:=b]) a
    \\
    &\reduce& t[y:=b][x:=a]
    \\
    \\
    (\lambda x^A. (\lambda y^B. t) b) a
    &\reduce& (\lambda y^B. t[x:=a]) b[x:=a]
    \\
    &\reduce& t[x:=a][y:=b[x:=a]]
\end{array}
$$









\subsection{Beta Reduction}
%%%%%%%%%%%%%%%%%%%%%%%%%%%%%%%%%%%%%%%%%%%%%%%%%%%%%%%%%%%%%%%%%%%%%%%%%%%%%%%%

Like in untyped lambda calculus, computation is done via \emph{beta reduction}.
A beta redex has the form $(\lambda x^A. e) a$ which reduces to the reduct
$e[x:=a]$. Beta reduction can be done in any subterm of a term.

\begin{definition}
    \emph{Beta reduction}
    %%%%%%%%%%%%%%%%%%%%%%%%%%%%%%%%%%%%%%%%%%%%%%%%%%%%%%%%%%%%%%%%%%%%%%%%%%%%
    is a binary relation $a \reduce b$ where the term $a$
    reduces to the term $b$. It is defined by the rules
    \begin{enumerate}
        \item Redex:
            $$
                (\lambda x^A. e) a \reduce e[x:=a]
            $$

        \item Reduce function:
            $$
                \ruleh{
                    f \reduce g
                }
                {
                    f a \reduce g a
                }
            $$

        \item Reduce argument:
            $$
                \ruleh{
                    a \reduce b
                }
                {
                    f a \reduce f b
                }
            $$

        \item Reduce abstraction argument type:
            $$
                \ruleh{
                    A \reduce B
                }
                {
                    \lambda x^A. e \reduce \lambda x^B . e
                }
            $$

        \item Reduce abstraction inner term:
            $$
                \ruleh{
                    e \reduce f
                }
                {
                    \lambda x^A. e \reduce \lambda x^A. f
                }
            $$

        \item Reduce product argument type:
            $$
                \ruleh{
                    A \reduce B
                }
                {
                    \Pi x^A. C \reduce \Pi x^B . C
                }
            $$

        \item Reduce product result type
            $$
                \ruleh{
                    B \reduce C
                }
                {
                    \Pi x^A. B \reduce \Pi x^A. C
                }
            $$
    \end{enumerate}
\end{definition}

\begin{lemma}
    \emph{Reduction to sort lemma}: A term which reduces to a sort must be a
    redex where the sort is the body or the body is the variable and the
    argument is the sort.
    $$
    \ruleh{
        t \reduce s
    }
    {
        t = (\lambda x^A. s) a \lor t = (\lambda x^A. x)s
    }
    $$

    \begin{proof}
        All rules except the redex rule reduce to something which cannot be
        syntactically a sort. Therefore the term has to be a redex which in
        general has the form $(\lambda x^A.e) a$. The redex reduces to $e[x:=a]$
        which by the substitution to sort lemma~\ref{SubstitutionToSort} proves
        the goal.
    \end{proof}
\end{lemma}


\begin{theorem}
    \label{SubstituteReduction}
    %%%%%%%%%%%%%%%%%%%%%%%%%%%%%%%%%%%%%%%%%%%%%%%%%%%%%%%%%%%%%%%%%%%%%%%%%%%%%%%%
    \emph{Substitute Reduction} A reduction remains valid if we do the same
    substitution before and after the reduction.
    $$
    \ruleh{
        t \reduce u
    }
    {
        t[y:=v] \reduce u[y:=v]
    }
    $$
    (Note that iterated application of this lemma results in the more general
    statement $t \reducestar u \imp t[y:=v] \reducestar u[y:=v]$.)

    \begin{proof}
        Proof by induction on $t \reduce u$.


        \begin{enumerate}

        \item Redex: We have to prove the goal
            $$
                ((\lambda x^A. e) a)[y:=v] \reduce e[x:=a][y:=v]
            $$

            We can see this by the sequence
            $$
            \begin{array}{lll}
                ((\lambda x^A. e) a)[y:=v]
                &=&
                (\lambda x^{A[y:=v]}. e[y:=v]) a[y:=v]
                \\
                &\reduce&
                e[y:=v][x:=a[y:=v]]
                \\
                &=& \text{by Double substitution lemma~\ref{DoubleSubstitution}}
                \\
                &&e[x:=a][y:=v]
            \end{array}
            $$

        \item Reduce function: We have to prove the goal
            $$
            \begin{array}{l|l}
                f \reduce g
                & f[y:=v] \reduce g[y:=v]
                \\
                \hline
                f a \reduce g a
                &(f a)[y:=v] \reduce (g a)[y:=v]
            \end{array}
            $$
            The validity of the final goal in the right lower corner can be seen
            by the following reasoning
            $$
            \begin{array}{lll}
                (f a)[y:=v]
                &=&
                f[y:=v] a[y:=v]
                \\
                &\reduce&
                g[y:=v] a[y:=v]
                \\
                &=&
                (g a)[y:=v]
            \end{array}
            $$

        \item Other rules: All other rules follow the same pattern as the proof
            of the rule \emph{reduce function}.
        \end{enumerate}
    \end{proof}
\end{theorem}




\begin{theorem}
    \label{ReducedSubstitutionTerm}
    \emph{Reduction in substitution term}. A reduction in the substitution term
    %     ------------------------------
    makes the corresponding substituted terms reducing in zero or more steps as
    well.
    $$
    \ruleh{
        a \reduce b
    }
    {
        t[x:=a] \reducestar t[x:=b]
    }
    $$
    \begin{proof}
        Intuitively it might be clear that the substituted term reduces in zero
        or more steps, because the variable $x$ might be contained in the term
        $t$ zero or more times.

        We prove the statement by induction on the structure of $t$.

        \begin{enumerate}
        \item Sort: Trival.

        \item Variable: We have to distinguish the cases that the variable is
            the same as $x$ or is different from $x$. In both case the goal is
                trivial.

        \item Abstraction:
            We have to prove the goal
            $$
                (\lambda y^B.e)[x:=a] \reducestar (\lambda y^B.e)[x:=b]
            $$
            The goal is a consequence of the induction hypotheses $B[x:=a]
                \reducestar B[x:=b]$ and $e[x:=a] \reducestar e[y:=b]$.

        \item Product: Same as abstraction.

        \item Application: Same as abstraction.
        \end{enumerate}
    \end{proof}
\end{theorem}



\begin{lemma}
    \label{ReductionProductAbstraction}
    \emph{Products and abstractions are preserved under beta reduction}
    \begin{enumerate}
    \item
    $
        \ruleh{
            \Pi x^A.B \reduce t
        }
        {
            \left( \exists C. A \reduce C \land t = \Pi x^C. B \right)
            \lor
            \left( \exists D. B \reduce C \land t = \Pi x^A. D \right)
        }
    $
    \item
    $
        \ruleh{
            \lambda x^A.e \reduce t
        }
        {
            \left( \exists B. A \reduce B \land t = \lambda x^B. e \right)
            \lor
            \left( \exists f. e \reduce f \land t = \lambda x^A. f \right)
        }
    $
    \end{enumerate}

    \begin{proof}
        The proofs for product and abstraction follow the same pattern.
        Therefore we prove only the preservation of products.

        Assume $\Pi x^A. B \reduce t$ and do induction on it. Only the two
        product rules are syntactically possible. Each one guarantees one
        alternative of the goal.
    \end{proof}
\end{lemma}



\subsection{Beta Equivalence}
%%%%%%%%%%%%%%%%%%%%%%%%%%%%%%%%%%%%%%%%%%%%%%%%%%%%%%%%%%%%%%%%%%%%%%%%%%%%%%%%


\begin{definition}
    We define \emph{beta equivalence} as a binary relation $a \equiv b$ between
    the terms $a$ and $b$ inductively by the rules
    \begin{enumerate}
    \item Reflexive:
        $$ a \betaeq a$$

    \item Forward:
        $$
        \rulev
        {
            a \betaeq b
            \\
            b \reduce c
        }
        {
            a \betaeq c
        }
        $$

    \item Backward:
        $$
        \rulev
        {
            a \betaeq b
            \\
            c \reduce b
        }
        {
            a \betaeq c
        }
        $$
    \end{enumerate}

    In other words the beta equivalence relation $\betaeq$ is the smallest
    equivalence relation which contains beta reduction $\reduce$.
\end{definition}


\begin{theorem}
    \label{BetaEquivalenceTransitive}
    \emph{Beta equivalence is transitive}.
    $$
    \rulev{a \betaeq b \\ b \betaeq c}{a \betaeq c}
    $$
    \begin{proof}
        Assume $a \betaeq b$. We prove the goal by induction on $b \betaeq
        c$.
        \begin{enumerate}
        \item Reflexive: Trivial.

        \item Forward:
            $$
            \begin{array}{l|l}
                b \betaeq c
                &
                a \betaeq c
                \\
                c \reduce d
                \\
                \hline
                b \betaeq d
                &
                a \betaeq d
            \end{array}
            $$
            The goal in the lower right corner is proved by the induction
                hypothesis and applying the forward rule.

        \item Backward:
            $$
            \begin{array}{l|l}
                b \betaeq c
                &
                a \betaeq c
                \\
                d \reduce c
                \\
                \hline
                b \betaeq d
                &
                a \betaeq d
            \end{array}
            $$
            The goal in the lower right corner is proved by the induction
                hypothesis and applying the backward rule.
        \end{enumerate}
    \end{proof}
\end{theorem}

\begin{theorem}
    \label{SubstituteEquivalence}
    \emph{The same substitution applied to beta equivalent terms results in beta
    equivalent terms}.
    $$
    \ruleh{
        t \betaeq u
    }
    {
        t[x:=a] \betaeq u[x:=a]
    }
    $$

    \begin{proof}
        By induction on $t \betaeq u$:

        The reflexive case is trivial, because $t$ and $u$ are identical.

        For the other two cases the goal is a consequence of the induction
        hypothesis and the transitivity of beta
        equivalence~\ref{BetaEquivalenceTransitive}.
    \end{proof}
\end{theorem}
