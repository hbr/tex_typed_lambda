\section{Proof of Strong Normalization}


\subsection{Model Set}

\begin{definition}
$$
    \nu(T) :=
    \left \{
    \begin{array}{llll}
        \nu(\Prop) &:=& \SAT
        \\
        \nu(\Pi x^A. B) &:=& \nu(A)\to \nu(B) & A,B \in \Kinds
        \\
        \nu(\Pi x^A. B) &:=& \nu(B) & A \notin \Kinds \land B \in \Kinds
        \\
        \nu (T) &:=& \set{\emptyset} &\text{otherwise}
    \end{array}
    \right.
$$
\end{definition}


\subsection{Type Interpretation}

The goal of this section is to find a function which maps any welltyped term $t$
for all types $T$ it can have in a certain context $\Gamma$ into a value
$\typeinter{t}{\xi\Gamma}$ which is an element of $\nu (T)$. The function is
based on a variable interpretation $\xi$ which assigns to each variable $x$ of
type $A$ in the context a model which is an element of $\nu(A)$.

The following property of $\nu$ is convenient:
$$
    \rulev{
        \Gamma \vdash t: T
        \\
        \Gamma \vdash t: U
    }
    {
        \nu(T) = \nu(U)
    }
$$


In order to prove the property we prove first a similar lemma for kinds.

\begin{lemma}
    \label{ModelEquivalentKinds}
    \emph{The model set of a kind is the same as the model set of any beta
    equivalent type}.
    $$
    \rulev{
        K \in \Kinds
        \\
        \Gamma \vdash K: \Any
        \\
        \Gamma \vdash T: s_T
        \\
        K \betaeq T
    }
    {
        \nu(K) = \nu(T)
    }
    $$

    \begin{proof}
        By induction on the structure of $K$.

        General observation: Since $K$ and $T$ are betaequivalent we get from
        theorem~\ref{TypeUniqueness3} $s_T = \Any$ and
        by~\ref{TypeAnyImpliesKind} that $T$ is a syntactical kind. This is
        valid for all induction hypotheses.

        For the induction proof we distiguish three cases:
        \begin{enumerate}
        \item $K = \Prop$: Using the general observation above we conclude that
            $T$ is a syntactical kind. $\Prop$ is the only possible syntactical
                kind beta equivalent to $\Prop$. Therefore $T = \Prop$ which
                implies the goal $\nu(T) = \nu(\Prop)$ trivially.

        \item $K = \Pi x^{K_1}. K_2$:
            Since $T$ is a syntactical kind betaequivalent to $K$ it must have
                the form of a product ($\Prop$ not possible). I.e. $T = \Pi
                x^A.B$ for some types $A$ and $B$.

            By theorem~\ref{EquivalentBinders} we get the equivalences $K_1
                \betaeq A$ and $K_2 \betaeq B$.

            From both induction hypotheses we conclude $\nu(K_1) = \nu(A)$ and
            $\nu(K_2) = \nu(B)$.


        \item $K = \Pi x^A . K_2$ (where $A \notin \Kinds$):
            By the same reasoning as above we get $T = \Pi x^{A_T}.B$ for some
            types $A_T$ and $B$ with $A \betaeq A_T$ and $ K_2 \betaeq B$.

                From the induction hypothesis we get $\nu(K_2) = \nu(B)$.

            Since both $A$ and $A_T$ are valid types and $A$ is not a
                syntactical kind, we conclude by using~\ref{TypeUniqueness3}
                that $A_T$ cannot be a syntactical kind either.

            Therefore we get
            $$
                \nu(K) = \nu(K_2) = \nu(B) = \nu(T).
            $$
        \end{enumerate}
    \end{proof}
\end{lemma}


\begin{theorem}
    $$
        \rulev{
            \Gamma \vdash t: T
            \\
            \Gamma \vdash t: U
        }
        {
            \nu(T) = \nu(U)
        }
    $$

    \begin{proof}
        MISSING
    \end{proof}
\end{theorem}


