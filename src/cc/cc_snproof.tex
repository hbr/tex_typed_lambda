\section{Proof of Strong Normalization}



\subsection{Syntactical Kinds}

\begin{definition}
    The set of \emph{syntactical kinds} $\Kinds$ is defined as the set of terms
    generated by the grammar
    $$
    K ::= \Prop \;|\; \Pi x^A. K
    $$
    where $K$ ranges over kinds and $A$ ranges over arbitrary terms.
\end{definition}


\begin{theorem}
    \label{TypeAnyImpliesKind}
    \emph{Every term $K$ which has type $\Any$ is a syntactical kind}
    $$
    \rulev{
        \Gamma \vdash K: \Any
    }
    {
        K \in \Kinds
    }
    $$

    \begin{proof}
        By induction on $\Gamma \vdash K: \Any$.

        \begin{enumerate}
        \item Sort:
            $$
            [] \vdash \Prop: \Any
            $$
            Trival, because $\Prop \in \Kinds$.

        \item Variable: The variable rule deriving $\Gamma, x^\Any \vdash x:
            \Any$ is not applicable, because it would require $\Gamma \vdash
                \Any : s$ which contradicts the first generation
                lemma~\ref{GenerationLemmata}.

        \item Product:
            $$
            \begin{array}{l|l}
                \Gamma \vdash A: s_1
                \\
                \Gamma, x^A \vdash B: \Any
                &
                B \in \Kinds
                \\
                \hline
                \Gamma \vdash \Pi x^A. B : \Any
                &
                \Pi x^A . B \in \Kinds
            \end{array}
            $$
            The goal in the lower right corner is a consequence of the induction
                hypothesis.

        \item Abstraction:

            The abstraction rule deriving $\Gamma \vdash \lambda x^A. e : \Pi
                x^A. B$ is syntactically not possible, because $\Pi x^A.B \ne
                \Any$.

        \item Application:
            In order to apply the rule
            $$
            \rulev{
                \Gamma\vdash f: \Pi x^A. B
                \\
                \Gamma \vdash a: A
            }
            {
                \Gamma \vdash f a : B[x:=a]
            }
            $$
            the validity of $B[x:=a] = \Any$ would be necessary. This would
                require either $B = \Any$ or $B = x \land a = \Any$ which both
                lead to a contradiction
            \begin{itemize}
            \item $B = \Any$: The generation lemma~\ref{GenerationLemmata} for a
                product requires the existence of a sort $s$ with $\Gamma, x^A
                    \vdash \Any : s$ which requires by the generation lemma for
                    sorts that $\Any = \Prop$ which is not possible.

            \item $B = x \land a = \Any$: The generation lemma for sorts would
                require in that case $\Any = \Prop$ which is not possible.
            \end{itemize}

        \item Weaken: Trivial, because the goal is the same as the induction
            hypothesis.

        \item Reduction:
            In order to derive $\Gamma \vdash K: \Any$ by the reduction rule the
            validity of $\Gamma \vdash \Any : s$ is required. This contradicts
            the generation lemma~\ref{GenerationLemmata} for sorts.
        \end{enumerate}
    \end{proof}
\end{theorem}

\begin{theorem}
    \label{KindImpliesTypeAny}
    \emph{Every valid kind $K$ has type $\Any$.}
    $$
    \rulev{
        K \in \Kinds
        \\
        \Gamma \vdash K : T
    }
    {
        \Gamma \vdash K: \Any
    }
    $$
    \begin{proof}
        By induction on the structure of $K$
        \begin{enumerate}
        \item $\Prop$: The validity of $\Gamma \vdash \Prop: T$ implies by the
            generation lemma~\ref{GenerationLemmata} for sorts $T = \Any$.

        \item $\Pi x^A. K$: We assume $\Gamma \vdash \Pi x^A.K : T$ and have to
            prove $\Gamma \vdash \Pi x^A: K : \Any$. The generation
                lemma~\ref{GenerationLemmata} for products guarantees the
                existence of a sort $s$ with $\Gamma, x^A \vdash K: s$. This
                together with the induction hypotheses implies $\Gamma, x^A
                \vdash K: \Any$. By
                the introduction rule for products we get $\Gamma \vdash \Pi
                x^A. K : \Any$.
        \end{enumerate}
    \end{proof}
\end{theorem}



\subsection{Model Set}

\begin{definition}
$$
    \nu(T) :=
    \left \{
    \begin{array}{llll}
        \nu(s) &:=& \SAT
        \\
        \nu(\Pi x^A. B) &:=& \nu(A)\to \nu(B) & A,B \in \Kinds
        \\
        \nu(\Pi x^A. B) &:=& \nu(B) & A \notin \Kinds \land B \in \Kinds
        \\
        \nu (T) &:=& \set{\emptyset} &\text{otherwise}
    \end{array}
    \right.
$$
\end{definition}


\subsection{Type Interpretation}

The goal of this section is to find a function which maps any welltyped term $t$
for all types $T$ it can have in a certain context $\Gamma$ into a value
$\typeinter{t}{\xi\Gamma}$ which is an element of $\nu (T)$. The function is
based on a variable interpretation $\xi$ which assigns to each variable $x$ of
type $A$ in the context a model which is an element of $\nu(A)$.

The following property of $\nu$ is convenient:
$$
    \rulev{
        \Gamma \vdash t: T
        \\
        \Gamma \vdash t: U
    }
    {
        \nu(T) = \nu(U)
    }
$$
