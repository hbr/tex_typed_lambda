\section{Proof of Strong Normalization}

Barendregt~1993~\cite{barendregt1993}, Geuvers~1994~\cite{geuvers1994}.




\subsection{Saturated Sets}

MISSING



\begin{definition}
    \emph{If $A$ and $B$ are sets of lambda terms we define the lambda functions
    space $A \tolambda B$ as the set of lambda terms $f$ such that whenever $a$
    is in $A$ then $fa$ is in $B$}.
    $$
    A \tolambda B := \set{f \mid \forall a. a \in A \imp fa \in B}
    $$
\end{definition}

MISSING

\begin{theorem}
    \label{SaturatedSetsIntersection}
    \emph{The intersection of any collection of saturated sets is a saturated
    set}.
    $$
    \rulev{
        C \subseteq \SAT
    }
    {
        \bigcap C \in \SAT
    }
    $$
    \begin{proof}
        MISSING!!!
    \end{proof}
\end{theorem}





\begin{theorem}
    \label{LambdaSpaceSaturated}
    \emph{The lambda function space between saturated sets is a saturated set}.

    $$
    \rulev{
        A \in \SAT
        \\
        B \in \SAT
    }
    {
        A \tolambda B \in \SAT
    }
    $$
\end{theorem}




\subsection{Model Set}

\begin{definition}
    \emph{Model Set} For $K \in \Kinds$ we define the model set
    $\nu(K)$ by
    $$
        \nu(K) :=
        \left \{
        \begin{array}{llll}
            \nu(s) &:=& \SAT & s \text{ is a sort}
            \\
            \nu(\Pi x^A. B) &:=& \nu(A)\to \nu(B) & A,B \in \Kinds
            \\
            \nu(\Pi x^A. B) &:=& \nu(B) & A \notin \Kinds \land B \in \Kinds
        \end{array}
        \right.
    $$
\end{definition}



\begin{definition}
    \emph{Canonical Model} For $K \in \Kinds$ we define the
    canonical model $\nu^c(K)$ by
    $$
        \nu(K) :=
        \left \{
        \begin{array}{llll}
            \nu^c(s) &:=& \SN & s \text{ is a sort}
            \\
            \nu^c(\Pi x^A. B)
            &:=&
            \bullet \mapsto \nu^c(B) & A,B \in \Kinds
            \\
            \nu^c(\Pi x^A. B)
            &:=&
            \nu^c(B) & A \notin \Kinds \land B \in \Kinds
        \end{array}
        \right.
    $$
    where $\bullet \mapsto v$ is the constant function which maps any argument
    to the value $v$.
\end{definition}


The following property of $\nu$ is convenient:
$$
    \rulev{
        K \in \Kinds
        \\
        \Gamma \vdash F: K
        \\
        \Gamma \vdash F: T
    }
    {
        T \in \Kinds \land \nu(K) = \nu(T)
    }
$$


In order to prove the property we prove first a similar lemma for kinds.

\begin{lemma}
    \label{ModelEquivalentKinds}
    \emph{The model set of a kind is the same as the model set of any beta
    equivalent type}.
    $$
    \rulev{
        K \in \Kinds
        \\
        \Gamma \vdash K: \Any
        \\
        \Gamma \vdash T: s_T
        \\
        K \betaeq T
    }
    {
        T \in \Kinds \land \nu(K) = \nu(T)
    }
    $$

    \begin{proof}
        By induction on the structure of $K$.

        General observation: Since $K$ and $T$ are betaequivalent we get from
        theorem~\ref{TypeUniqueness3} $s_T = \Any$ and
        by~\ref{TypeAnyImpliesKind} that $T$ is a syntactical kind. This is
        valid for all induction hypotheses.

        For the induction proof we distiguish three cases:
        \begin{enumerate}
        \item $K = s$: In that case $K$ has to be $\Prop$, otherwise it would
            not be welltyped.

            Using the general observation above we conclude that
            $T$ is a syntactical kind. $\Prop$ is the only possible syntactical
                kind beta equivalent to $\Prop$. Therefore $T = \Prop$ which
                implies the goal $\nu(T) = \nu(\Prop)$ trivially.

        \item $K = \Pi x^{K_1}. K_2$:
            Since $T$ is a syntactical kind betaequivalent to $K$ it must have
                the form of a product (a sort is not possible). I.e. $T = \Pi
                x^A.B$ for some types $A$ and $B$.

            By theorem~\ref{EquivalentBinders} we get the equivalences $K_1
                \betaeq A$ and $K_2 \betaeq B$.

            From both induction hypotheses we conclude $\nu(K_1) = \nu(A)$ and
            $\nu(K_2) = \nu(B)$.


        \item $K = \Pi x^A . K_2$ (where $A \notin \Kinds$):
            By the same reasoning as above we get $T = \Pi x^{A_T}.B$ for some
            types $A_T$ and $B$ with $A \betaeq A_T$ and $ K_2 \betaeq B$.

                From the induction hypothesis we get $\nu(K_2) = \nu(B)$.

            Since both $A$ and $A_T$ are valid types and $A$ is not a
                syntactical kind, we conclude by using~\ref{TypeUniqueness3}
                that $A_T$ cannot be a syntactical kind either.

            Therefore we get
            $$
                \nu(K) = \nu(K_2) = \nu(B) = \nu(T).
            $$
        \end{enumerate}
    \end{proof}
\end{lemma}


\begin{theorem}
    \label{ModelAllTypesSame}
    \emph{If a type function $F$ is welltyped (i.e. its type is a kind), then
    the model sets of all its possible types are the same}.
    $$
    \rulev{
        K \in \Kinds
        \\
        \Gamma \vdash F: K
        \\
        \Gamma \vdash F: T
    }
    {
        T \in \Kinds \land \nu(K) = \nu(T)
    }
    $$

    \begin{proof}
        From~\ref{TypeUniqueness1} we infer $T \betaeq U$ and
        from~\ref{TypeUniqueness3} we infer that either both are $\Any$ or both
        are types of the same sort.

        If both are $\Any$, then $\nu(K) = \nu(T)$ is valid trivially.

        Assume both are types of the same sort i.e. we have $\Gamma \vdash K: s$
        and $\Gamma \vdash T: s$. Since $K$ is a syntactical kind $s = \Any$ is
        valid.

        Therefore the assumptions of lemma~\ref{ModelEquivalentKinds} are valid
        and we infer the goal by applying the lemma.
    \end{proof}
\end{theorem}



As a next step we prove the fact that the model set of a kind is not affected by
a welltyped substitution.


\begin{theorem}
    \label{ModelSubstitutionSame}
    \emph{A type correct variable substitution does not affect the model set of
    a kind}.
    $$
    \rulev{
        K \in \Kinds
        \\
        \Gamma,x^A \vdash K: \Any
        \\
        \Gamma \vdash a : A
    }
    {
        \nu(K) = \nu(K[x:=a])
    }
    $$
    \begin{proof}

        By induction on $K \in \Kinds$.

        \def\goal#1#2{
            \forall #1.
            \left[
                \rulev{
                    \Gamma,x^A,#1 \vdash #2: \Any
                }
                {
                    \nu(#2) = \nu((#2)[x:=a])
                }
            \right]
        }

        \begin{enumerate}
        \item $s \in \Kinds$: Trivial

        \item
            $$
            \begin{array}{l|l}
                B \notin \Kinds
                \\
                K \in \Kinds
                &
                \goal {\Delta'} K
                \\
                \hline
                \Pi y^B. K \in \Kinds
                &
                \goal \Delta {\Pi y^B. K}
            \end{array}
            $$

            We assume $\Gamma,x^A,\Delta \vdash \Pi y^B.K : \Any$ and derive the
            goal $\nu(\Pi y^B. K) = \nu((\Pi y^B.K)[x:=a])$.

            From the generation lemma~\ref{GenerationLemmata} for products we
            postulate the existence of $s_B$ and $s_K$ such that
            $$
            \begin{array}{lll}
                \Gamma,x^A,\Delta &\vdash& B: s_B
                \\
                \Gamma,x^A,\Delta,y^B &\vdash& K : s_K
                \\
                s_K \betaeq \Any
            \end{array}
            $$
            This implies $s_B = \Prop$ (because $B\notin \Kinds$ and
            corollary~\ref{NotKindImpliesProp}) and $s_K = \Any$.

            Applying the substitution lemma~\ref{SubstitutionLemma} we get
            $$
            \Gamma,x^A,\Delta[x:=a] \vdash B[x:=a] : \Prop
            $$
            which by theorem~\ref{KindImpliesTypeAny} implies $B[x:=a] \notin
            \Kinds$.

            We use $\Delta' = \Delta,y^B$ and derive $\nu(K) = \nu(K[x:=a])$
            from the induction hypothesis.

            Therefore by the equalities
            $$
            \begin{array}{lll}
                \nu(\Pi y^B.K)
                &=& \nu(K)
                \\
                &=& \nu(K[x:=a])
                \\
                &=& \nu(\Pi y^{B[x:=a]}. K[x:=a])
                \\
                &=& \nu((\Pi y^B . K)[x:=a])
            \end{array}
            $$
            we derive the desired goal.

        \item
            $$
            \begin{array}{l|l}
                B \in \Kinds
                &
                \goal \Delta B
                \\
                K \in \Kinds
                &
                \goal {\Delta'} K
                \\
                \hline
                \Pi y^B. K \in \Kinds
                &
                \goal \Delta {\Pi y^B. K}
            \end{array}
            $$
            We assume $\Gamma,x^A,\Delta \vdash \Pi y^B.K : \Any$ and derive the
            goal $\nu(\Pi y^B. K) = \nu((\Pi y^B.K)[x:=a])$.

            From the generation lemma~\ref{GenerationLemmata} for products we
            postulate the existence of $s_B$ and $s_K$ such that
            $$
            \begin{array}{lll}
                \Gamma,x^A,\Delta &\vdash& B: s_B
                \\
                \Gamma,x^A,\Delta,y^B &\vdash& K : s_K
                \\
                s_K \betaeq \Any
            \end{array}
            $$
            This implies $s_B = \Any$ and $s_K = \Any$ because of $B,K \in
            \Kinds$ and therorem~\ref{KindImpliesTypeAny}.

            By using $\Delta' = \Delta,y^B$ the preconditions of both induction
            hypotheses are satisfied and we derive the facts
            $$
            \begin{array}{lll}
                \nu(B) &=& \nu(B[x:=a])
                \\
                \nu(K) &=& \nu(K[x:=a])
            \end{array}
            $$

            Therefore by the equalities
            $$
            \begin{array}{lll}
                \nu(\Pi y^B.K)
                &=& \nu(B) \to \nu(K)
                \\
                &=& \nu(B[x:=a]) \to \nu(K[x:=a])
                \\
                &=& \nu(\Pi y^{B[x:=a]}. K[x:=a])
                \\
                &=& \nu((\Pi y^B . K)[x:=a])
            \end{array}
            $$
            we derive the desired goal.
        \end{enumerate}
    \end{proof}
\end{theorem}





\subsection{Context Interpretation}

\begin{definition}
    \emph{Context Interpretation}:
    We call $\xi = [x_1^{M_1}, x_2^{M_2}, \ldots, x_n^{M_n}]$ an interpretation
    of a context $\Gamma$ if and only if it satisfies the relation $\xi \vDash
    \Gamma$ defined by the rules
    \begin{enumerate}
    \item Empty context
        $$
        [] \vDash []
        $$
    \item Term variable
        $$
        \rulev{
            \xi \vDash \Gamma
            \\
            \Gamma \vdash A : \Prop
            \\
            x \notin \Gamma
        }
        {
            \xi \vDash \Gamma,x^A
        }
        $$
    \item Type (function) variable
        $$
        \rulev{
            \xi \vDash \Gamma
            \\
            \Gamma \vdash K: \Any
            \\
            F \notin \Gamma
            \\
            M \in \nu(K)
        }
        {
            \xi,F^M \vDash \Gamma,F^K
        }
        $$
    \end{enumerate}
\end{definition}







\subsection{Type Interpretation}

The goal of this section is to find a function which maps any welltyped type
function $F$
for all types $K$ it can have in a certain context $\Gamma$ into a value
$\typeinter{F}{\xi\Gamma} \in \nu (K)$ or $\Any$ into
$\typeinter{\Any}{\xi\Gamma} \in \SAT$. The function is based on a context
interpretation $\xi$ which assigns to each type variable $x$ of type $A$ (which
is a kind) in the context a model which is an element of $\nu(A)$.

\begin{definition}
    \label{SpecificationTypeInterpretation}
    \emph{The type interpretation function $\typeinter{F}{\xi\Gamma}$ must
    satisfy the following specification}.
    $$
    \rulev{
        \xi \vDash \Gamma
        \\
        \Gamma \vdash F : K
        \\
        K \in \Kinds
    }
    {
        \typeinter{F}{\xi\Gamma} \in \nu(K)
    }
    \land
    \ruleh{
        \xi \vDash \Gamma
    }
    {
        \typeinter{\Any}{\xi\Gamma} \in \SAT
    }
    $$

    I.e. it has the preconditions
    \begin{itemize}
    \item The context $\Gamma$ has to be valid and $\xi$ is a context
        interpretation (i.e. $\xi \vdash \Gamma$).

    \item $F$ is either $\Any$ or it is a welltyped type function (i.e. there
        exist some type $K \in \Kinds$ such that $\Gamma \vdash t : K$ is
            valid).
    \end{itemize}

    In the case that $F$ is a welltyped type function, then the
    typeinterpretation has to be an element of $\nu(T)$ for all possible types
    of $F$. If $F$ is $\Any$
    then the typeinterpretation must be a saturated set.
\end{definition}


\begin{definition}
    \label{DefinitionTypeInterpretation}
    \emph{The type interpretation function $\typeinter{F}{\xi\Gamma}$ is defined
    recursively on the structure of $F$}.

    Note that by theorem~\ref{ModelAllTypesSame} we can choose any type of a
    welltyped type function to prove the satisfaction of the specification.

    \begin{enumerate}
    \item Sort:
        $$
        \typeinter{s}{\xi\Gamma} := \SN
        $$
        This definition satisfies the
            specification~\ref{SpecificationTypeInterpretation}. There are two
            cases possible.

            If $s = \Prop$ then it is welltyped and its type is $\Any$ and we
            have $\SN \in \SAT$ and $\SAT = \nu(\Any)$.

            If $s = \Any$ then the specification is trivially satisfied.

    \item Variable:
        $$
        \typeinter{x}{\xi\Gamma} := M
        $$
        where $x^{M} \in \xi$.

        The precondition $\xi \vDash \Gamma$ is possible only if there is a type
        $A$ such that $x^A \in \Gamma$ and $\Gamma \vdash A: \Any$ is valid.
        By the start lemma~\ref{StartLemma} $\Gamma \vdash x : A$ is valid
        and since $\xi$ is a valid context interpretation for the context
        $\Gamma$ we get $M \in \nu(A)$.


    \item Product:
        $$
            \typeinter{\Pi x^A.B}{\xi\Gamma}
            :=
            \typeinter{A}{\xi\Gamma} \tolambda I_B
        $$
        where
            $$
            I_B :=
            \left\{
            \begin{array}{ll}
                \typeinter{B}{\xi(\Gamma,x^A)}
                & \text{if } A \notin \Kinds
                \\
                \bigcap_{M\in\nu(A)}
                \typeinter{B}{(\xi,x^M)(\Gamma,x^A)}
                & \text{if }  A \in \Kinds
            \end{array}
            \right.
        $$

        Since a product is not $\Any$ we have to prove the left part of the
            specification
        $$
        \rulev{
            \xi \vDash \Gamma
            \\
            \Gamma \vdash \Pi x^A.B : K
            \\
            K \in \Kinds
        }
        {
            \typeinter{A}{\xi\Gamma} \tolambda I_B \in \nu(K)
        }
        $$
        where $I_B$ is the type interpretation of $B$ (see above).

        From the generation lemma~\ref{GenerationLemmata} for products we
        postulate the existence of the sorts $s_A$ and $s_B$ such that
        $\Gamma\vdash A: s_A$ and $\Gamma,x^A\vdash B: s_B$ are valid and
        $s_B$ is beta equivalent to $K$.

        Since products cannot be beta equivalent to sorts
        by~\ref{BinderNotEquivalentSortVariable} $K$ must be a sort and the only
        sort beta equivalent to $s_B$ is $s_B$. Therefore we have $K = s_B$ and
        $\nu(K) = \SAT$.

        In order to prove that $\typeinter{A}{\xi\Gamma} \tolambda I_B$ is a
        saturated set by theorem~\ref{LambdaSpaceSaturated} it is sufficient to
        prove that $\typeinter{A}{\xi\Gamma}$ and $I_B$ are saturated sets.
        \begin{enumerate}
        \item $\typeinter{A}{\xi\Gamma} \in \SAT$: The preconditions $\xi \vDash
            \Gamma$, $\Gamma \vdash A : s_A$ and $s_A \in \Kinds$ of the
                typeinterpretation function are satisfied. Therefore we can
                conclude the goal.
        \item $I_B \in \SAT$: We have to distinguish two cases:
            \begin{enumerate}
            \item $A \notin \Kinds$:
                In that case we have $\xi \vDash \Gamma,x^A$ i.e. the
                    preconditions for the typeinterpretation function are
                    satisfied and we get $\typeinter{B}{\xi(\Gamma,x^A)} \in
                    \SAT$.

                \item $A \in \Kinds$: By
                    theorem~\ref{SaturatedSetsIntersection} it is sufficient to
                    prove $\typeinter{B}{(\xi,x^M)(\Gamma,x^A)} \in \SAT$ for
                    all $M \in \nu(A)$.

                    Assume $M \in \nu(A)$. Then $\xi,x^M \vDash \Gamma,x^A$ i.e.
                    the preconditions of the typeinterpretation function are
                    satisfied and we infer the goal.
            \end{enumerate}
        \end{enumerate}

    \item Abstraction:
        $$
        \typeinter{\lambda x^A. e}{\xi\Gamma} :=
            \left\{
            \begin{array}{ll}
                \typeinter{e}{\xi(\Gamma,x^A)}
                & \text{if } A \notin \Kinds
                \\
                M \mapsto \typeinter{e}{(\xi,x^M)(\Gamma,x^A)}
                & \text{if }  A \in \Kinds, M \in \nu(A)
            \end{array}
            \right.
        $$

        Since an abstraction is not $\Any$ we have to prove the left part of the
        specification
        $$
        \rulev{
            \xi \vDash \Gamma
            \\
            \Gamma \vdash \lambda x^A. e : K
            \\
            K \in \Kinds
        }
        {
            \typeinter{\lambda x^A.e}{\xi\Gamma} \in \nu(K)
        }
        $$
        By the generation lemma~\ref{GenerationLemmata} for abstractions we
        postulate the existence of $B$ and $s$ such that $\Gamma \vdash \Pi x^A
        . B : s$, $\Gamma,x^A \vdash e : B$ and $K \betaeq \Pi x^A. B$ are
        satisfied.

        By the lemma~\ref{ModelAllTypesSame} we get $\Pi x^A.B \in \Kinds$
        and $\nu(K) = \nu(\Pi x^A. B)$.

        In order to prove the goal $\typeinter{\lambda x^A.e}{\xi\Gamma} \in
        \nu(\Pi x^A.B)$ we distinguish two cases:
        \begin{enumerate}
        \item $A \notin \Kinds$:
            In that case the goal is
            $$
                \typeinter{e}{\xi(\Gamma,x^A)} \in \nu(B)
            $$
            Since $A$ is not a kind we have $\xi \vDash \Gamma,x^A$
            and therefore the preconditions for
            $\typeinter{e}{\xi(\Gamma,x^A)}$ are satisfied and the
            specification of the typeinterpretation function guarantees the
            goal.

        \item $A \in \Kinds$:
            In that case the goal is
            $$
                M \mapsto \typeinter{e}{(\xi,x^M)(\Gamma,x^A)}
                \in \nu(A) \to \nu(B)
            $$
                where $M \in \nu(A)$. The function argument is in the correct
                domain. Because of $\xi,x^M \vDash \Gamma,x^A$ the preconditions
                of $\typeinter{e}{(\xi,x^M)(\Gamma,x^A)}$ are satisfied and the
                specification of the typeinterpretation function guarantees that
                the function maps its argument to a value in the correct range.
        \end{enumerate}

    \item Application:
        $$
        \typeinter{Fa}{\xi\Gamma} :=
        \left\{
        \begin{array}{ll}
            \typeinter{F}{\xi\Gamma}
            &
            \text{if $\Gamma \vdash a : A$ for some $A \notin \Kinds$}
            \\
            \typeinter{F}{\xi\Gamma}(
                \typeinter{a}{\xi\Gamma}
            )
            &
            \text{if $\Gamma \vdash a : A$ for some $A \in \Kinds$}
        \end{array}
        \right.
        $$
        where $\Gamma \vdash a : A$ for some $A$.

        Since an application is not $\Any$ we have to prove the left part of the
        specification
        $$
        \rulev{
            \xi \vDash \Gamma
            \\
            \Gamma \vdash F a: K
            \\
            K \in \Kinds
        }
        {
            \typeinter{F a}{\xi\Gamma} \in \nu(K)
        }
        $$
        \begin{itemize}
        \item
            By the generation lemma~\ref{GenerationLemmata} for applications we
            postulate the existence of $A$ and $B$ such that
            $$
            \begin{array}{l}
                \Gamma \vdash F : \Pi x^A. B
                \\
                \Gamma \vdash a : A
                \\
                K \betaeq B[x:=a]
            \end{array}
            $$
            are valid. By the lemma~\ref{ModelAllTypesSame} we get $B[x:=a] \in
            \Kinds$ and $\nu(K) = \nu(B[x:=a])$ i.e. we have to prove the goal
            $$\typeinter{F a}{\xi\Gamma} \in \nu(B[x:=a])
            $$

        \item
            Using the type of types lemma~\ref{TypeOfTypes} we can derive the
            existence of some sort $s$ such that $\Gamma \vdash \Pi x^A.B :
            s$ is valid. This implies by the generation
            lemma~\ref{GenerationLemmata} for products the existence of the
            sorts $s_A$ and $s_B$ such that
            $\Gamma \vdash A : s_A$, $\Gamma,x^A \vdash B : s_B$ and $s \betaeq
            s_B$ are valid (i.e. $s = s_B$). Furthermore by the substitution
            theorem~\ref{SubstitutionLemma} we get $\Gamma \vdash B[x:=a] : s$.

            This implies that $K$ and $B[x:=a]$ are welltyped and therefore
            cannot be $\Any$. Since $K$ is a kind, $s = \Any$ must be valid.

            I.e. we get
            $$
            \begin{array}{lll}
                \Gamma &\vdash& A : s_A
                \\
                \Gamma,x^A &\vdash& B: \Any
                \\
                \Gamma &\vdash& \Pi x^A. B : \Any
            \end{array}
            $$

            Because of theorem~\ref{TypeAnyImpliesKind} we have
            $B \in \Kinds$ and $\Pi x^A . B \in \Kinds$
            i.e. $F$ is a type function.

        \item
            Since $F$ is a type function, the preconditions for the type
                interpretation are satisfied and we get
            $$
                \typeinter{F}{\xi\Gamma} \in \nu(\Pi x^A.B)
            $$


        \item
            We distinguish two cases
            \begin{enumerate}
            \item $A \notin \Kinds$: In that case we have to prove the goal
                $$
                    \typeinter{F}{\xi\Gamma} \in \nu(B[x:=a])
                $$

                We prove the goal by using the equivalence
                $$
                \begin{array}{llll}
                    \nu(B[x:=a]
                    &=& \nu(B)
                    &\text{\ref{ModelSubstitutionSame}}
                    \\
                    &=& \nu(\Pi x^A.B)
                    &\text{definition of $\nu$}
                \end{array}
                $$

            \item $A \in \Kinds$: In that case the  preconditions of the type
                interpretation function for $a$ are satisfied and we get
                $$
                    \typeinter{a}{\xi\Gamma} \in \nu(A)
                $$
                Furthermore we have $\typeinter{F}{\xi\Gamma} \in \nu(A) \to
                    \nu(B)$ and therefore
                    $\typeinter{F}{\xi\Gamma}(\typeinter{a}{\xi\Gamma}$ is a
                    valid function application with
                $$
                    \typeinter{F}{\xi\Gamma}(\typeinter{a}{\xi\Gamma})
                    \in \nu(B)
                $$
                Since typesafe substitution does not change the model we get by
                    using~\ref{ModelSubstitutionSame} $\nu(B) = \nu(B[x:=a])$
                    which proves the goal
                $$
                    \typeinter{F}{\xi\Gamma)}(\typeinter{a}{\xi\Gamma})
                    \in \nu(B[x:=a])
                $$
            \end{enumerate}
        \end{itemize}
    \end{enumerate}
\end{definition}



\begin{theorem}
    \label{TypeInterpretationSubstitution}
    \emph{??}
    $$
    \rulev{
        \xi \vDash \Gamma
        \\
        \Gamma \vdash a: A
        \\
        \Gamma,x^A \vdash B : s
    }
    {
        \typeinter{B[x:=a]}{\xi\Gamma}
        =
        \left\{
        \begin{array}{ll}
            \typeinter{B}{\xi(\Gamma,x^A)}
            &
            A \notin \Kinds
            \\
            \typeinter{B}{(\xi,x^M)(\Gamma,x^A)}
            &
            A \in \Kinds, M = \typeinter{a}{\xi\Gamma}
        \end{array}
        \right.
    }
    $$
    \begin{proof}
        MISSING
    \end{proof}
\end{theorem}



\begin{theorem}
    \label{TypeInterpretationEquivalence}
    \emph{Equivalent type functions have the same type interpretation}
    $$
    \rulev{
        \xi \vDash \Gamma
        \\
        \Gamma \vdash F : K
        \\
        \Gamma \vdash G : s
        \\
        K \in \Kinds
        \\
        F \betaeq G
    }
    {
        \typeinter{F}{\xi\Gamma} = \typeinter{G}{\xi\Gamma}
    }
    $$
    \begin{proof}
        MISSING
    \end{proof}
\end{theorem}




\subsection{Term Interpretation}

$\terminter{t}{\rho}$







\subsection{Context Model}

\begin{definition}
    \label{DefinitionContextModel}
    \emph{Context Model}: We call a term interpretation $\rho$ and a context
    interpretation $\xi$ (i.e. $\xi \vDash \Gamma$) a model of a context which
    we write
    $$
        \rho\xi \vDash \Gamma
    $$
    when
    $\rho$ and $\Gamma$ have the form
    $$
    \begin{array}{lll}
        \Gamma &=& [x_1^{A_1}, \ldots, x_n^{A_n}]
        \\
        \rho   &=& [x_1^{t_1}, \ldots, x_n^{t_n}]
    \end{array}
    $$
    where $t_i \in \typeinter{A_i}{\xi\Gamma}$ for all $i \in \set{1,\ldots,n}$.
\end{definition}




\subsection{Soundness Theorem}

\begin{theorem}
    $$
    \rulev{
        \Gamma \vdash t : T
        \\
        \rho\xi \vDash \Gamma
    }
    {
        \terminter{t}{\rho} \in \typeinter{T}{\xi\Gamma}
    }
    $$
    \def\goal#1#2#3#4#5{
        \forall #1 #2.
        \left[
        \ruleh{
            #1#2 \vDash #3
        }
        {
            \terminter{#4}{#1} \in \typeinter{#5}{#2#3}
        }
        \right]
    }

    \begin{proof}
        By induction on $\Gamma \vdash t : T$.

        \begin{enumerate}
        \item Sort:

            MISSING
        \item Variable:

            MISSING
        \item Product:
            $$
            \begin{array}{l|l}
                \Gamma \vdash A : s_A
                &\goal \rho \xi \Gamma A {s_A}
                \\
                \Gamma, x^A \vdash B: s_B
                & \goal {\rho'} {\xi'} {\Gamma,x^A} B {s_B}
                \\
                \hline
                \Gamma \vdash \Pi x^A. B : s_B
                &
                \goal \rho \xi \Gamma {\Pi x^A.B} {s_B}
            \end{array}
            $$

            Assume $\rho\xi \vDash \Gamma$. We have to prove the goal
                $\terminter{\Pi x^A.B}{\rho} \in \SN$ which by definition of the
                terminterpretation is equivalent to the two goals
                $\terminter{A}\rho \in \SN$
                and $\terminter{B}{\rho,x^x} \in SN$.

            From the first induction hypothesis we can prove
            the first subgoal $\terminter{A}{\rho} \in \SN$.

            In order to prove the second subgoal we distinguish two cases:
            \begin{enumerate}
            \item $A \notin \Kinds$:
                We choose $\rho' = \rho,x^x$ and $\xi' = \xi$.
                Since $A$ is not a kind, $\xi' \vDash \Gamma,x^A$ is
                certainly valid. Therefore the second induction hypothesis
                    proves the second subgoal.

            \item $A \in \Kinds$:
                We choose $\rho' = \rho, x^x$ and $\xi' = \xi,x^{\nu^c(A)}$
                    which makes $\rho'\xi' \vDash \Gamma,x^A$ valid. We use the
                    second induction hypothesis to prove the second subgoal.
            \end{enumerate}

        \item Abstraction:
            $$
            \begin{array}{l|l}
                \Gamma \vdash \Pi x^A.B : s
                \\
                \Gamma, x^A \vdash e : B
                &
                \goal {\rho'} {\xi'} {(\Gamma,x^A)} e B
                \\
                \hline
                \Gamma \vdash \lambda x^A.e : \Pi x^A. B
                &
                \goal \rho \xi \Gamma {\lambda x^A.e} {\Pi x^A.B}
            \end{array}
            $$

            Assume $\rho\xi \vDash \Gamma$. We have to prove the goal
            $\terminter{\lambda x^A.e}{\rho} \in [I_A \tolambda I_B]$ where
            $$
            \begin{array}{lll}
                I_A &=& \typeinter{A}{\xi\Gamma}
                \\
                I_B &=&
                \begin{cases}
                    \typeinter{B}{\xi\Gamma}
                    &
                    A \notin \Kinds
                    \\
                    \bigcap_{M \in \nu(A)}
                    \typeinter{B}{(\xi,x^M)(\Gamma,x^A)}
                    &
                    A \in \Kinds
                \end{cases}
            \end{array}
            $$
            Remember that $I_A \tolambda I_B$ is defined as the set of terms
            $\set{f | \forall a. a \in I_A \imp f a \in I_B}$.

            In order to prove the goal we assume $a \in I_A$ and have to prove
            $\terminter{\lambda x^A. e}{\rho} a \in I_B$.

            From the definition of the term interpretation we get the following
            $$
            \begin{array}{lll}
                \terminter{\lambda x^A. e}{\rho} a
                &=&
                (\lambda x^{\terminter{A}{\rho}}. \terminter{e}{\rho,x^x}) a
                \\
                &\keyreduce&
                \terminter{e}{\rho,x^x} [x:=a]
                \\
                &=&
                \terminter{e}{\rho,x^a}
            \end{array}
            $$

            MISSING

        \item Application:

            MISSING
        \item Weaken:

            MISSING
        \item Type equivalence:
            $$
            \begin{array}{l|l}
                \Gamma \vdash t : T
                &
                \goal \rho \xi \Gamma t T
                \\
                \Gamma \vdash U : s
                \\
                T \betaeq U
                \\
                \hline
                \Gamma \vdash t : U
                &
                \goal \rho \xi \Gamma t U
            \end{array}
            $$

            MISSING
        \end{enumerate}
    \end{proof}
\end{theorem}
