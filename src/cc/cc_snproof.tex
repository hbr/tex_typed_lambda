\section{Proof of Strong Normalization}

Barendregt~1993~\cite{barendregt1993}, Geuvers~1994~\cite{geuvers1994}.




\subsection{Model Set}

\begin{definition}
    \emph{Model Set} For $T \in \Kinds \cup \set{\Any}$ we define the model set
    $\nu(T)$ by
    $$
        \nu(T) :=
        \left \{
        \begin{array}{llll}
            \nu(s) &:=& \SAT & s \text{ is a sort}
            \\
            \nu(\Pi x^A. B) &:=& \nu(A)\to \nu(B) & A,B \in \Kinds
            \\
            \nu(\Pi x^A. B) &:=& \nu(B) & A \notin \Kinds \land B \in \Kinds
        \end{array}
        \right.
    $$
\end{definition}



\begin{definition}
    \emph{Canonical Model} For $T \in \Kinds \cup \set{\Any}$ we define the
    canonical model $\nu^c(T)$ by
    $$
        \nu(T) :=
        \left \{
        \begin{array}{llll}
            \nu^c(s) &:=& \SN & s \text{ is a sort}
            \\
            \nu^c(\Pi x^A. B)
            &:=&
            \bullet \mapsto \nu^c(B) & A,B \in \Kinds
            \\
            \nu^c(\Pi x^A. B)
            &:=&
            \nu^c(B) & A \notin \Kinds \land B \in \Kinds
        \end{array}
        \right.
    $$
\end{definition}


The following property of $\nu$ is convenient:
$$
    \rulev{
        \Gamma \vdash t: T
        \\
        \Gamma \vdash t: U
    }
    {
        \nu(T) = \nu(U)
    }
$$


In order to prove the property we prove first a similar lemma for kinds.

\begin{lemma}
    \label{ModelEquivalentKinds}
    \emph{The model set of a kind is the same as the model set of any beta
    equivalent type}.
    $$
    \rulev{
        K \in \Kinds
        \\
        \Gamma \vdash K: \Any
        \\
        \Gamma \vdash T: s_T
        \\
        K \betaeq T
    }
    {
        \nu(K) = \nu(T)
    }
    $$

    \begin{proof}
        By induction on the structure of $K$.

        General observation: Since $K$ and $T$ are betaequivalent we get from
        theorem~\ref{TypeUniqueness3} $s_T = \Any$ and
        by~\ref{TypeAnyImpliesKind} that $T$ is a syntactical kind. This is
        valid for all induction hypotheses.

        For the induction proof we distiguish three cases:
        \begin{enumerate}
        \item $K = \Prop$: Using the general observation above we conclude that
            $T$ is a syntactical kind. $\Prop$ is the only possible syntactical
                kind beta equivalent to $\Prop$. Therefore $T = \Prop$ which
                implies the goal $\nu(T) = \nu(\Prop)$ trivially.

        \item $K = \Pi x^{K_1}. K_2$:
            Since $T$ is a syntactical kind betaequivalent to $K$ it must have
                the form of a product ($\Prop$ not possible). I.e. $T = \Pi
                x^A.B$ for some types $A$ and $B$.

            By theorem~\ref{EquivalentBinders} we get the equivalences $K_1
                \betaeq A$ and $K_2 \betaeq B$.

            From both induction hypotheses we conclude $\nu(K_1) = \nu(A)$ and
            $\nu(K_2) = \nu(B)$.


        \item $K = \Pi x^A . K_2$ (where $A \notin \Kinds$):
            By the same reasoning as above we get $T = \Pi x^{A_T}.B$ for some
            types $A_T$ and $B$ with $A \betaeq A_T$ and $ K_2 \betaeq B$.

                From the induction hypothesis we get $\nu(K_2) = \nu(B)$.

            Since both $A$ and $A_T$ are valid types and $A$ is not a
                syntactical kind, we conclude by using~\ref{TypeUniqueness3}
                that $A_T$ cannot be a syntactical kind either.

            Therefore we get
            $$
                \nu(K) = \nu(K_2) = \nu(B) = \nu(T).
            $$
        \end{enumerate}
    \end{proof}
\end{lemma}


\begin{theorem}
    \label{ModelAllTypesSame}
    \emph{If a term $t$ is welltyped, then the model set of all its possible
    types is the same}.

    NEEDS REWORK!!!
    $$
    \rulev{
        \Gamma \vdash t: T
        \\
        \Gamma \vdash t: U
    }
    {
        \nu(T) = \nu(U)
    }
    $$

    \begin{proof}
        From~\ref{TypeUniqueness1} we infer $T \betaeq U$ and
        from~\ref{TypeUniqueness3} we infer that either both are $\Any$ or both
        are types of the same sort.

        If both are $\Any$, then $\nu(T) = \nu(U)$ is trivial.

        Assume both are types of the same sort i.e. we have $\Gamma \vdash T: s$
        and $\Gamma \vdash U: s$. We have to distinguish $s = \Any$ and $s =
        \Prop$.

        \begin{enumerate}
        \item $s = \Any$:
            In that case the assumptions of lemma~\ref{ModelEquivalentKinds} are
                valid and we infer the goal by applying the lemma.

        \item $s = \Prop$:
            In that case neither $T$ nor $U$ are kinds and therefore by
                definition $\nu(T) = \nu(U) = \SAT$ is valid.
        \end{enumerate}
    \end{proof}
\end{theorem}




\subsection{Context Interpretation}

\begin{definition}
    \emph{Context Interpretation}:
    We call $\xi = [x_1^{M_1}, x_2^{M_2}, \ldots, x_n^{M_n}]$ an interpretation
    of a context $\Gamma$ which we write
    $$
        \xi \vDash \Gamma
    $$
    if and only if $\Gamma = [x_1^{A_1}, x_2^{A_2}, \ldots, x_n^{A_n}]$ is a
    valid context and $M_i \in \nu(A_i)$ is valid for all $i \in \set{1, 2,
    \ldots, n}$.
\end{definition}



\subsection{Type Interpretation}

The goal of this section is to find a function which maps any welltyped term $t$
for all types $T$ it can have in a certain context $\Gamma$ into a value
$\typeinter{t}{\xi\Gamma} \in \nu (T)$ or $\Any$ into
$\typeinter{\Any}{\xi\Gamma} \in \SAT$. The function is based on a context
interpretation $\xi$ which assigns to each variable $x$ of type $A$ in the
context a model which is an element of $\nu(A)$.

\begin{definition}
    \label{SpecificationTypeInterpretation}
    \emph{The type interpretation function $\typeinter{t}{\xi\Gamma}$ must
    satisfy the following specification}.
    $$
    \rulev{
        \xi \vDash \Gamma
        \\
        \Gamma \vdash t : T
    }
    {
        \typeinter{t}{\xi\Gamma} \in \nu(T)
    }
    \land
    \ruleh{
        \xi \vDash \Gamma
    }
    {
        \typeinter{\Any}{\xi\Gamma} \in \SAT
    }
    $$

    I.e. it has the preconditions
    \begin{itemize}
    \item The context $\Gamma$ has to be valid and $\xi$ is a context
        interpretation (i.e. $\xi \vdash \Gamma$.

    \item $t$ is either $\Any$ or it is a welltyped constructor (i.e. there
        exist some type $T \in \Kinds \cap \set{\Any}$ such that $\Gamma \vdash
            t : T$ is
    valid.
    \end{itemize}

    In case that $t$ is a welltyped constructor, then the typeinterpretation has
    to be an element of $\nu(T)$ for all possible types of $t$. If $t$ is $\Any$
    then the typeinterpretation must be a saturated set.
\end{definition}


\begin{definition}
    \label{DefinitionTypeInterpretation}
    \emph{The type interpretation function $\typeinter{t}{\xi\Gamma}$ is defined
    recursively on the structure of $t$}.

    Note that by theorem~\ref{ModelAllTypesSame} we can choose any type of a
    welltyped term to prove the satisfaction of the specification.

    \begin{enumerate}
    \item Sort:
        $$
        \typeinter{s}{\xi\Gamma} := \SN
        $$
        This definition satisfies the
            specification~\ref{SpecificationTypeInterpretation}. There are two
            cases possible.

            If $s = \Prop$ then it is welltyped and its type is $\Any$ and we
            have $\SN \in \SAT$ and $\SAT = \nu(\Any)$.

            If $s = \Any$ then the specification is trivially satisfied.

    \item Variable:
        $$
        \typeinter{x}{\xi\Gamma} := M
        $$
        where $x^{M} \in \xi$.

        The precondition $\xi \vDash \Gamma$ is possible only if there is a type
            $A$ such that $x^A \in \Gamma$ and $\Gamma \vdash A: \Any$ is valid. By the start
            lemma~\ref{StartLemma} $\Gamma \vdash x : A$ is valid and since
            $\xi$ is a valid context interpretation for the context $\Gamma$ we
            get $M \in \nu(A)$.


    \item Product:
        $$
            \typeinter{\Pi x^A.B}{\xi\Gamma}
            :=
            \left\{
            \begin{array}{ll}
                \typeinter{A}{\xi\Gamma}
                \tolambda
                \typeinter{B}{(\xi,x^\SN)(\Gamma,x^A)}
                & \text{if } A \notin \Kinds
                \\
                \typeinter{A}{\xi\Gamma}
                \tolambda
                \bigcap_{M\in\nu(A)}
                \typeinter{B}{(\xi,x^M)(\Gamma,x^A)}
                & \text{if }  A \in \Kinds
            \end{array}
            \right.
        $$

        By the precondition of the type interpretation function we can assume
            that $\Gamma$ is a valid context, $\xi$ is a context
            interpretation i.e. $\xi \vDash \Gamma$. Since a product is not
            $\Any$ we have to prove the desired property from the specification
            only for welltyped terms $\Pi x^A.B$.

            From the generation lemma~\ref{GenerationLemmata} for products we
            postulate the existence of the sorts $s_A$ and $s_B$ such that
            $\Gamma\vdash A: s_A$ and $\Gamma,x^A\vdash B: s_B$ are valid and
            $s_B$ is beta equivalent to any type of $\Pi x^A.B$ in the context
            $\Gamma$.


            MISSING!!!

    \item Abstraction:

            MISSING!!!

    \item Application:

            MISSING!!!
    \end{enumerate}
\end{definition}



\subsection{Term Interpretation}

$\terminter{t}{\rho}$



\subsection{Context Model}

\begin{definition}
    \label{DefinitionContextModel}
    \emph{Context Model}: We call a term interpretation $\rho$ and a context
    interpretation $\xi$ (i.e. $\xi \vDash \Gamma$) a model of a context which
    we write
    $$
        \rho\xi \vDash \Gamma
    $$
    when
    $\rho$ and $\Gamma$ have the form
    $$
    \begin{array}{lll}
        \Gamma &=& [x_1^{A_1}, \ldots, x_n^{A_n}]
        \\
        \rho   &=& [x_1^{t_1}, \ldots, x_n^{t_n}]
    \end{array}
    $$
    where $t_i \in \typeinter{A_i}{\xi\Gamma}$ for all $i \in \set{1,\ldots,n}$.
\end{definition}




\subsection{Soundness Theorem}

\begin{theorem}
    $$
    \rulev{
        \Gamma \vdash t : T
        \\
        \rho\xi \vDash \Gamma
    }
    {
        \terminter{t}{\rho} \in \typeinter{T}{\xi\Gamma}
    }
    $$
    \def\goal#1#2#3#4#5{
        \forall #1 #2.
        \left[
        \ruleh{
            #1#2 \vDash #3
        }
        {
            \terminter{#4}{#1} \in \typeinter{#5}{#2#3}
        }
        \right]
    }

    \begin{proof}
        By induction on $\Gamma \vdash t : T$.

        \begin{enumerate}
        \item Sort:

            MISSING
        \item Variable:

            MISSING
        \item Product:
            $$
            \begin{array}{l|l}
                \Gamma \vdash A : s_A
                &\goal \rho \xi \Gamma A {s_A}
                \\
                \Gamma, x^A \vdash B: s_B
                & \goal {\rho'} {\xi'} {\Gamma,x^A} B {s_B}
                \\
                \hline
                \Gamma \vdash \Pi x^A. B : s_B
                &
                \goal \rho \xi \Gamma {\Pi x^A.B} {s_B}
            \end{array}
            $$

            Assume $\rho\xi \vDash \Gamma$. We have to prove the goal
                $\terminter{\Pi x^A.B}{\rho} \in \SN$ which by definition of the
                terminterpretation is equivalent to the two goals
                $\terminter{A}\rho \in \SN$
                and $\terminter{B}{\rho,x^x} \in SN$.

            From the first induction hypothesis we can prove
            the first subgoal $\terminter{A}{\rho} \in \SN$.

            In order to prove the second subgoal we distinguish two cases:
            \begin{enumerate}
            \item $A \notin \Kinds$:
                We choose $\rho' = \rho,x^x$ and $\xi' = \xi$.
                Since $A$ is not a kind, $\xi' \vDash \Gamma,x^A$ is
                certainly valid. Therefore the second induction hypothesis
                    proves the second subgoal.

            \item $A \in \Kinds$:
                We choose $\rho' = \rho, x^x$ and $\xi' = \xi,x^{\nu^c(A)}$
                    which makes $\rho'\xi' \vDash \Gamma,x^A$ valid. We use the
                    second induction hypothesis to prove the second subgoal.
            \end{enumerate}

        \item Abstraction:
            $$
            \begin{array}{l|l}
                \Gamma \vdash \Pi x^A.B : s
                \\
                \Gamma, x^A \vdash e : B
                &
                \goal {\rho'} {\xi'} {(\Gamma,x^A)} e B
                \\
                \hline
                \Gamma \vdash \lambda x^A.e : \Pi x^A. B
                &
                \goal \rho \xi \Gamma {\lambda x^A.e} {\Pi x^A.B}
            \end{array}
            $$

            Assume $\rho\xi \vDash \Gamma$. We have to prove the goal
            $\terminter{\lambda x^A.e}{\rho} \in [I_A \tolambda I_B]$ where
            $$
            \begin{array}{lll}
                I_A &=& \typeinter{A}{\xi\Gamma}
                \\
                I_B &=&
                \begin{cases}
                    \typeinter{B}{\xi\Gamma}
                    &
                    A \notin \Kinds
                    \\
                    \bigcap_{M \in \nu(A)}
                    \typeinter{B}{(\xi,x^M)(\Gamma,x^A)}
                    &
                    A \in \Kinds
                \end{cases}
            \end{array}
            $$
            Remember that $I_A \tolambda I_B$ is defined as the set of terms
            $\set{f | \forall a. a \in I_A \imp f a \in I_B}$.

            In order to prove the goal we assume $a \in I_A$ and have to prove
            $\terminter{\lambda x^A. e}{\rho} a \in I_B$.

            From the definition of the term interpretation we get the following
            $$
            \begin{array}{lll}
                \terminter{\lambda x^A. e}{\rho} a
                &=&
                (\lambda x^{\terminter{A}{\rho}}. \terminter{e}{\rho,x^x}) a
                \\
                &\keyreduce&
                \terminter{e}{\rho,x^x} [x:=a]
                \\
                &=&
                \terminter{e}{\rho,x^a}
            \end{array}
            $$

            MISSING

        \item Application:

            MISSING
        \item Weaken:

            MISSING
        \item Type equivalence:

            MISSING
        \end{enumerate}
    \end{proof}
\end{theorem}
