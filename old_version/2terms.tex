\section{Terms}

\subsection{Basic Definition}

\begin{definition}
  Typed lambda terms are defined by the grammar
  $$
  \begin{array}{llll}
    t &::=& s                 &\text{sort}
    \\
      &\mid& x                &\text{variable}
    \\
      &\mid& t t              &\text{application}
    \\
      &\mid& \lambda x^t.t    &\text{abstraction}
    \\
      &\mid& \Pi x^t.t        &\text{product}
  \end{array}
  $$
  where $s$ ranges over sorts, $x$ ranges over variables and $t$ ranges over
  terms.
\end{definition}

\subsection{Substitution}

\begin{definition}
  The expression $t[x:=a]$ substitutes in the term $t$ all free occurrences of
  the variable $x$ by the term $a$. It is defined by
  $$
  t[x:=a] :=
  \begin{cases}
    s[x:=a] &:= s
    \\
    x[x:=a] &:= a
    \\
    y[x:=a] &:= y \quad\text{for } x \ne y
    \\
    (f \, t)[x:=a] &:= f[x:=a] \, t[x:=a]
    \\
    (\lambda y^B.e)[x:=a] &:= \lambda y^{B[x:=a]}. e[x:=a]
    \quad y \text{ is fresh}
    \\
    (\Pi y^B.C)[x:=a] &:= \Pi y^{B[x:=a]}. C[x:=a]
    \quad y \text{ is fresh}
  \end{cases}
  $$.

  Note that bound variables in lambda abstractions and products must be chosen
  to be fresh i.e. not to occuer outside the binder. If the bound variables
  occur outside the binder, they have to be renamed into fresh variables
  before doing the substitution.
\end{definition}

For every lambda term with subterms, substitution is done on all substerms
separately. Care has to be taken when entering binders in abstractions and
products. Before entering the binder it has to be made sure that the bound
variable is renamed so that it does not occur outside the binder.


\begin{lemma}
  \label{doublesubstitution}
  \emph{Double Substitution}:
  $$
  \ruleh
  {x \neq y
    \\
    x \notin FV(b)
  }
  {t[x:=a][y:=b]\, = \, t[y:=b]\big[x:=a[y:=b]\big]}
  $$
  \begin{proof}
    By induction on the structure of $t$.

    The only interesting case is if $t$ is a variable e.g. $z$. The we have to
    prove
    $$ z[x:=a][y:=b]\, = \, z[y:=b]\big[x:=a[y:=b]\big].$$

    Three cases are possible:
    \begin{enumerate}
    \item
      $z \ne x \land z \ne y$: In this case no substitution takes place and we
      have the trivial equality $z = z$.

    \item $z = x$: Since $x \ne y$ we get $a = a$.

    \item $z = y$: In that case we get $b = b$.
    \end{enumerate}

    If $t$ is an application, an abstraction or a product, the induction
    hypothesis and the definition of substitution are sufficient to prove the
    corresponding goal.
  \end{proof}
\end{lemma}





\subsection{Reduction}

\begin{definition}
  The reduction relation $\reduce$ is defined as the compatible closure
  of
  $$
  (\lambda x^A. e) a \reduce e[x:=a]
  $$
\end{definition}

Compatible closure means that we can reduce a term by reducing one of its
subterms. I.e. we have the following additional rules to define the reduction
relation completely: $\ruleh {a \reduce b}{ac \reduce bc}$,
$\ruleh {b \reduce c}{ab \reduce ac}$,
$\ruleh {A \reduce B} {\lambda x^A.e \reduce \lambda x^B. e}$,
$\ruleh {e \reduce f} {\lambda x^A.e \reduce \lambda x^A. f}$,
$\ruleh {A \reduce B} {\Pi x^A.C \reduce \Pi x^B. C}$,
$\ruleh {B \reduce C} {\Pi x^A.B \reduce \Pi x^A. C}$.



\begin{lemma}
  \label{reductionsubstitution}
  \emph{Reduction Substitution}:
  $$
  \ruleh
  {t \reduce u
  }
  {t[x:=a] \reduce u[x:=a]}
  $$
  \begin{proof} By induction on $t \reduce u$ where we use the abbreviation
    $w' := w[x:=a]$ for any term $w$.

    Let's first analyze the case that $t \reduce u$ is valid because of the
    rule $(\lambda y^B.e) b \reduce e[y:=b]$.
    $$
    \begin{array}{llll}
      ((\lambda y^B.e) b)'
      &=& (\lambda y^{B'}. e') b'
      &\text{definition of substitution}
      \\
      &\reduce & e'[y:=b'] &\text{reduction rule}
      \\
      &=& e[x:=a]\big[[y:=b[x:=a]\big]   &\text{abbreviation}
      \\
      &=& e[y:=b][x:=a]
      &\text{double substitution~\ref{doublesubstitution}}
      \\
      &=& e[y:=b]'  &\text{abbreviation}
    \end{array}
    $$

    Next we analyze the case $\ruleh{b \reduce c}{bd \reduce cd}$.

    $$
    \begin{array}{l|ll}
      b \reduce c    & b' \reduce c' & \text{induction hypo}
      \\ \hline
      bd \reduce cd  & (bd)' \reduce (cd)' & \text{goal}
    \end{array}
    $$
    The goal $(bd)' \reduce (cd)'$ is proved by the derivation
    $$
    \begin{array}{llll}
      (bd)'
      &=& b' d'        &\text{definition of substitution}
      \\
      &\reduce& c' d'  &\text{induction hypo}
      \\
      &=& (cd)'        &\text{definition of substitution}
    \end{array}
    $$

    For the remaining rules where just subterms are substituted the proofs are
    similar. The corresponding induction hypothesis with the definition of
    substitution is sufficient to proof the goal.
  \end{proof}
\end{lemma}


It is easy to see that terms of the form $\lambda x^A.e$, $\Pi x^A.B$ do not
change their structure after reduction.

\begin{lemma} Reduction maintains structure of abstraction and products:
  $$
  \ruleh
  {\lambda x^A. e \reduce t}
  {(\exists B. A \reduce B \land t = \lambda x^B.e)
    \lor
  (\exists f. e \reduce f \land t = \lambda x^A.f)}
  $$

  $$
  \ruleh
  {\Pi x^A. C \reduce T}
  {(\exists B. A \reduce B \land T = \Pi x^B.C)
    \lor
  (\exists D. C \reduce D \land T = \Pi x^A.D)}
  $$

  \begin{proof}
    The theorem is proved by induction on $\lambda x^A. e \reduce t$ and
    $\Pi x^A. C \reduce T$ where in each induction proof only 2 cases are
    syntactically possible.
  \end{proof}
\end{lemma}




\subsection{Syntactic Categories}

\begin{definition}
  We define within the set of terms the set of objects $\Objects$, types
  $\Types$, $\Kinds$ and sorts $\Sorts$ according to the following syntax
  where $s$ ranges over sorts, $x,\alpha$ over variables, $o$ over objects,
  $\tau$ over types and $\kappa$ over kinds
  % 
  $$
  % 
  \begin{array}{llll}
    o &::= & x
    \\
      & | & o \; o
    \\
      & | & o \; \tau & \text{polymorphic function application}
    \\
      & | & \lambda x^\tau. o
    \\
      & | & \lambda \alpha^\kappa. o
      & \text{polymorphic function}
    \\
    
    \tau & ::= & \alpha
    \\
      & | & \tau \; o & \text{dependent type e.g. \code{isZero 5}}
    \\
      & | & \tau \; \tau & \text{polymorphic type application e.g. List}
    \\
      & | & \lambda x^\tau. \tau        & \text{function type}
    \\
      & | & \lambda \alpha^\kappa. \tau &\text{polymorphic type}
    \\
      & | & \Pi x^\tau. \tau
    \\
      & | & \Pi \alpha^\kappa. \tau
    \\
    
    \kappa & ::= & s
    \\
      & | & \kappa \; o                     &\text{application of a kind function}
    \\
      & | & \kappa \; \tau                  &\text{application of a kind function}
    \\
      & | & \lambda x^\tau. \kappa          &\text{function from objects to kinds}
    \\
      & | & \lambda \alpha^\kappa. \kappa   &\text{function from types to kinds}
    \\
      & | & \Pi x^\tau . \kappa
    \\
      & | & \Pi \alpha^\kappa . \kappa
  \end{array}
  $$
\end{definition}
%
Note that $\Objects \subseteq \Types$ and $\Sorts \subseteq
\Kinds$. Furthermore the set of types and kinds are disjoint i.e. $\Types \cap
\Kinds = \emptyset$.

The union of types and kinds do not cover all possible terms. A term of the
form $\tau \, \kappa$ is in neither category. The typing relation will make
sure that all wellformed terms are in one of the syntactic categories.


%%% Local Variables:
%%% mode: latex
%%% TeX-master: "0typed"
%%% End:
