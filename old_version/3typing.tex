\section{Typing}

\subsection{Basic Typing}


\begin{definition}
  Universes: For the type system we use a set of universes $\Sorts$ (called
  sorts as well) and a wellfounded relation $<$ on sorts.
\end{definition}


\begin{definition}
  Wellfounded relation $<$ on terms: We extend the relation $<$ from sorts to
  terms by the rules
  \begin{enumerate}
  \item
    $ \ruleh
    {T < U \quad A \betaeq B \quad A \in \Types \cup \Kinds}
    {\Pi x^A. T < \Pi x^B. U}$

  \item
    $ \ruleh
    {T < U \quad A \betaeq B \quad A \in \Types \cup \Kinds}
    {\lambda x^A. T < \lambda x^B. U}$

  \item
    $ \ruleh
    {T < U \quad a \betaeq b \quad a \notin \Kinds}
    {T \, a  <  U \, b}$
  \end{enumerate}
  The relation $<$ on terms remains wellfounded. There has been nothing added
  which allows for infinitely decreasing chains (or cycles in the $<$ relation).
\end{definition}

\begin{theorem}
  $\ruleh
  {T < U}
  {T,U \in \Kinds}$
  \begin{proof}
    Easy by induction on $T < U$.
  \end{proof}
\end{theorem}

\begin{definition}
  Term preorder: We define a preorder on terms inductively by
  \begin{description}

  \item[reflexive] $ a \le a$

  \item[transitive] $\ruleh{ a \le b \land b < c } {a \le c}$

  \item[$\beta$ equivalence] $ \ruleh {a \reduce b \lor b \reduce a} {a \le b} $
  \end{description}
  %
  Note that $\le$ is not a partial order because it lacks antisymmetry. We can
  only state the lemma
  $$
  \ruleh
  {a \le b \quad b \le a}
  {a \betaeq b}
  $$
\end{definition}



\begin{definition}
  A \emph{context} $\Gamma$ is a finite sequence of variables with their types
  $$
  \Gamma = [x_1: T_1, x_2: T_2, \ldots , x_n: T_n]
  $$
  where no variable occurs more than once. Therefore it makes sense to speak
  about the type of a variable in a context.
\end{definition}

\begin{definition}
  Typing relation: $\Gamma \vdash t: T$ where $\Gamma$ is a context and $t$
  and $T$ are terms.
  \begin{description}

  \item[Axiom] $$\ruleh{s_1 < s_2}{[] \vdash s_1:s_2}$$

  \item[Variable]
    $$\ruleh{\Gamma \vdash A:s\quad x \notin \Gamma}{\Gamma,x:A \vdash x:A}$$


  \item[Product]
    $$\ruleh
    {\Gamma \vdash A:s_1 \quad
      \Gamma,x:A \vdash B:s_2 \quad
      (s_1 = s_2 \lor s_2 \text{ is minimal}) }
    {\Gamma \vdash \Pi x^A . B : s_2}
    $$

  \item[Abstraction]
    $$\ruleh
    {\Gamma \vdash \Pi x^A.B: s \quad
      \Gamma,x:A \vdash e:B
    }
    {\Gamma \vdash \lambda x^A.e : \Pi x^A . B}
    $$

  \item[Application]
    \footnote{$a \in \Objects \cup \Types$ is necessary in order to avoid that
      a variable is substituted by a kind}
   $$\ruleh
    {\Gamma \vdash f: \Pi x^A.B \quad
      \Gamma \vdash a:A
      \quad a \in \Objects \cup \Types
    }
    {\Gamma \vdash f a : B[x:=a]}
    $$

  \item[Weaken]
    $$\ruleh
    {\Gamma \vdash A:s \quad  \Gamma \vdash t:T   \quad x \notin \Gamma}
    {\Gamma,x:A \vdash t:T}$$


  \item[Conversion]
    $$\ruleh
    {\Gamma \vdash t: T \quad
      \Gamma\vdash U: s \quad
      (T \reduce U \lor U \reduce T)
    }
    {\Gamma \vdash t: U}
    $$

  \item[Subtyping]
    $$\ruleh
    {\Gamma \vdash t: T \quad
      \Gamma\vdash U: s \quad
      T < U
    }
    {\Gamma \vdash t: U}
    $$
  \end{description}
\end{definition}

\begin{definition}
  Some common definitions:
  \begin{description}
  \item[Valid context] A context $\Gamma$ is valid if there are two terms
    $t,T$ such that $\Gamma \vdash t : T$ is valid.

  \item[Subject and predicate] In the typing judgment $\Gamma \vdash t : T$
    the term $t$ is in the subject position and the term $T$ is in the
    predicate position.

  \item[Welltyped] A term $T$ is welltyped if there is a contex $\Gamma$ and
    another term $U$ such that $\Gamma \vdash T:U$ or $\Gamma \vdash U:T$ are
    valid typing judgements.

  \item[Minmal type] In the typing judgement $\Gamma \vdash t : T$ the term
    $T$ is called a minimal type of $t$ if there is no other term $U$ with $U
    < T$ and $\Gamma \vdash t :U$.
  \end{description}
\end{definition}

\paragraph{Example: Extra clause in the application rule}
Consider a type system with the sorts $\{\Prop,\Any,\Box\}$ and the
wellfounded relation $\Prop < \Any < \Box$. Then the following derivation is
legal if we ignore the clause $a \in \Objects \cup \Types$ in the application
rule.
%
$$
%
\begin{array}{llll}
  []       & \vdash & \Any : \Box                 & \text{axiom}
  \cr
  [x:\Any] & \vdash & \Any : \Box                 & \text{weaken}
  \cr
  []       & \vdash & \Pi x^{\Any} . \Any : \Box  & \text{product}
  \cr
  [L:\Pi x^{\Any}.\Any]& \vdash & L : \Pi x^{\Any}.\Any  & \text{variable}
  \cr
  [L:\dots, x: \Any]& \vdash & x: \Any            & \text{variable}
  \cr
  [L:\dots, x: \Any]& \vdash & L x: \Any          & \text{application}
  \cr
  [L:\Pi x^{\Any}.\Any]& \vdash & \Pi x^{\Any}. L x: \Box  & \text{producct}
  \cr
  [L:\dots, f: \Pi x^{\Any}. L x]& \vdash & f: \Pi x^{\Any}.L x & \text{variable}
  \cr
  [L:\dots, f: \Pi x^{\Any}. L x]& \vdash & f {\Prop}: L \Prop & \text{application}
\end{array}
$$
%
This is certainly not what we wanted. We introduced the variable $L$ as a
type constructor having type $\Pi x^\Any. \Any$ i.e. mapping a type to a
type. The function $f$ maps any type $T$ to an object of type $L T$. $f$ can
be imagined as the constructor $\text{nil}$ which constructs the empty list of
a certain type.

Because of the construction we can feed the sort $\Prop$ as an argument to
$\text{nil}$ giving a meaningless expression of a meaningless type.


\subsection{Syntactic Categories}

\begin{theorem}
  $$
  \ruleh
  {\Gamma \vdash A : B}
  {(A,B) \in
    (\Objects \times \Types) \,\cup\,
    (\Types   \times \Kinds) \,\cup\,
    (\Kinds   \times \Kinds)
  }
  $$
  \begin{proof} By induction on $\Gamma \vdash A : B$
    \begin{description}

    \item[Axiom] $[] \vdash s_1 : s_2$.

      Since $\Sorts \subseteq \Kinds$ we have $s_1 \in \Kinds$ and $s_2 \in
      \Sorts$.


    \item[Variable] $\ruleh {\Gamma \vdash A:s} {\Gamma,x:A \vdash x: A}$.

      Induction hypothesis: $A$ is either a type or a kind. The variable $x$
      can be either regarded as an object or a type.

    \item[Product]
      $
      \ruleh
      {\Gamma \vdash A: s_1 \quad \Gamma,x:A \vdash B:s}
      {\Gamma \vdash \Pi x^A . B : s}
      $

      From the premises we know that the types of $A$ and $B$ are
      sorts. Therefore the induction hypotheses imply that $A$ and $B$ are
      either types or kinds. If $B$ is a type then the product is a type and
      if $B$ is a kind then the product is a kind which is compatible with $s$
      being either a kind or a sort ($\Sorts \subset \Kinds$).

    \item[Abstraction]
      %
      $\ruleh
      {\Gamma \vdash \Pi x^A.B : s \quad
      \Gamma,x:A \vdash e : B}
      {\Gamma \vdash \lambda x^A.e : \Pi x^A. B}$

      By the first induction hypothesis the product is syntactically either a
      type or a kind. $B$ must have the same syntactic category as the product
      (evident from the definition of the syntactic categories).

      \begin{enumerate}
        \item $B \in \Kinds$: Then $e$ is either a type or a kind by the second
          induction hypotesis. This puts $\lambda x^A.e$ either in $\Types$ or
          $\Kinds$ according to the definitions of the syntactic
          categories. Since $\Pi x^A.B$ is a kind this case is ok.

        \item $B \in \Types$: This puts $e$ in $\Objects$. Therefore $\lambda
          x^A.e$ is an object as well which is consistent with $\Pi x^A.B$
          being a type.
      \end{enumerate}

    \item[Application]
      $ \ruleh
      {\Gamma \vdash f : \Pi x^A.B
        \quad
        \Gamma \vdash a : A
        \quad a \in \Objects \cup \Types}
      {\Gamma \vdash f\; a : B[x:=a]}
      $

      Syntactically a product is either a type or a kind. Therefore $f$ is an
      object or a type. $B$ is of the same category as the product. The
      substitution of the variable $x$ in $B$ by an object or a type does not
      change the syntactic category.

      The application $f \, a$ is syntactically either an object or a type
      depending on the category of $f$ which is determined by the category of
      $B$. Therefore $f \, a$ is in the correct category for the category of
      $B[x:=a]$.

    \item[Weaken] Goal already identical to an induction hypothesis.

    \item[Conversion]
      $
      \ruleh {
        \Gamma \vdash t : T \quad
        \Gamma \vdash U : s \quad
        T \reduce U \lor U \reduce T
      } {
        \Gamma \vdash t : U
      } $

      \begin{enumerate}

      \item $T\reduce U$:

        $T \notin \Sorts$. I.e. $T$ is either a type or a kind.

        The reduction of a type remains a type and the reduction of a kind
        remains a kind. Therefore $U$ is in the same syntactic category as
        $T$.


      \item $U \reduce T$:

        $U$ is a type or a kind and reduction does not change the
        category. Therefore $T$ and $U$ are in the same category.
      \end{enumerate}

    \item[Subtyping]
      $ \ruleh {
        \Gamma \vdash t : T \quad
        \Gamma \vdash U : s \quad
        T < U
      } {
        \Gamma \vdash t : U
      } $

      By definition only kinds of the same arity are comparable. Therefore $T$
      and $U$ have to be kinds of the same arity (or sorts in case the arity
      is 0).

    \end{description}

  \end{proof}
\end{theorem}


\begin{corolary}
  $$
  \ruleh {(x,A) \in \Gamma} {A \in \Types \cup \Kinds}
  $$
  provided that $\Gamma$ is wellformed.
  \begin{proof}
    By weakening we conclude $\Gamma \vdash x:A$ for all $(x,A) \in \Gamma$.

    Since a variable can be only an object or a type $A$ must be either a type
    or a kind by the previous lemma.
  \end{proof}
\end{corolary}


\subsection{Generalized Subtyping}

The conversion and subtyping rule transform the type part one step. We can
generalize both rules to one theorem.
\begin{theorem}
  \label{generalizedsubtyping}
  Generalized subtyping
  $$
  \rulev
  {
    \Gamma \vdash t : T
    \\
    \Gamma \vdash U: s
    \\
    T \le U
  }
  {
    \Gamma \vdash t : U
  }
  $$

  \begin{proof}
    By induction on $T \le U$
  \end{proof}
\end{theorem}


\subsection{Some Lemmas}

Free variables lemma

Transitivity lemma

Arbitrary type expansion


\subsection{Substitution Lemma}


Thinning lemma

\subsection{Minimal Types}

\begin{theorem}
  For every valid term in the subject position there exist minimal types.
  $$
  \ruleh
  %
  {\Gamma \vdash t: T}
  {\exists M.
    \left (
      \Gamma \vdash t : M \land
      M \le T \land
      \forall V.
      \ruleh {\Gamma \vdash t: V}  {\lnot\, V < M}
    \right )
  }
  $$

  \begin{proof}
    By induction on $\Gamma \vdash t: T$
    \begin{description}

    \item[Axiom] Trivial, because $<$ is a wellfounded relation of sorts.

    \item[Variable]
      $\ruleh
      {\Gamma \vdash A:s\quad x \notin \Gamma}
      {\Gamma,x:A \vdash x:A}$

      We use $A$ as the minimal type. Proof by induction on
      $\Gamma,x:A \vdash x:A$.
      \begin{description}

      \item[axiom, application, abstraction, product] Syntactically not
        possible.

      \item[variable] Trivial, because $A = V$ and $A < A$ is a contradiction.

      \item[conversion]
        $\rulev
        {
          \Gamma,x:A \vdash x: T \\
          \Gamma,x:A \vdash U: s \\
          T \reduce U
        }
        {\Gamma,x:A \vdash x: U}$

        Goal $\lnot\, U < T$.  Trivial because $T \reduce U$ implies $T \le U$
        which contradicts $U < T$.


      \item[subtyping]
        $\rulev
        {
          \Gamma,x:A \vdash x: T \\
          \Gamma,x:A \vdash U: s \\
          T < U
        }
        {\Gamma,x:A \vdash x: U}$

        Goal $\lnot\, U < T$.  Trivial, because $T < U$ and $U < T$ are a
        contradiction.
      \end{description}

    \item[Product]
      $
      \ruleh
      {\Gamma \vdash A:s_1 \quad
        \Gamma,x:A \vdash B:s_2 \quad
        (s_1 = s_2 \lor s_2 \text{ is minimal}) }
      {\Gamma \vdash \Pi x^A . B : s_2}      $

      From the induction hypotheses we conclude the existence of a minimal
      $c_2$ such that $\Gamma,x:A \vdash B: c_2$ is valid. This is a minimal
      sort for $\Pi x^A. B$ as well.

    \item[Application]
      $\ruleh
      {\Gamma \vdash f: \Pi x^A.B \quad
        \Gamma \vdash a:A
        \quad a \in \Objects \cup \Types
      }
      {\Gamma \vdash f a : B[x:=a]}
      $

      From the first induction hypothesis we conclude the existence of a
      minimal product as the type of $f$ which must have the form
      $\Pi x^A.B_\mu$ with $B_\mu \le B$ because of the definition of the
      order relation $\le$.
      %
      This implies $B_\mu[x:=a] \le B[x:=a]$. Since $\Pi x^A. B_\mu$ is a
      minimal type for $f$, $B_\mu[x:=a]$ is a minimal type for $f\; a$.

    \item[Abstraction]
      $\ruleh
      {\Gamma \vdash \Pi x^A.B: s \quad
        \Gamma,x:A \vdash e:B
      }
      {\Gamma \vdash \lambda x^A.e : \Pi x^A . B}
      $

      From the second induction hypothesis we conclude the existence of
      $B_\mu$ which is a minimal type for $e$. Therefore $\Pi x^A. B_\mu$ is
      a minimal type for $\lambda x^A.e$.


    \item[Weaken] Trivial, the goal is already included in the induction
      hypothesis.

    \item[Conversion] Trivial because $T \reduce U$ implies $T \le U$ and
      transitivity.

    \item[Subtyping] Trivial because of transitivity.
    \end{description}
  \end{proof}
\end{theorem}








\subsection{Generation Lemmata}

MISSING!!!!






\subsection{Subject Reduction Theorem}

\begin{theorem}
  \label{subjectreduction}
  Reduction preserves typing i.e.
  $$
  \ruleh
  {\Gamma \vdash t : T
    \\
    t \reduce u
  }
  {\Gamma \vdash u : T}
  $$
  \begin{proof}
    We define $\Gamma \reduce \Delta$ if $\Delta$ is $\Gamma$ where one type
    has been reduced. Then we proof the stronger theorem
    $$
    \ruleh
    {\Gamma \vdash t : T}
    {
      \rulev
      {t \reduce u}
      {\Gamma \vdash u : T}
      \land
      \rulev
      {\Gamma \reduce \Delta}
      {\Delta \vdash t : T}
    }
    $$
    by induction on $\Gamma \vdash t : T$

    \begin{description}

    \item[Axiom] Not possible, because the context is empty and terms are
      sorts and neither of them can be reduced.

    \item[Variable]
      $$
      \ruleh
      {\Gamma \vdash A:s \quad x \notin \Gamma}
      {\Gamma,x:A \vdash x:A}
      $$

      The left part is vacuously satisfied because a variable is in normal
      form and cannot be reduced.

      For the right part of the goal assume $\Gamma,x:A$ reduces to another
      context. There are two cases possible
      \begin{enumerate}

      \item $\Gamma,x:A \reduce \Delta,x:A$ where $\Gamma \reduce
        \Delta$. From the induction hypothesis we can conclude $\Delta \vdash
        A:s$ and therefore $\Delta,x:A \vdash x:A$ is valid.

      \item  $\Gamma,x:A \reduce \Gamma,x:B$ where $A \reduce B$. From the
        induction hypothesis we get $\Gamma \vdash B: s$ and therefore
        $\Gamma,x:B \vdash x:B$ is valid.
      \end{enumerate}

    \item[Weaken]
      $$
      \ruleh
      {\Gamma \vdash A:s \quad x \notin \Gamma \quad \Gamma \vdash t : T}
      {\Gamma,x:A \vdash t : T}
      $$

      The left part of the goal is an immediate consequence of the induction
      hypothesis derived from $\Gamma \vdash t : T$. The right part needs the
      same case split as for the previous rule. For both cases the right part
      of the goal can be derived from the induction hypothesis.


    \item[Product]
      $$
      \rulev
      { \Gamma \vdash A:s_1
        \\
        \Gamma,x:A \vdash B: s_2
        \\
        r(s_1,s_2,s_3)}
      {\Gamma \vdash \Pi x^A. B : s_3}
      $$

      For the left part of the goal assume that $\Pi x^A. B$ reduces. Then
      either  $A$ or $B$ reduces. From the induction hypotheses we conclude
      that the reduced terms have the same sort. So the reduced product has the
      same sort.

      For the right part of the goal assume that the context $\Gamma$
      reduces. The induction hypotheses immediately gives that the terms $A$
      and $B$ have the same sorts, therefore the product has the same sort
      under the reduced context.

    \item[Abstraction]
      $$
      \rulev
      { \Gamma \vdash \Pi x^A. B : s
        \\
        \Gamma,x:A \vdash e: B
      }
      {\Gamma \vdash \lambda x^A. e : \Pi x^A. B}
      $$

      Left part: Assume that $\lambda x^A. e$ reduces. I.e. either $A$ or $e$
      reduce. From the induction hypotheses we conclude the same types.

      Right part: Assume that $\Gamma$ reduces and use the induction
      hypotheses to prove the goal.


    \item[Application]
      $$
      \rulev
      { \Gamma \vdash f : \Pi x^A. B
        \\
        \Gamma \vdash a: A
      }
      {\Gamma \vdash f a : B[x:=a]}
      $$

      Right part: Immediate consequence of the induction hypotheses.

      Left part: Assume that $f a$ reduces and prove that the reduct has type
      $B[x:=a]$. There are 3 cases to consider. Either $f$ reduces or $a$
      reduces or $f$ is a lambda term and $f a$ is a redex.
      \begin{enumerate}

      \item $f \reduce g$: We get $g: \Pi x^A. B$ by the induction hypothesis
        and $g a : B[x:=a]$.

      \item $a \reduce b$: Same reasoning.

      \item $(\lambda x^{A'} .e) a \reduce e[x:=a]$ where
        $\lambda x^{A'} .e: \Pi x^A. B$:
        Goal: $$\Gamma \vdash e[x:=a] : B[x:=a].$$
        By the subsitution lemma we
        generate the subgoal $$\Gamma,x:A \vdash e : B.$$

        From the generation lemma for abstractions we get
        $$
        \exists s' B'.
        \left[
          \begin{array}{l}
          \Gamma \vdash\Pi x^{A'}.B' : s' \quad\land
          \\
          \Gamma,x:A' \vdash e:B' \quad\land
          \\
          \Pi x^A. B \reducestar \Pi x^{A'}.B'
          \end{array}
          \right]
        $$
        Since reduction preserves product we get $A \reducestar A'$ and
        $b \reducestar B'$. Together with $\Gamma,x:A' \vdash e:B'$ and the
        generalized conversion rule~\ref{generalizedsubtyping} we prove the
        subgoal.
      \end{enumerate}


    \item[Conversion]
      $$
      \rulev
      { \Gamma \vdash t : U
        \\
        \Gamma \vdash T : s
        \\
        T \reduce U \lor U \reduce T
      }
      {\Gamma \vdash t: T}
      $$
      Left and right part of the goal are immediate consequences of the
      induction hypothesis and the conversion rule.
    \end{description}
  \end{proof}
\end{theorem}


MISSING: Condensing lemma!!

%%% Local Variables:
%%% mode: latex
%%% TeX-master: "0typed"
%%% End:
