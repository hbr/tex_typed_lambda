\section{Pure Type Systems}

\subsection{Contexts}

\begin{definition}
  A context $\Gamma = [x_1:T_1, \ldots, x_n:T_n]$ is a list of variables
  together with their types. The types $T_i$ are lambda terms. No duplicate
  variables are allowed. We use the notation
  $\Gamma,x:T = [x_1:T_1, \ldots, x_n:T_n, x:T]$ to introduce the new fresh
  variable $x$ with its type $T$ into the context $\Gamma$.
\end{definition}

The substitution function can be extended to contexts $\Gamma[x:=a]$ by
applying the substitution to each type in the context or formally
$$
\Gamma[x:=a] :=
\begin{cases}
  [][x:=a] &:= []
  \\
  (\Gamma,y:T)[x:=a] &:= \Gamma[x:=a],y:T[x:=a]
\end{cases}
$$



\subsection{The Typing Relation}

\begin{definition}
  A pure type system has a set of sorts $S$, a set of axioms
  $\{(s_1,s_2), \ldots\}$ and a set of rules $\{(s_1,s_2,s_3), \ldots \}$ and
  the inductively defined typing relation $\Gamma \vdash t: T$ where $\Gamma$
  is a context and $t$ and $T$ are terms.
  \begin{description}

  \item[Axiom] $$\ruleh{ \text{axiom}(s_1,s_2)}{[] \vdash s_1:s_2}$$

  \item[Variable]
    $$\ruleh{\Gamma \vdash A:s\quad x \notin \Gamma}{\Gamma,x:A \vdash x:A}$$


  \item[Weaken]
    $$\ruleh
    {\Gamma \vdash A:s \quad  \Gamma \vdash t:T   \quad x \notin \Gamma}
    {\Gamma,x:A \vdash t:T}$$


  \item[Product]
    $$\ruleh
    {\Gamma \vdash A:s_1 \quad
      \Gamma,x:A \vdash B:s_2 \quad
      \text{rule}(s_1,s_2,s_3)}
    {\Gamma \vdash \Pi x^A . B : s_3}
    $$

  \item[Abstraction]
    $$\ruleh
    {\Gamma \vdash \Pi x^A.B: s \quad
      \Gamma,x:A \vdash e:B
    }
    {\Gamma \vdash \lambda x^A.e : \Pi x^A . B}
    $$

  \item[Application]
    $$\ruleh
    {\Gamma \vdash f: \Pi x^A.B \quad
      \Gamma \vdash a:A
    }
    {\Gamma \vdash f a : B[x:=a]}
    $$

  \item[Type Reduction]
    $$\ruleh
    {\Gamma \vdash t: T \quad
      \Gamma\vdash U: s \quad
      (T \reduce U) \lor (U \reduce T)
    }
    {\Gamma \vdash t: U}
    $$
  \end{description}
\end{definition}

We use the following definitions:
\begin{enumerate}

\item A context $\Gamma$ is legal if $\Gamma \vdash t:T$ is valid for some $t$
  and $T$.


\item A term $A$ is legal if $\Gamma \vdash a:A$ is valid for some $\Gamma$
  and $a$ or if $\Gamma\vdash A:T$ is valid for some $\Gamma$ and $T$.
\end{enumerate}




\begin{theorem} Substitution Lemma
  \label{substitutionlemma}
  $$
  \rulev
  { \Gamma \vdash a:A
    \\
    \Gamma,x:A,\Delta \vdash t:T
  }
  % --------------------------------------------
  {\Gamma,\Delta[x:=a] \vdash t[x:=a] : T[x:=a]}
  $$
  \begin{proof}
    We abbreviate $V[x:=a]$ by $V'$ where $V$ can be a term or a context.

    The assertion $\Gamma,x:A,\Delta \vdash t:T$ is not very suited for doing
    induction on. Therefore we reformulate the goal to
    $$
    \rulev{
      \Gamma_0\vdash a:A
      \\
      \Gamma \vdash t : T
    }
    {\forall \Delta .
      \rulev{\Gamma = \Gamma_0,x:A,\Delta}
      {\Gamma_0,\Delta' \vdash t' : T'}
    }
    $$
    which is equivalent to the original goal. Now we assume
    $\Gamma_0\vdash a:A$ and do induction on $\Gamma \vdash t : T$.

    \begin{description}

    \item[Axiom] The goal is vacuously satisfied because there is no $\Delta$
      to satisfy $[] = \Gamma_0,x:A,\Delta$.

    \item[Variable]
      $$
      \begin{array}{l|l}
        \Gamma \vdash B:s
        & \forall \Delta .
          \rulev
          {\Gamma = \Gamma_0,x:A,\Delta}
          {\Gamma_0,\Delta' \vdash B' : s'} \quad(h_1)
        \\
        y \notin \Gamma
        \\
        \hline
        \Gamma,y:B \vdash y:B
        & \goalbg{
          \forall \Delta .
          \rulev
          {\Gamma,y:B = \Gamma_0,x:A,\Delta}
          {\Gamma_0,\Delta' \vdash y': B'}
          }
      \end{array}
      $$

      Now we assume $\Gamma,y:B = \Gamma_0,x:A,\Delta$ and do a case split on
      the structure of $\Delta$.
      \begin{enumerate}

      \item Case $\Delta = []$. This is possible only for $\Gamma = \Gamma_0$,
        $x=y$ and $A = B$. Since $x \notin \Gamma_0$ and by assumption
        $\Gamma_0 \vdash a:A$ we can conclude that $x \notin \freevars(A)$ and
        therefore $A = A'$ so that the goal $\Gamma_0 \vdash x': A'$ is
        satisfied.

      \item Case $\Delta = \Delta_0,y:B$. This implies
        $\Gamma = \Gamma_0,x:A,\Delta_0$. If we feed $\Delta_0$ into the
        induction hypothesis $(h_1)$ we get
        $ \Gamma_0,\Delta_0' \vdash B': s'$. Since $y$ is fresh we have
        $y' = y$. By using the \emph{variable} rule we get the final goal
        $\Gamma_0,\Delta_0',y:B' \vdash y:B'$ i.e.
        $\Gamma_0,\Delta' \vdash y:B'$ because $\Delta' = \Delta_0',y:B'$ by
        definition.

      \end{enumerate}


    \item[Weaken]
      $$
      \begin{array}{l|l}
        \Gamma \vdash B:s
        & \forall \Delta .
          \rulev
          {\Gamma = \Gamma_0,x:A,\Delta}
          {\Gamma_0,\Delta' \vdash B' : s'} \quad(h_1)
        \\
        y \notin \Gamma
        \\
        \Gamma \vdash t:T
        & \forall \Delta .
          \rulev
          {\Gamma = \Gamma_0,x:A,\Delta}
          {\Gamma_0,\Delta' \vdash t': T'} \quad (h_2)
        \\
        \hline % -------------------------------------
        \Gamma,y:B \vdash y:B
        & \goalbg{
          \forall \Delta .
          \rulev
          {\Gamma,y:B = \Gamma_0,x:A,\Delta}
          {\Gamma_0,\Delta' \vdash t': T'}
          }
      \end{array}
      $$

      Now we have to do the same case split on the structure of $\Delta$ as in
      the \emph{variable} rule, this time however using the induction
      hypothesis $(h_2)$ and the \emph{weakening} rule to prove the final
      goal.

    \item[Product]
      $$
      \begin{array}{l|l}
        \text{rule}(s_1,s_2,s_3)
        \\
        \Gamma \vdash B:s_1
        % hypo 1
        & \forall \Delta .
          \rulev
          {\Gamma = \Gamma_0,x:A,\Delta}
          {\Gamma_0,\Delta' \vdash B': s_1'} \quad (h_1)
        \\
        \Gamma,y:B \vdash C:s_2
        % hypo 2
        &  \forall \Delta .
          \rulev
          {\Gamma,y:B = \Gamma_0,x:A,\Delta}
          {\Gamma_0,\Delta' \vdash C': s_2'} \quad (h_2)
        \\
        \hline
        \Gamma \vdash \Pi y^B . C: s_3
        & \goalbg{
          \forall \Delta .
          \rulev
          {\Gamma = \Gamma_0,x:A,\Delta}
          {\Gamma_0,\Delta' \vdash (\Pi y^B. C)' : s_3'}
          }
      \end{array}
      $$

      We assume $\Gamma = \Gamma_0,x:A,\Delta$ and want to derive the goal
      $\Gamma_0,\Delta' \vdash \Pi y^{B'}. C' : s_3$.

      In order to derive the goal by the \emph{product} rule we need the
      premises $\Gamma_0,\Delta' \vdash B': s_1$ and
      $\Gamma_0,\Delta',y:B' \vdash C': s_2$.

      The first one can be derived directly from $(h_1)$. The second one can
      be obtained for $(h_2)$ by using $\Delta,y:B'$ for $\Delta$. The premise
      $\Gamma,y:B = \Gamma_0,x:A,\Delta,y:B$ of $(h_2)$ is then a direct
      consequence of the assumption.

    \item[Abstraction] Like the \emph{product} rule.

    \item[Application]
      $$
      \begin{array}{l|ll}
        \Gamma\vdash f: \Pi y^B.C
        &\forall \Delta .
          \rulev
          {\Gamma = \Gamma_0,x:A,\Delta}
          {\Gamma_0,\Delta' \vdash f': (\Pi y^B.C)'} & (h_1)
        \\
        \Gamma \vdash b: B
        &\forall \Delta .
          \rulev
          {\Gamma = \Gamma_0,x:A,\Delta}
          {\Gamma_0,\Delta' \vdash b': B'} & (h_2)
        \\
        \hline %-------------------------------------------
        \Gamma \vdash f\, b : C[y:=b]
        & \goalbg{
          \forall \Delta .
          \rulev
          {\Gamma = \Gamma_0,x:A,\Delta}
          {\Gamma_0,\Delta' \vdash (f b)': C[y:=b]'}
          }
      \end{array}
      $$
      Let's assume $\Gamma = \Gamma_0,x:A,\Delta$.

      The hypotheses $(h_1)$ and $(h_2)$ let us derive
      $\Gamma_0,\Delta' \vdash f': \Pi y^{B'}.C'$ and
      $\Gamma_0,\Delta' \vdash b': B'$
      which by the \emph{application} rule lead to
      $\Gamma_0,\Delta' \vdash (f b)' : C'[y:=b']$.

      The double substitution law~\ref{doublesubstitution}
      $$
      \begin{array}{llll}
        C[y:=b]'
        & = & C[y:=b][x:=a]       & \text{definition } V'\\
        & = & C[x:=a]\big[y:=b[x:=a]\big] & \text{double substitution} \\
        & = & C'[y:=b']           & \text{definition } V'
      \end{array}
      $$
      derives the equality $C'[y:=b'] = C[y:=b]'$ which immediately proves the
      goal.

    \item[Reduction]
      $$
      \begin{array}{l|ll}
        \Gamma \vdash t:T
        & \forall \Delta .
          \rulev
          {\Gamma = \Gamma_0,x:A,\Delta}
          {\Gamma_0,\Delta' \vdash t': T'} & (h_1)
        \\
        \Gamma \vdash U: s
        & \forall \Delta .
          \rulev
          {\Gamma = \Gamma_0,x:A,\Delta}
          {\Gamma_0,\Delta' \vdash U': s'} & (h_2)
        \\
        (T \reduce U) \lor (U \reduce T)
        \\
        \hline %-------------------------------------------
        \Gamma \vdash t:U
        & \goalbg{
          \forall \Delta .
          \rulev
          {\Gamma = \Gamma_0,x:A,\Delta}
          {\Gamma_0,\Delta' \vdash t': U'}
          }
      \end{array}
      $$
      From the substitution rule for reduction~\ref{reductionsubstitution} and
      the premise $(T \reduce U) \lor (U \reduce T)$ we get
      $(T' \reduce U') \lor (U' \reduce T')$.

      Together with the two induction hypotheses $(h_1)$ and $(h_2)$ and
      applying the \emph{reduction} rule we can immediately prove the goal.
    \end{description}
  \end{proof}
\end{theorem}










\begin{theorem} The type $T$ of a term $t$ is either a sort or its type is a
  sort
  $$
  \ruleh
  {\Gamma \vdash t:T}
  {\exists s. T = s \lor \Gamma\vdash T:s}.
  $$

  \begin{proof}
    By induction on $\Gamma\vdash t:T$.
    \begin{description}

    \item[Axiom] Trivial.


    \item[Variable]
      $\ruleh
      {\Gamma\vdash B:s}
      {\Gamma,x:B \vdash x:B}
      $. Trivial, witness $s$.

    \item[Weaken]
      $$
      \begin{array}{l|l}
        \Gamma \vdash A:s &
        \\
        \Gamma \vdash t:T & \exists s. T = s \lor \Gamma \vdash T:s
        \\
        \hline
        \Gamma,x:A \vdash t:T   & \exists s.
                                  T = s \lor \Gamma,x:A \vdash T:s
      \end{array}
      $$
      The goal is a consequence of the induction hypothesis and an
      application of the weakening rule in case $\Gamma\vdash T:s$.

    \item[Product] Trivial, because the type of a product is always a sort.

    \item[Abstraction] Trivial, because the type of an abstraction is a
      product whose type is always a sort.

    \item[Application]
      $$
      \begin{array}{l|l}
        \Gamma\vdash f:\Pi x^A.B & \exists s. \Pi x^A.B = s \lor
                                   \Gamma\vdash \Pi x^A.B: s
        \\
        \Gamma\vdash a:A
        \\
        \hline
        \Gamma\vdash f a:B[x:=a] & \exists s. B[x:=a] = s \lor
                                   \Gamma\vdash B[x:=s]: s
      \end{array}
      $$
      The type of a product is always a sort. Therefore
      $\Gamma\vdash \Pi x^A.B : s $ must be valid for some sort $s$. The
      product can only be formed if $\Gamma,x:A \vdash B:s$ is valid for some
      sort $s$. By the substition lemma~\ref{substitutionlemma} we get $\Gamma
      \vdash B[x:=a]: s$.

    \item[Reduction]
      $$
      \begin{array}{l|l}
        \Gamma\vdash t:T \\
        \Gamma\vdash U: s \\
        (T \reduce U) \lor (U \reduce T) \\
        \hline
        \Gamma\vdash t:U & \exists s. U = s \lor \Gamma\vdash U:s
      \end{array}
      $$
      Trivial by the second premise.
    \end{description}
  \end{proof}
\end{theorem}






%%% Local Variables:
%%% mode: latex
%%% TeX-master: "0typed"
%%% End:
