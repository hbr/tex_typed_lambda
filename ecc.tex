\section{Terms}

\subsection{Basic Definition}

\begin{definition}
  Typed lambda terms need a set of sorts $\Sorts = {\uni_0, \uni_1, \ldots}$ and
  a finite set of variables. They are defined by the grammar
  $$
  \begin{array}{llll}
    t &::=& s                 &\text{sort}
    \\
      &\mid& x                &\text{variable}
    \\
      &\mid& t t              &\text{application}
    \\
      &\mid& \lambda x^t.t    &\text{abstraction}
    \\
      &\mid& \Pi x^t.t        &\text{product}
  \end{array}
  $$
  where $s$ ranges over sorts, $x$ ranges over variables and $t$ ranges over
  terms.


\end{definition}



\subsection{Substitution}

\subsection{Reduction}

\subsection{Kinds}

\begin{definition}
  The set $\Kinds_s$ of kinds over the sort $s$ is defined by the grammar
  $$
  \begin{array}{lll}
    \kappa &::=& s
    \\
           &\mid& \Pi x^t. \kappa
  \end{array}
  $$
  where $\kappa$ ranges over elements of $\Kinds_s$, $T$ over terms and $x$
  over variables.
\end{definition}


\begin{definition}
  The set $\Kinds_n$ is defined as
  $$
  \Kinds_n :=
  \cup_{i < n} \Kinds_{\uni_i} =
  \Kinds_{\uni_0} \cup \Kinds_{\uni_1} \cup \ldots \cup \Kinds_{\uni_{n-1}}
  $$
\end{definition}


\begin{definition}
  The set $\Kinds$ of all kinds is defined as
  $$
  \Kinds := \Kinds_{\uni_0} \cup \Kinds_{\uni_1} \cup \ldots
  $$
\end{definition}




\section{Typing}

\begin{definition}
  The set of sorts is ordered by
  $$
  \uni_i < \uni_{i+1}
  $$
\end{definition}


\begin{definition}
  The relation $<$ on sorts is extended to terms by the rule
  $$
  \ruleh
  {T < U \quad A \betaeq B}
  {\Pi x^A. T < \Pi x^B. U}
  $$
\end{definition}


\begin{definition}
  The relation $\le$ is defined inductively by the rules
  \begin{description}
  \item[reflexive] $A \le B$

  \item[transitive] $\ruleh{A \le B \quad B < C}{A \le C}$

  \item[$\beta$ equivalence] $\ruleh{A \reduce B \lor B \reduce A}{A \le B}$
  \end{description}
  i.e. $\le$ is a preorder.
\end{definition}

\begin{definition}
  Typing relation as usual.
\end{definition}



\begin{theorem}
  Substitution lemma \label{substitutionlemma}
  %------------------
  $$
  \ruleh
  {\Gamma,x:A,\Delta \vdash t : T
    \quad
    \Gamma \vdash a: A
  }
  {\Gamma,\Delta[x:=a] \vdash t[x:=a] : T[x:=a]}
  $$
\end{theorem}



\begin{theorem}
  Subject reduction theorem.
  %
  $$
  \ruleh
  {\Gamma \vdash t : T  \quad t \reduce u}
  {\Gamma \vdash u : T}
  $$
\end{theorem}

Note that the reverse of subject reduction is not valid. E.g. For
$\Prop < s_1 < s_2$ the judgement
%
$[] \vdash \Prop: s_1$ is valid while
%
$[] \vdash (\lambda x^{s_2}. x) \Prop : s_1$
%
although $(\lambda x^{s_2}. x) \Prop \reduce \Prop$ is invalid.
%
Only
%
$[] \vdash (\lambda x^{s_2}. x) \Prop : s$
%
is valid for $s_2 \le s$.




\begin{lemma}
  Strengthening lemma.
  %-------------------
  $$
  \ruleh
  {\Gamma,x:A,\Delta \vdash t:T
    \quad
    x \notin \FV(t) \cup \FV(T) \cup \FV(\Delta)}
  {\Gamma,\Delta \vdash t : T}
  $$
  \begin{proof}
    Not straightforward. See Luo p.38.
  \end{proof}
\end{lemma}



\begin{definition}
  A \emph{principal type}
  % ----------------------
  $T$ of a term $t$ is a lowest type of the term.
  $$
  \Gamma \vdash^p t: T :=
  \Gamma \vdash t :T \land
  \forall U. \ruleh{\Gamma \vdash t: U}{T \le U}
  $$
\end{definition}
%
Note: All principal types of a term are beta equivalent.












\section{Quasinormalization}

\subsection {Examples}

$
[] \vdash \lambda x^s.x : \underbrace{\Pi x^s. s}_{s \to s}
$


$[] \vdash \Pi x^s B^{s \to \Prop}. B x : s$: We have $s \to \Prop: \ell(s)$
because $\ell(\Prop) \le \ell(s)$.






\subsection{Type Level}


\begin{definition}
  The level $\ell(T)$ of a type $T$ is the lowest universe in which $T$
  resides up to reduction.
  $$
  \ruleh
  { \Gamma \vdash T: s_1}
  { \ell(T) := \mu s_2. \exists U. \Gamma \vdash U:s_2 \land T \reducestar U}
  $$
  Note that the level of a sort is its principal type.
\end{definition}


\begin{lemma}
  Levels for Products.
  % -------------------
  $$
  \ell(\Pi x^A.B) =
  \begin{cases}
    s_\text{min}
    & \text{for } \Gamma,x:A \vdash^p B:s_\text{min}
    \\
    \text{max}(\ell(A), \ell(B))
    & \text{for } \Gamma,x:A \vdash^p B:s \land s_\text{min} < s
  \end{cases}
  $$
  which implies $\ell(B) \le \ell(\Pi x^A.B)$
  \begin{proof} Evident by the product, conversion, subtyping rules and the
    definition of $\ell$.
  \end{proof}
\end{lemma}


\begin{lemma}
  % Levels and beta equivalence and subtypes
  % ----------------------------------------
  Let $A$ and $B$ be types valid in some context (i.e. $\Gamma \vdash A : s_A$
  and $\Gamma \vdash B : s_B$ for some $s_A,s_B$).
  \begin{enumerate}
  \item $\ruleh{A \betaeq B}{\ell(A) = \ell(B)}$

  \item $\ruleh{A < B}{\ell(A) < \ell(B)}$

  \item $\ruleh{A \le B}{\ell(A) \le \ell(B)}$
  \end{enumerate}

  \begin{proof} \
    \begin{enumerate}
    \item
      By Church Rosser exists a $C$ with $A \reducestar C$ and  $B \reducestar
      C$. By definition $\ell(A) = \ell(C) = \ell(B)$.

    \item Proof by induction on $A < B$.
    \begin{enumerate}
    \item $s_1 < s_2$ implies trivially $\ell(s_1) < \ell(s_2)$

    \item $\ruleh{T < U \quad A \betaeq B}{\Pi x^A.T < \Pi x^B.U}$

      By induction hypothesis $\ell(T) < \ell(U)$.

       MISSING
    \end{enumerate}

    \item
      Consequence of 1. and 2. and the definition of $\le$.
    \end{enumerate}
  \end{proof}
\end{lemma}



\begin{lemma}
  A type preserving substitution in a type can only lower the level
  $$
  \ruleh
  {\Gamma,x:A \vdash B:s \quad \Gamma \vdash a:A}
  {\ell(B[x:=a] \le \ell(B)}
  $$
  \begin{proof}

    By definition of $\ell$ there exists some $B' \betaeq B$ with
    %
    $\Gamma,x:A \vdash B': \ell(B)$
    %
    which implies (by using the substitution lemma~\ref{substitutionlemma})
    %
    $\Gamma \vdash B'[x:=a] : \ell(B)$.
    %
    Since $\ell(B'[x:=a])$ is the minimal universe in which it can reside we
    get  $\ell(B'[x:=a]) \le \ell(B)$ and because $B'[x:=a]$ and $B[x:=a]$ are
    beta equivalent we conclude the goal $\ell(B[x:=a]) \le \ell(B)$.

  \end{proof}
\end{lemma}


\begin{lemma}
  $$
  \ruleh
  {\Gamma \vdash^p f : F
    \quad
    \Gamma \vdash^p f a : U
  }
  {\ell(U) \le \ell(F)}
  $$
  \begin{proof}
    By generation lemma there exist $x,A,B$ such that
    %
    $\Gamma,x:A \vdash^p \Pi x^A.B$ of $f$
    %
    and
    %
    $\Gamma \vdash a:A$.  where $\Gamma \vdash f a : B[x:=a]$,
    $\Pi x^A.B \betaeq F$ and $B[x:=a] \betaeq U$.

    The by the previous lemmas we have
    $$
    \ell(U) = \ell(B[x:=a]) \le \ell(B) \le \ell(\Pi x^A.B) = \ell(F)
    $$

    NOT YET COMPLETE
  \end{proof}
\end{lemma}


\begin{definition}
  Level of a redex $\ell_r((\lambda x^A.e) a)$.
  % --------------------------------------------
  $$
  \ruleh
  {\Gamma \vdash^p \lambda x^A.e : \Pi x^A.B
    \quad
    \Gamma \vdash a: A
    }
  {\ell_r((\lambda x^A.e) a) := \ell(\Pi x^A.B)}
  $$
\end{definition}



\subsection{$\ECC^n$}

\begin{definition}
The extended calculus of constructions $\ECC^n$ is the extendend calculus of
constructions where $\uni_n$ is the highest sort/universe.
\end{definition}

\begin{lemma}
  Form of a type in the highest universe.
  % -------------------------------------
  In $\ECC^n$ a type $T$ with $\ell(T) = \uni_n$ i.e. $\Gamma \vdash T :
  \uni_n$ has one of the forms
  \begin{itemize}
  \item $T = \uni_{n-1}$

  \item $T = \Pi x^A.B$
  \end{itemize}
  \begin{proof}
    By induction on $\Gamma \vdash T: \uni_n$.

    MISSING
  \end{proof}
\end{lemma}



\begin{definition}
  $j$-quasinormal term:
  % -------------------
  A term $t$ with $\Gamma \vdash t : T$ is \emph{j-quasinormal} if it does not
  contain any redex whose level is $j$ (The level of a redex is the level of
  a principal type of the function term of the redex).
\end{definition}

\begin{definition}
  $j$-degree $\degree^0_j(T)$  of a type $T$ which is $i$-quasinormal for all
  $j < i \le n$:
  $$
  \degree^0_j(T) :=
  \begin{cases}
    0 & \text{if } \ell(T) \not= j
    \\
    1 & \text{if } T = \uni_{j-1} \text{ i.e. } \ell(T) = j
    \\
    1 & \text{if $T$ is a base term } \land \ell(T) = j
    \\
    1 + \text{max}(\degree^0_j(A), \degree^0_j(B)) &
    \text{if } T = \Pi x^A.B \land \ell(T) = j
  \end{cases}
  $$
\end{definition}



\subsection{Proof of Quasinormalization}

The theorem has the form
\begin{quote}
  Every term in $\ECC^n$ can be reduced to some term which is $i$-quasinormal
  for $0 < i \le n$.
\end{quote}


Proof by a strong induction argument proving
$$
\rulev
{
  \Gamma \vdash t : T
  \\
  0 < j \le n
  \\
  \forall i.
  \left(
    \rulev
    {j+1 \le i \le n}
    {\exists u. t \reducestar u \land u \text{ is $i$-quasinormal}}
  \right)
  \\
  j \le i \le n
}
{
  \exists u. t \reducestar u \land u \text{ is $i$-quasinormal}
}
$$

With the definition
$$
P(i,j) :=
\ruleh
{j \le i \le n}
{\exists u. t \reducestar u \land u \text{ is $i$-quasinormal}}
$$
%
we can formulation the induction condition as
$$
\rulev
{
  \Gamma \vdash t : T
  \\
  0 < j \le n
  \\
  \forall i. P(j+1,i)
}
{
  \forall i. P(j,i)
}
$$



Assume $t$ is $i$-quasinormal for $j+1 \le i \le n$. Then there is some $u$
with $t \reduce u$ such that $u$ is $i$-quasinormal for $j \le i \le n$.

\begin{quote}
  All terms are $n$-quasinormal because there are no $n$-level redexes. An
  $n$-level redex needs universe $\uni_{n+1}$ which does not exist in $\ECC^n$.
\end{quote}


\subsection{Essence of the Proof}

Assume we have a welltyped $j$-level ($0 < j$) redex of the form
$$
\lambda (x^A.e) a \reduce e[x:=a]
$$
and $e,a$ are $i$-quasinormal for $j < i \le n$. This requires
$$
\begin{array}{l}
  \Gamma \vdash^p \lambda x^A.e : \Pi x^A.B
  \\
  \Gamma \vdash a: A
  \\
  \ell(\Pi x^A.B) = j
  \\
  j = \text{max}(\ell(A), \ell(B))
  \\
  \ell(A) \le j \land \ell(B) \le j \land (\ell(A) = j \lor \ell(B) = j)
\end{array}
$$

The term $e[x:=a]$ has new redexes (i.e. redexes which are neither in $e$ nor
in $a$) only if the variable $x$ is somewhere in a key position and $a$ is an
abstraction. (AGUAS: The redex might be part of a term
$(\lambda x^A.e) a b c\ldots$ and $e$ is an abstraction whose principal type
has level $j$.) The level of the new redexes depend on the level of the
principal type of $a$ or the level of the principal type of $e$.

If the level is greater than $j$, then the term can be reduced to a
quasinormal term (induction premise).

If the level is less than $j$, then $e[x:=a]$ is $j$-quasinormal.

The only interesting case is that the level of the principal type of $a$ has
level $j$.


$A$ is the principal type of $x$, but the principal type of $a$ can be less or
equal $A$ (i.e. $\text{PT}(a) \le A$). We are interested only in cases where
$\ell(\text{PT}(a)) = j$. Since $\ell(A) \le j$ this can only happen if
$\ell(A) = j$ i.e. $A$ is a principal type of $a$.



Let's assume
$$
\begin{array}{l}
  a = \lambda y^C.d
  \\
  \Gamma \vdash^p \lambda y^C.f: \Pi y^C.D
  \\
  \ell(\Pi y^C.D) = j = \text{max}(\ell(C), \ell(D))
\end{array}
$$
and $e$ contains a base term of the form
$$
x c u v \ldots
$$
i.e. $e[x:=a]$ contains the term
$$
(\lambda y^C.f) c u v \ldots
$$
where $(\lambda y^C.f) c$ is a $j$-level redex.
$$
\Gamma \vdash^p \lambda x^{\Pi y^C.D}.e : \Pi x^{\Pi y^C.D}. B
$$


\section{Universe Consistency}

\subsection{Polymorphic Identity Function}

In untyped lambda calculus we use the term $\lambda x.x$ for the identity
function. In typed lambda calculus the argument must have a type. Since the
type is arbitrary we need a type argument.

{\def\id{\text{id}}

  The polymorphic typed identity function:
  $$
  %
  \begin{array}{lllll}
    T_\id &:=& \Pi \alpha^s. \alpha \to \alpha &:& s_\id
    \\
    \id &:=& \lambda \alpha^s x^\alpha . x &:& T_\id
  \end{array}
  $$
  %
  If $s$ is non propositional then $s < s_\id$ must be satisfied. This
  condition makes sure that
  $$
  \id \, T_\id \, \id
  $$
  is an illegal expression.
}



\subsection{Many Identity Functions}


$$
\begin{array}{lllll}
  T_i &:=& \Pi \alpha^{s_i}. \alpha \to \alpha  &:& s'_i
  \\
  f_i &:=& \lambda \alpha x . x &:& T_i
\end{array}
$$
with the constraint $s_i < s'_i$ because $s_i$ is non propositional. Now the
term
$$
  f_i T_j f_j
$$
requires $s'_j \le s_i$ which implies $s_j < s_i$. I.e. $f_i$ must be from a
strictly higher universe than $f_j$ and $T_j$.





\subsection{Mockingbird Combinator}

In untyped lambda calculus we defined the mockingbird combinator as $M :=
\lambda x . x x$. This makes the infinite loop
$$
MM = (\lambda x . x x) M \reduce M M \reduce M M \ldots
$$
%
possible.

If we wanted to define a typed mocking bird combinator it would have the form
$$
M := \lambda \alpha^s x^\alpha. x \alpha x
$$
because we get
$$
M T M
= (\lambda \alpha^s x^\alpha. x \alpha x) T M
\reduce M T M
\reduce M T M \ldots
$$

The type $T$ of the mockingbird combinator must have the form
$$
T := \Pi \alpha^s x^\alpha. U : s_M
$$
where $s < s_M$ must be satisfied for non propositional $U$. This condition
makes the expression $M T M$ impossible because $T$ cannot be fed as an
argument for $\alpha$.

Another argument shows that such a term is not typable. In order for
$\lambda \alpha^s x^\alpha. x \alpha x$ to be welltyped we must find a valid
context $\Gamma$ such that
$$
\Gamma,\alpha:s,x:\alpha \vdash x \alpha x : U
$$
is a valid typing judgement. This requires that there exists some $\Pi y^A.B$
with $x: \Pi y^A.B$ and $\Pi y^A.B \reduceeq \alpha$ which is impossible
because $\alpha$ is a variable which is always in normal form and therefore
cannot be equivalent to a product.


%%% Local Variables:
%%% mode: latex
%%% TeX-master: "0typed"
%%% End:
