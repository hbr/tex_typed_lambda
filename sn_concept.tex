\section{Concept of Strong Normalisation}


\begin{definition}
  A term is in \emph{normal form} if it cannot be reduced.
  $$
  \ruleh{a \reduce b}
  {\perp}
  $$
\end{definition}



\begin{definition}
  A term in \emph{strongly normalizing} if all reduction paths end up in a
  normal form. We define the set $\SN$ of strongly normalizing terms
  inductively by
  $$
  \ruleh{
    \forall b.
    \ruleh{a \reduce b}{b \in \SN}
  }
  {a \in \SN}
  $$
\end{definition}


If we know that an term $a$ is strongly normalizing i.e. $a \in \SN$ and we
want to prove a property $p(a)$ of the term it is sufficient to prove
$$
\ruleh
{\forall b.
  \ruleh {a \reduce b}{ p(b)}
}
{p(a)}
$$
i.e. in proving the goal $p(a)$ we can assume the induction hypothesis that
all terms $b$ to which $a$ reduces already satisfy the goal.



\begin{lemma}
  \emph{Strongly normalizing substitution}
  \label{stronglynormalizingsubstitution}
  $$
  \rulev
  {
    a \in \SN
    \\
    e[x:=a] \in \SN}
  {e \in \SN}
  $$

  \begin{proof}
    We prove the modified equivalent lemma
    $$
    \rulev
    {
      a \in \SN
      \\
      b \in \SN
      \\
      b = e[x:=a]
    }
    {e \in \SN}
    $$
    by assuming $a \in \SN$ and doing induction on $b \in \SN$.

    Goal $p(b) := \forall e. \ruleh{b = e[x:=a]} {e \in \SN}$. Let's assume
    $b = e[x:=a]$ and prove the subgoal $e \in \SN$. I.e. we have to assume
    $e \reduce f$ and proof $f \in SN$. In order to do this we can use the
    induction hypothesis
    $$
    \forall c. \ruleh{b \reduce c}{p(c)}.
    $$

    $$
    \begin{array}{llll}
      e \reduce f
      & \imp & e[x:=a] \reduce f[x:=a]
      & \text{Reduction substitution~\ref{reductionsubstitution}}
      \\
      & \imp & b \reduce f[x:=a]
      & b = e[x:=a]
      \\
      & \imp & p(f[x:=a])  & \text{induction hypothesis}
      \\
      & \imp & \forall e. \ruleh{f[x:=a] = e[x:=a]}{e \in \SN}
      & \text{definition of $p$}
      \\
      & \imp & f \in \SN & \text{using } e = f
    \end{array}
    $$
  \end{proof}
\end{lemma}



\begin{lemma}
  \emph{Strongly normalizing redex}
  \label{stronglynormalizingredex}
  $$
  \ruleh
  {e[x:=a] \in \SN \quad a \in \SN \quad A \in \SN}
  {(\lambda x^A e)a \in \SN}
  $$
  \begin{proof}
    In the proof we use the abbreviation
    $$ p(e,a,A) := (\lambda x^A e) a \in \SN. $$

    From the previous lemma~\ref{stronglynormalizingsubstitution} we know that
    $e[x:=a] \in \SN$ and $a \in \SN$ imply $e \in SN$. Therefore we can add
    the assumption $a \in \SN$ and prove the equivalent lemma
    $$
    \rulev
    {
      e \in SN
      \\ e[x:=a] \in \SN
      \\ a \in SN
      \\ A \in SN
      }
    {p(e,a,A)}.
    $$
    We do this by 4 nested inductions.
    \begin{enumerate}

    \item Induction on $e \in SN$
      $$
      \begin{array}{ll}
        \text{goal}:
        & \forall a A.
          \left[
          \rulev{
          e[x:=a] \in \SN
          \\ a \in \SN
          \\ A \in \SN
          }
        {p(e,a,A)}\right]
        \\
        \text{hypo }h_1:
        & \forall f a A.
          \left[
          \rulev{e \reduce f \\ e[x:=a] \in \SN \\ a \in \SN \\ A \in \SN}
        {p(f,a,A)}\right]

      \end{array}
      $$

    \item Induction on $a \in SN$:
      We assume $e[x:=a] \in \SN$ and
      $a \in \SN$ and do induction on $a \in SN$.
      $$
      \begin{array}{ll}
        \text{goal}:
        & \forall A. \left[\rulev{A \in \SN}{p(e,a,A)}\right]
        \\
        \text{hypo }h_2:
        & \forall b A.
          \left[
          \rulev{a \reduce b \\ A \in \SN}{p(e,b,A)}
          \right]
      \end{array}
      $$

    \item Induction on $A \in SN$:
      We assume $A \in \SN$ and do induction on the assumption.
      $$
      \begin{array}{ll}
        \text{goal}:
        &  p(e,a,A)
        \\
        \text{hypo }h_3:
        & \forall B. \rulev{A \reduce B}{p(e,a,B)}
      \end{array}
      $$

    \item Now we want to prove $p(e,a,A)$ i.e. $(\lambda x^A.e) a \in \SN$. By
      definition of $\SN$ we have to prove
      $\forall c. \ruleh{(\lambda x^A. e) a \reduce c}{c \in \SN}$. We prove
      this by induction on $(\lambda x^A. e) a \reduce c$ which has only 4
      valid cases.

      \begin{enumerate}

      \item $(\lambda x^A. e) a \reduce e[x:=a]$:

        In this case the goal is $e[x:=a] \in \SN$ which is valid by
        assumption.

      \item
        $(\lambda x^A.e) a \reduce (\lambda x^A.f) a$ where $e \reduce f$:

        The goal is $(\lambda x^A.f) a \in \SN$ i.e. $p(f,a,A)$ which can be
        proved by $h_1$.
      \item
        $(\lambda x^A.e) a \reduce (\lambda x^B.e) a$ where $A \reduce B$:

        The goal is $(\lambda x^B.e) a \in \SN$ i.e. $p(e,a,B)$ which can be
        proved by $h_3$.

      \item
        $(\lambda x^A.e) a \reduce (\lambda x^A.e) b$ where $a \reduce b$:


        The goal is $(\lambda x^A.e) b \in \SN$ i.e. $p(e,b,A)$ which can be
        proved by $h_2$.
      \end{enumerate}
    \end{enumerate}
  \end{proof}
\end{lemma}



\begin{lemma}
  All subterms of a strongly normalizing term are strongly
  normalizing\label{stronglynormalizingsubterms}. A term is either a sort, a
  variable, an appliction, an abstraction or a product. The application $ab$
  has subterms $a$ and $b$, the abstraction $\lambda x^A.e$ has subterms $A$
  and $e$ and the product $\Pi x^A.B$ has subterms $A$ and $B$. It is not
  necessary to analyze sorts and variables because the don't have subterms.

  \begin{enumerate}


  \item Application
    $\ruleh{ab \in \SN}{a \in \SN \land b \in \SN}$:

    We have to prove the two equivalent subgoals
    $\ruleh{e \in \SN \quad e = ab}{a \in \SN}$ and
    $\ruleh{e \in \SN \quad e = ab}{b \in \SN}$. Here we prove only the first
    one because the proof of the second one is nearly indentical.

    \begin{proof} By induction on $e \in \SN$ using the definition
      $p(e):= \forall a b. \ruleh{e = a b}{a \in \SN}$.

      $$
      \begin{array}{ll}
        \text{goal}: & p(e)
        \\
        \text{hypo}:
        & \forall f. \ruleh{e \reduce f}{p(f)}
      \end{array}
      $$
      In order to prove the goal we assume $e = a b$ and $a \reduce c$ and try
      to prove the subgual $c \in \SN$.
      $$
      \begin{array}{llll}
        a \reduce c
        & \imp & ab \reduce cb &\text{definition of }\reduce
        \\
        & \imp & e \reduce cb   &\text{assumption}
        \\
        & \imp & p(cb)          & \text{hypo}
        \\
        & \imp & c \in \SN      & \text{definition of }p(cb)
      \end{array}
      $$
    \end{proof}


  \item Abstraction
    $\ruleh{\lambda x^A.e \in \SN}{A \in \SN \land e \in \SN}$:
    Again we define two equivalent subgoals
    $\ruleh{a \in \SN \quad a = \lambda x^A.e}{A \in \SN}$ and
    $\ruleh{a \in \SN \quad a = \lambda x^A.e}{e \in \SN}$ and prove only the
    first one.

    \begin{proof} By induction on $a \in \SN$ using the definition
      $p(a) := \forall A e. \rulev{a = \lambda x^A.e}{A \in \SN}$.
      $$
      \begin{array}{ll}
        \text{goal}: & p(a)
        \\
        \text{hypo}: & \forall b.\ruleh{a \reduce b}{p(b)}
      \end{array}
      $$

      Assume $a = \lambda x^A.e$ and $A \reduce B$ and prove the subgoal
      $B \in \SN$.
      $$
      \begin{array}{llll}
        A \reduce B
        & \imp & \lambda x^A.e \reduce \lambda x^B.e
        &\text{definition of }\reduce
        \\
        & \imp & a \reduce  \lambda x^B.e & \text{assumption}
        \\
        & \imp & p(\lambda x^B.e) & \text{hypo}
        \\
        & \imp & B \in \SN & \text{definition of }p(\lambda x^B.e)
      \end{array}
      $$
    \end{proof}

  \item Product
    $\ruleh{\Pi x^A.B \in \SN}{A \in \SN \land B \in \SN}$:
    \begin{proof}
      Similar to the above proofs for application and abstraction.
    \end{proof}

  \end{enumerate}
\end{lemma}
