\section{Proof of Strong Normalization}

\subsection{Overview}


\begin{itemize}

\item $\nu:\Kinds \to \SAT^*$: A mapping from kinds to model sets.

\item $\xi = [x_1:M_1, x_2:M_2, \ldots]$: Interpretation of type variables of
  a context i.e. a mapping from the type variables of a context to a model
  (which is an element of $\SAT^*$ i.e either a saturated set or a function
  mapping models to models).

  $\xi \vDash \Gamma$ means that $\Gamma$ is a valid context and for all
  $(x:\kappa) \in \Gamma$ the statement $\xi(x) \in \nu(\kappa)$ is valid.

  Property: For each valid context $\Gamma$ there exists a type variable
  interpretation $\xi$ with $\xi \vDash \Gamma$.

\item Extension of type variable interpretation $\xi$ to a type/kind
  interpretation $\typint{T}{\xi}$:
  %
  Maps all elements of $\Types \cup \Kinds$
  for which $\Gamma \vdash \tau : \kappa$ or $\Gamma \vdash \kappa : s$ is
  valid, to a model.

  Properties:
  \begin{enumerate}
  \item
    $$\ruleh
    {\Gamma\vdash t: T \quad \xi \vDash \Gamma}
    {\typint{T}{\xi}  \in \SAT}$$

  \item
    $$\ruleh
    {\Gamma\vdash \Pi x^A . B : s \quad \xi \vDash \Gamma}
    {\typint{\Pi x^A . B}{\xi}
      =
      \typint{A}{\xi} \tolam
      \begin{cases}
        \typint{B}{\xi} & A \in \Types
        \\
        \cap_{M \in \nu(A)} \typint{B}{\xi,x:M}
        & A \in \Kinds
      \end{cases}
    }$$


  \item
    $$\ruleh
    {\Gamma\vdash \Pi x^A . B : s
      \quad
      \Gamma\vdash a:A
      \quad
      \xi \vDash
      \Gamma}
    { \typint{B[x:=a]}{\xi}
      =
      \begin{cases}
        \typint{B}{\xi} & A \in \Types
        \\
        \typint{B}{\xi,x:\typint{a}{\xi}} & A \in \Kinds
      \end{cases}
    }$$
\end{enumerate}

\item $\rho = [x_1: t_1, x_2 : t_2, \ldots]$ Variable interpretation: Mapping
  of all variables of a context to a term. $\rho,\xi \vDash \Gamma$ is defined
  as $\xi \vDash \Gamma$ and $\rho(x) \in \typint{T}{\xi}$ for all
  $(x,T) \in \Gamma$.


\item Extension of $\rho$ to a term interpretation
  %
  $\trmint{t}{\rho} := t[x_1:= \rho(x_1), x_2:= \rho(x_2), \ldots]$
\end{itemize}







\subsection{Saturated Sets}

\begin{definition} % Base terms
  The set of base terms $\BT$ is defined inductively by the rules
  \begin{enumerate}

  \item $x,s \in \BT$

  \item $\ruleh{a \in \BT \quad b \in \SN}{a b \in \BT}$
  \end{enumerate}
\end{definition}


\begin{lemma}
  \label{BTSN}
  Base terms are strongly normalizing $\ruleh{a \in \BT}{a \in \SN}$.
  \begin{proof}
    According to the definition base terms begin with a variable or a sort and
    are followed by zero or more strongly normalizing terms (e.g. $x a b c
    \ldots$). Evidently these terms are strongly normalizing.
  \end{proof}
\end{lemma}





\begin{definition} % Key reduction
  The relation \emph{key reduction} denoted by the binary operator
  $\keyreduce$ is defined inductively by the rules
  \begin{enumerate}

  \item $(\lambda x^A.e) a \keyreduce e[x:=a]$

  \item $\ruleh{a \keyreduce b}{a c \keyreduce b c}$
  \end{enumerate}
\end{definition}

I.e. key reduction is possible only if the term is an application and the
leftmost application is a redex.






\begin{definition} % Saturation
  A saturation of a set $S$ of terms $S^\sat$ is a closure defined inductively
  by the rules
  \begin{enumerate}

  \item $\ruleh{a \in S}{a \in S^\sat}$

  \item $\ruleh{a \in \BT}{a \in S^\sat}$

  \item $\ruleh
    {a \keyreduce b \quad a \in \SN \quad b \in S^\sat}
    {a \in S^\sat}$
  \end{enumerate}
\end{definition}

\begin{definition}
  The set $\SAT$ of all saturated sets is defined by
  $$
  \SAT := \{S \mid  S^\sat \subseteq S \land S \subseteq \SN \}
  $$
\end{definition}


\begin{lemma}
  \label{SNinSAT}
  The set of strongly normalizing terms is saturated i.e. $\SN \in \SAT$.

  \begin{proof} Since $\SN \subseteq \SN$ is satisfied trivially we have to
    prove the condition
    $$
    \ruleh{a \in \SN^\sat}{a \in \SN}
    $$
    which we prove by induction on $a \in \SN^\sat$.

    \begin{enumerate}
    \item $\ruleh{a \in \SN}{a \in \SN^\sat}$. Goal $a \in \SN$. Trivial by
      assumption.

    \item $\ruleh{a \in \BT}{a \in S^\sat}$. Goal $a \in \SN$. By lemma
      \emph{base terms are strongly normalizing}~\ref{BTSN}

    \item $\ruleh
      {a \keyreduce b \quad a \in \SN \quad b \in S^\sat}
      {a \in \SN^\sat}$: Goal $a \in \SN$. Trivial by assumption.
    \end{enumerate}
  \end{proof}
\end{lemma}






\subsection{Model Sets}

\begin{definition}
  A model is either a saturated set of terms or a function mapping a set of
  models to another set of models. The function $\nu$ maps kinds to collection
  of models and is defined as
  %
  $$
  \nu(A) :=
  \begin{cases}
    \nu(s) := \SAT
    \\
    \nu(\Pi x^A.B) := \nu(B) & A \in \Types, B \in \Kinds
    \\
    \nu(\Pi x^A.B) := \{ f \mid f:\nu(A) \to \nu(B)\} & A,B \in \Kinds
    \\
    \nu(\lambda x^A. B) := \nu(B)       & B \in \Kinds
    \\
    \nu(A B) := \nu(A)                  & A \in \Kinds
  \end{cases}
  $$
\end{definition}


\begin{definition}
  Let $A$ be a kind. Then we define a canonical model $\nu^c(A) \in \nu(A)$
  according to the definition
  $$
  \nu^c(A) :=
  \begin{cases}
    \nu^c(s) := \SN
    \\
    \nu^c(\Pi x^A. B) := \nu^c(B) & A \in \Types, B \in \Kinds
    \\
    \nu^c(\Pi x^A. B) := \bullet \mapsto \nu^c(B) & A,B \in \Kinds
    \\
    \nu^c(\lambda x^A. B) := \nu^c(B)       & B \in \Kinds
    \\
    \nu^c(A B) := \nu^c(A)                  & A \in \Kinds

  \end{cases}
  $$
\end{definition}



\subsection{Interpretation of Type Variables}

\begin{definition}
  An interpretation of type variables is an assignment of models to type
  variables of the form
  $$
  \xi = [x_1: M_1, x_2: M_2, \ldots].
  $$

  We say that $\xi$ is a valid interpretation of the type variables of a
  context $\Gamma$ i.e. $\xi \vDash \Gamma$. if the following rules are satisfied.
  \begin{enumerate}

  \item $ [] \vDash []$

  \item
    $\ruleh
    {\xi \vDash \Gamma
      \quad
      \Gamma,x:A \vdash x:A
      \quad
      A \in \Types
    }
    {\xi \vDash \Gamma,x:A}
    $

  \item
    $\ruleh
    {\xi \vDash \Gamma
      \quad
      \Gamma,x:A \vdash x:A
      \quad
      A \in \Kinds
      \quad
      M \in \nu(A)
    }
    {\xi,x:M \vDash \Gamma,x:A}
    $
  \end{enumerate}

  Note that according to this definition $\xi \vDash \Gamma$ implies that the
  context $\Gamma$ is a valid context.
\end{definition}



\subsection{Type Interpretation}

\begin{definition}
  For each valid interpretation $\xi$ of type variables (i.e. $\xi \vDash
  \Gamma$) we define an interpretation of types
  %
  $\dbrack{T}_\xi$ such that
  %
  $
  \ruleh
  {\Gamma \vdash T: C
    \quad
    C \in \Kinds
    \quad
    \xi \vDash \Gamma}
  {\dbrack{T}_\xi \in \nu(C)}
  $
  recursively on the structure of $T$.

  \begin{description}

  \item[Sort] $$\dbrack{s}_\xi := \SN$$

  \item[Variable]
    $$
    \ruleh
    {\Gamma \vdash x:A \quad A \in \Kinds \quad \xi \vDash \Gamma}
    {\dbrack{x}_\xi := \xi(x)}
    $$

  \item[Application]
    $$
    \ruleh
    {\Gamma \vdash f a : C
      \quad
      C \in \Kinds
      \quad
      \xi \vDash \Gamma
    }
    {\dbrack{f a}_\xi :=
      \begin{cases}
        \dbrack{f}_\xi &
        \ruleh{\Gamma \vdash a:A}{A\in \Types}
        \\
        \dbrack{f}_\xi ( \dbrack{a}_\xi ) &
        \ruleh{\Gamma \vdash a:A}{A\in \Kinds}
      \end{cases}
    }
    $$

    From the generation lemma we conclude the existence of minimal $A$ and $B$
    such that $\Gamma \vdash f: \Pi x^A.B$ and $\Gamma \vdash a:A$ and
    $B[x:=a] \le C$ and $a \notin \Kinds$.

    Since $C$ is a kind, $B[x:=a]$ and the product are kinds as well and
    $\dbrack{f}_\xi$ exists and is in $\nu (\Pi x^A.B)$ which is $\nu(B)$ or
    $\nu(A) \to \nu(B)$ depending on $A$ being a type or a kind.

    If $A$ is a kind, then $\dbrack{a}_\xi \in \nu(A)$ exists and the function
    application in the definition is meaningful.

    In any case $\dbrack{f a}_\xi \in \nu(B)$ which together with $\nu(B) =
    \nu(C)$ proves the required property of the function.


  \item[Abstraction]
    $$
    \ruleh
    {\Gamma \vdash \lambda x^A. e: C \quad C \in \Kinds \quad \xi \vDash \Gamma}
    {\dbrack{\lambda x^A. e}_\xi
      :=
      \begin{cases}
        \dbrack{e}_\xi & A \in \Types
        \\
        M \mapsto \dbrack{e}_{\xi,x:M} & A \in \Kinds
      \end{cases}
    }
    $$
    The generation lemma guarantees the existence of $A,B,s$ with $\Gamma
    \vdash \Pi x^A.B : s$, $\Gamma,x:A \vdash e:B$ and $\Pi x^A.B \le
    C$. Since $C$ is a kind, the product and $B$ must be kinds as
    well. Furthermore $\nu(\Pi x^A.B) = \nu(C)$ is valid.

    If $A$ is a type, then $\xi \vDash \Gamma,x:A$ and therefore
    $\dbrack{e}_\xi \in \nu(B)$. By $\nu(B) = \nu(\Pi x^A.B) = \nu(C)$ the
    goal is proved.

    If $A$ is a kind, then $\xi,x:M \vDash \Gamma,x:A$ when $M \in
    \nu(A)$. Furthermore we have $\nu(C) = \nu(A) \to \nu(B)$, so
    $M \mapsto \dbrack{e}_{\xi,x:M}$ is a correct value for
    $\dbrack{\lambda x^A. e}_\xi$

  \item[Product]
    $$\ruleh
    {\Gamma\vdash \Pi x^A . B : s \quad s \in \Kinds \quad \xi \vDash \Gamma}
    {\typint{\Pi x^A . B}{\xi}
      =
      \typint{A}{\xi} \tolam
      \begin{cases}
        \typint{B}{\xi} & A \in \Types
        \\
        \cap_{M \in \nu(A)} \typint{B}{\xi,x:M}
        & A \in \Kinds
      \end{cases}
    }$$

    From the generation lemma we conclude the existence of $s_1,s_2$ with
    $\Gamma \vdash A:s_1$, $\Gamma,x:A \vdash B:s_2$ and
    $s_1 = s_2 \lor s_2 \text{ minimal}$. Since $s_1,s_2,s$ are sorts
    $\nu(s_1) = \nu(s_2) = \nu(s) = \SAT$ is valid and we get $\dbrack{A}_\xi
    \in \SAT$.

    \begin{itemize}
    \item $A \in \Types$: In that case we get $\xi \vDash \Gamma,x:A$ and
      $\dbrack{B}_\xi \in \SAT$ and
      $\dbrack{A}_\xi \tolam \dbrack{B}_\xi \in \SAT$.

    \item $A \in \Kinds$: In that case we get $\xi,x:M \vDash \Gamma,x:A$ for
      all $M \in \nu(A)$ and $\dbrack{B}_{\xi,x:M} \in \SAT$.
      %
      Therefore $\cap_{M \in \nu(A)} \dbrack{B}_{\xi,x:M} \in \SAT$ ($\SAT$ is
      a complete lattice) and
      %
      $\dbrack{A}_\xi \tolam \cap_{M \in \nu(A)} \dbrack{B}_{\xi,x:M} \in
      \SAT$ as well.
    \end{itemize}


  \end{description}
\end{definition}


\begin{lemma}
  $$
  \ruleh
  {\Gamma \vdash t : T
    \quad
    \xi \vDash \Gamma
  }
  {\dbrack{T}_\xi \in \SAT}
  $$
  \begin{proof}
    From $\Gamma \vdash t : T$ we conclude that $T$ is either a sort or there
    is some sort $s$ wuch that $\Gamma \vdash T:s$ is valid.

    In the first case we get by definition $\dbrack{T}_\xi = \SN$ which is in
    $\SAT$.

    In the second case we conclude $\dbrack{T}_\xi \in \nu(s)$ where $\nu(s) =
    \SAT$ by definition.
  \end{proof}
\end{lemma}


\subsection{Term Interpretation}

\subsection{Soundness Lemma}

\begin{theorem}
  $$
  \ruleh
  {\Gamma \vdash t : T
    \quad
    \rho,\xi \vDash \Gamma
  }
  {\banana{t}_\rho \in \dbrack{T}_\xi}
  $$
\end{theorem}



%%% Local Variables:
%%% mode: latex
%%% TeX-master: "0typed"
%%% End:
