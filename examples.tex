\section{Programming}


\subsection{Boolean}

\begin{ocaml}
  module Bool =
    struct
      type t = True | False

      let true:  t = True
      let false: t = False
      ...
    end
\end{ocaml}

Recipe for a folding functions
%
\begin{enumerate}

\item We need one function (or constant) per constructur

\item The function corresponding to a constructor uses one argument for
  each argument of the constructor and returns the desired value.
\end{enumerate}

\begin{ocaml}
  let fold (b:t) (f1:'a) (f2:'a): 'a =
     match a with
     | True -> f1
     | False -> f2
\end{ocaml}

We can write this more explicitly

\begin{ocaml}
  let fold (type a) (b:t) (f1:a) (f2:a): a =
     match a with
     | True -> f1
     | False -> f2
  (* type of fold: t -> a -> a -> a *)
\end{ocaml}


We represent an object of an algebraic data type as a lambda term whose type
has the type of the folding function without the object of the type. The type
of the folding function without the object type is \code{a -> a -> a}. The
variable \code{a} can be replaced by any type. The variable \code{a} is the
first argument of the function.


Therefore we get as the type of boolean values in lambda calculus
$$
    \Bool := \Pi \alpha^*. \alpha \to \alpha \to \alpha
$$

Each constructor has the type of the folding function. For the 2 constructors
we get the lambda terms
%
$$
\begin{array}{lllll}
  \true
  &:=& \lambda \alpha^* x^\alpha y^\alpha . x
  &:& \Pi \alpha^*. \alpha \to \alpha \to \alpha
  \\
  \false
  &:=& \lambda \alpha^* x^\alpha y^\alpha . y
  &:& \Pi \alpha^*. \alpha \to \alpha \to \alpha
\end{array}
$$
%
There is some redundancy in the notation. Having the type of an expression it
is easy to \emph{infer} some types. Therefore we can express the same with
%
$$
\begin{array}{lllll}
  \true
  &:=& \lambda \alpha x y . x
  &:& \Pi \alpha^*. \alpha \to \alpha \to \alpha
  \\
  \false
  &:=& \lambda \alpha x y . y
  &:& \Pi \alpha^*. \alpha \to \alpha \to \alpha
\end{array}
$$
%
In the following we often leave out type annotations if the types can be
infered from the context.


In untyped lambda calculus the boolean value true is represented by the lambda
term $\lambda x y. x$ and the value false is represented by $\lambda x y
.y$. In typed lambda calculus we need an extra type argument to make sure that
both arguments have the same type.


Now lets implement some boolean functions. E.g. negation in ocaml looks like
\begin{ocaml}
  let negate (a:t): t =
     match a with
     | True  -> False
     | False -> True
\end{ocaml}
%
However we want to implement functions using the folding function in order to
translate it into the lambda calculus.
%
\begin{ocaml}
  let not_ (a:t): t =
     fold a False True
\end{ocaml}

$$
\text{not}
:= \lambda a \alpha .
  a \, \alpha \, (\false\,\alpha) \, (\true \, \alpha)
: \Bool \to \Bool
$$

Let's compute $\text{not} \, \true$
$$
%
\begin{array}{lll}
  \text{not} \, \true
  &=
  & (\lambda a \alpha .
    a \, \alpha \, (\false\,\alpha) \, (\true \, \alpha)) \,
    \true
  \\
  & \reduce
  & \lambda \alpha .
    \true \, \alpha \, (\false\,\alpha) \, (\true \,
    \alpha)
  \\
  & \reduce
  & \lambda \alpha. \false\,\alpha
  \\
  & =
  & \lambda \alpha . (\lambda \beta x y. y) \, \alpha
  \\
  & \reduce & \lambda \alpha . \lambda x y . y
  \\
  & = & \lambda \alpha x y . y
  \\
  & = & \false
\end{array}
$$


\begin{ocaml}
  let and_ (a:t) (b:t): t =
    match a with
    | True -> b
    | False -> False

  (* with fold *)
  let and_ (a:t) (b:t): t =
    fold a b False
\end{ocaml}

The same in lambda calculus
$$
\andb
:= \lambda a b \alpha. a \, (b\, \alpha) (\false\, \alpha)
: \Bool \to \Bool \to \Bool
$$


The function \code{or}:
\begin{ocaml}
  let or_ (a:t) (b:t): t =
    fold a True b
\end{ocaml}
%
and in lambda calculus
$$
\orb
:= \lambda a b \alpha. a \, (\true \, \alpha) (b \, \alpha)
: \Bool \to \Bool \to \Bool
$$



Summary:
$$
%
\begin{array}{lllll}
  \Bool
  &:=& \Pi \alpha^* . \alpha\to\alpha\to\alpha
  &:& *

  \\

  \true
  &:=& \lambda \alpha x y . x
  &:& \Bool

  \\

  \false
  &:=& \lambda \alpha x y . y
  &:& \Bool

  \\

  \text{not}
  &:=& \lambda a \alpha .
  a \, \alpha \, (\false\,\alpha) \, (\true \, \alpha)
  &:& \Bool \to \Bool

  \\

  \andb
  &:=& \lambda a b \alpha. a \, (b\, \alpha) (\false\, \alpha)
  &:& \Bool \to \Bool \to \Bool

  \\

  \orb
  &:=& \lambda a b \alpha. a \, (\true \, \alpha) (b \, \alpha)
  &:& \Bool \to \Bool \to \Bool
\end{array}
$$



\subsection{Pair}

\begin{ocaml}
  module Pair =
    struct
      type ('a,'b) t = Pair of 'a * 'b

      let pair (a:'a) (b:'b): ('a,'b) t =
        Pair (a,b)

      ...
    end
\end{ocaml}

\begin{ocaml}
  let fold (p: ('a,'b) t) (f:'a -> 'b -> 'c): 'c =
    match p with
    | Pair (a,b) -> f a b
\end{ocaml}

\begin{ocaml}
  let first (p: ('a,'b) t): 'a =
    fold p (fun a -> b -> a)

  let second (p: ('a,'b) t): 'b =
    fold p (fun a -> b -> b)
\end{ocaml}

\begin{ocaml}
  let swap (p: ('a,'b) t): ('b,'a) t =
    fold p (fun a -> b -> Pair(b,a))
\end{ocaml}

$$
%
\begin{array}{lllll}
  \Pair
  &:=
  & \lambda \alpha^* \beta^* .
    \Pi \gamma^* . (\alpha \to \beta \to \gamma)
    \to \gamma
  &:
  & * \to * \to *

  \\

  \pair
  &:=
  & \lambda \alpha \beta a b . \lambda \gamma f . f \, a \, b
  &:
  & \Pi \alpha^* \beta^* . \alpha \to \beta \to \Pair\alpha\beta

  \\

  \pi_1
  &:=
  & \lambda \alpha \beta p. p \alpha (\lambda a b .a)
  &:
  &\Pi \alpha^* \beta^* . \Pair\alpha\beta \to \alpha

  \\

  \pi_2
  &:=
  & \lambda \alpha \beta p. p \alpha (\lambda a b .b)
  &:
  &\Pi \alpha^* \beta^* . \Pair\alpha\beta \to \beta

  \\

  \text{swap}
  &:=
  & \lambda\alpha\beta p.
    p (\Pair\beta\alpha) (\lambda a b . \pair\beta\alpha b a)
  &:
  &\Pi \alpha^* \beta^* . \Pair\alpha\beta \to \Pair\beta\alpha
\end{array}
$$


\subsection{Natural Number}

\begin{ocaml}
  module Nat =
    struct
      type t =
      | Zero
      | Succ of t

      let zero: t =
        Zero

      let succ (n:t): t =
        Succ(n)
      ...
    end
\end{ocaml}


\begin{ocaml}
  let rec fold (n:t) (a:'a) (f:'a -> 'a): 'a =
    match n with
    | Zero -> a
    | Succ p -> f (fold p a f)
\end{ocaml}

\begin{ocaml}
  let (+) (a:t) (b:t): t =
    fold a b succ
\end{ocaml}

\begin{ocaml}
  let ( * ) (a:t) (b:t): t =
    fold a zero ((+) b)
\end{ocaml}


\begin{ocaml}
  let is_zero (a:t): Bool.t =
    fold a Bool.true (fun _ -> Bool.false)
\end{ocaml}


\begin{ocaml}
  (* 'recurse' with the help of 'fold' *)
  let recurse (n:t) (a:'a) (f:'a -> t -> 'a): 'a =
    (* f computes from the result of the predecessor and the predecessor
       the result of the next number. *)
    Pair.first (fold n (Pair.pair a zero) (fun (a,p) -> f a p, succ p))

  (* 'recurse' directly as recursive function *)
  let rec recurse (n:t) (a:'a) (f:'a -> t -> 'a): 'a =
    match n with
    | Zero -> a
    | Succ p -> f (recurse p a f) p
\end{ocaml}

\begin{ocaml}
  let predecessor (n:t): t =
    recurse n zero (fun res p -> p)
\end{ocaml}

\begin{ocaml}
  let factorial (n:t): t =
    recurse
      n
      (succ zero)     (* start value 0! = 1 *)
      (fun res p ->   (* res = p! *)
         (succ p) * res)
\end{ocaml}




\section{Examples}


\paragraph{Booleans}
$$
\begin{array}{lllll}
  \text{Bool}
  &:=& \Pi \alpha^* . \alpha\to\alpha\to\alpha
  &:& *
  \\
  \text{true}
  &:=& \lambda \alpha^* x^\alpha y^\alpha . x
  &:& \text{Bool}
  \\
  \text{false}
  &:=& \lambda \alpha^* x^\alpha y^\alpha . y
  &:& \text{Bool}
\end{array}
$$


\paragraph{Pairs}

$$
\begin{array}{lllll}
  P
  &:=& \lambda \alpha^* \beta^*
       . \Pi \gamma^*
       . (\alpha\to\beta\to\gamma) \to \gamma
  &:& * \to * \to *
  \\
  \text{pair}
  &:=& \lambda
       \alpha^*
       \beta^*
       a^\alpha
       b^\beta
       \gamma^*
       f^{\alpha\to\beta\to\gamma}
       . f a b
  &:& \Pi \alpha^* \beta^* a^\alpha b^\beta . P \alpha \beta
  \\
  \pi_1
  &:=& \lambda
       \alpha^*
       \beta^*
       p^{P\alpha\beta}
       . p \alpha (\lambda a^\alpha b^\beta. a)
  &:& \Pi \alpha^* \beta^* (P\alpha\beta)^* . \alpha
  \\
  \pi_2
  &:=& \lambda
       \alpha^*
       \beta^*
       p^{P\alpha\beta}
       . p \beta (\lambda a^\alpha b^\beta. b)
  &:& \Pi \alpha^* \beta^* (P\alpha\beta)^* . \beta
\end{array}
$$



\paragraph{Natural numbers}
$$
\begin{array}{lllll}
  \Nat
  &:= & \Pi \alpha^* . (\alpha \to \alpha) \to \alpha \to \alpha
  &:& *
  \\
  0
  &:=& \lambda \alpha^* f^{\alpha\to\alpha} z^\alpha. z
  &:& \Nat
  \\
  (')
  &:=& \lambda p^\Nat \alpha^* f^{\alpha\to\alpha} z^\alpha . f (p \alpha f z)
  &:& \Nat\to\Nat
\end{array}
$$

Primitiv recursion is done with two functions $g:\tau$ and
$h:\Nat \to \tau \to \tau$. We need a step function which takes the type
$\tau$ and the function $h$ and returns a function with the type
$\text{Pair}\,\Nat\,\tau \to \text{Pair}\,\Nat\,\tau$.
$$
\begin{array}{lll}
  \text{start}
  &:=& \lambda
       \alpha^*
       g^\alpha
       . \text{pair}\,\Nat\,\alpha\,0\,g
  \\
  & &: \Pi \alpha^*. \alpha \to P\Nat\alpha
  \\
  \text{step}
  &:=& \lambda
       \alpha^*
       h^{\Nat\to\alpha\to\alpha}
       p^{P\Nat\alpha}
       . \text{pair}\,\Nat\,\alpha\,
       (\pi_1\,p)'\,
       (h (\pi_1 p) (\pi_2 p)
  \\
  & &: \Pi \alpha^* (\Nat\to\alpha\to\alpha)
      . (P\Nat\alpha)\to (P\Nat\alpha)
\end{array}
$$


Predecessor function
$$
\text{pred} :=
\lambda n^\Nat.
\pi_2 (
  n
  \Nat (\text{step}\,\Nat (\lambda n^\Nat \_^\Nat. n))
  (\text{pair}\,\Nat\,\Nat\,0\,0)
)
$$



\paragraph{Lists}

$$
\begin{array}{lllll}
  L
  &:=& \lambda \alpha^* .
       \Pi \beta^* . (\alpha\to\beta\to\beta) \to \beta \to \beta
  &:& * \to *
  \\
  \text{nil}
  &:=& \lambda \alpha^* \beta^* f^{(\alpha\to\beta\to\beta)} s^\beta . s
  &:& \Pi \alpha^* . L \alpha
  \\
  \text{cons}
  &:=& \lambda
       \alpha^*
       h^\alpha
       t^{L\alpha}
       \beta^*
       f^{(\alpha\to\beta\to\beta)}
       s^\beta
       . f h (t \beta f s)
  &:& \Pi \alpha^* . \alpha \to L \alpha \to L \alpha
\end{array}
$$




\paragraph{Trees}

$$
\begin{array}{lllll}
  T
  &:=& \lambda \alpha^* .
       \Pi \beta^* . (\alpha\to\beta\to\beta\to\beta) \to \beta \to \beta
  &:& * \to *
  \\
  \text{empty}
  &:=& \lambda \alpha^* \beta^* f^{\alpha\to\beta\to\beta\to\beta} e^\beta . e
  &:& \Pi \alpha^* . T \alpha
  \\
  \text{node}
  &:=& \lambda
       \alpha^*
       a^\alpha
       l^{T\alpha}
       r^{T\alpha}
       \beta^*
       f^{\alpha\to\beta\to\beta\to\beta}
       e^\beta
       . f a (l \beta f e) (r \beta f e)
  &:& \Pi \alpha^* . \alpha \to T \alpha \to T \alpha \to T \alpha
\end{array}
$$

%%% Local Variables:
%%% mode: latex
%%% TeX-master: "0typed"
%%% End:
